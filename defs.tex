\usepackage[a4paper,twoside,top=2.6cm,bottom=2.6cm,inner=3cm,outer=2.6cm]{geometry}

\usepackage[utf8]{inputenc}            % kodowanie znaków
\usepackage[polish]{babel}             % język polski
\usepackage[T1]{polski}                % j.w.

\usepackage[font=small,labelfont=bf,justification=centering]{caption}




\input glyphtounicode.tex              % Zaznaczanie tekstu w programach
\pdfgentounicode = 1                   % j.w.

\usepackage[font=footnotesize,labelfont=bf,justification=raggedright,singlelinecheck=false]{caption}

\usepackage{lmodern}


\usepackage{colortbl}
\usepackage{xcolor}
\usepackage{tabularx}
\usepackage{float}
\usepackage{needspace}
\usepackage{fancyhdr}
\usepackage{graphicx}
\usepackage{multicol}
\usepackage{amsmath}
\usepackage{amsthm}
\usepackage{amssymb}
\usepackage{url}
\usepackage{longtable}
\usepackage{array,hhline}
\usepackage{acronym}                   % skróty literowe
\usepackage{fancyvrb}                  % kod komputerowy
\usepackage{hyperref}                  % hiperłącza na komputerach
\usepackage{cleveref}
\usepackage{todonotes}                 % notatki na marginesie
\newcommand\sidenote[1]{\todo[linecolor=lightgray,backgroundcolor=lightgray!25,bordercolor=lightgray]{\raggedright\footnotesize{#1}}}
\newcommand\todonote[1]{\todo[linecolor=red!50,backgroundcolor=lightgray!25,bordercolor=red!50]{\raggedright\footnotesize{#1}}}
\newcommand{\foreign}[2]{(#1. \textit{#2})}

\usepackage{booktabs} \heavyrulewidth=1.5bp \lightrulewidth=0.5bp

\newcommand\allsource[0]{źródło: opracowanie własne; 
                         dane: UPRP, patents.google.com, Lens.org, USPTO}

\let\leq\leqslant\let\le\leq\let\geq\geqslant\let\ge\geq 
                                       % sic! ucywilizowanie znaków niewiększości


\theoremstyle{plain}
\newtheorem{twier}{Twierdzenie}[chapter] % pierwsze to nazwa środowiska,
                                      %drugie to wyświetlana nazwa
				% to trzecie w~nawiasie kwadratowym
				% wskazuje numer dolepiony z~lewej do
				% numeru twierdzenia (tu numer
				% 'chapter', 
\newtheorem{lemat}{Lemat}[chapter]

\crefname{figure}{rys.}{rys.}
\Crefname{figure}{Rys.}{Rys.}

\crefname{table}{tab.}{tab.}
\Crefname{table}{Tab.}{Tab.}

\crefname{section}{sekcja}{sekcje}
\Crefname{section}{Sekcja}{Sekcje}

\theoremstyle{definition}
\newtheorem{defi}{Definicja}[chapter]
\crefname{defi}{def.}{def.}
\Crefname{defi}{Def.}{Def.}

\newcommand{\crefpairconjunction}{ oraz }
\newcommand{\creflastconjunction}{ oraz }

\theoremstyle{definition}
\newtheorem{przyp}{Przypadek}[chapter]

\theoremstyle{definition}
\newtheorem{przykład}{Przykład}[chapter]

\crefname{przyp}{przypadek}{przypadki}
\Crefname{przyp}{Przypadek}{Przypadki}

\theoremstyle{remark}
\newtheorem{uwaga}{Uwaga}[chapter]
\newtheorem{wniosek}{Wniosek}[chapter]


\newcommand{\chart}[2]{
\begin{figure}[H]
\centering
\includegraphics[scale=.64]{#1}
\caption{#2}
\label{#1}
\end{figure}}

\newcommand{\tbl}[2]{
  \begin{table}\centering
    \input{#1}
  \caption{#2}\label{#1}
  \end{table}
}


%%%%% więcej możliwości w~dokumentacji amsthm



%%%%%%%%%%%%%%%%%%%%%%%%%%%%%%%%%%%%%%%%%5
%%%%%%%%%%%%%%%%%%%%%%%%%%%%%%%%%%%%%%%%%%
%%%%%%%%% wcięcie akapitowe %%%%%%%%%%%%%%
%%%%%%%%%%%%%%%%%%%%%%%%%%%%%%%%%%%%%%%%%%
%%%%%% ustawić w~zaleceń i~gustu %%%%%%%%%
%%%%%%%%%%%%%%%%%%%%%%%%%%%%%%%%%%%%%%%%%%
%%%%%%%% zalecenie na stronie wydziałowej
%%%%%%%% było 1.25cm i wyglądało jakoś 
%%%%%%%% śmiesznie duże, więc spłoszony nieco
%%%%%%%% wpisałem 1cm, ale uważny czytelnik już
%%%%%%%% zapewne się domyśli, że podmiana napisu 
%%%%%%%% =1cm na =1.25cm sprawi, że wcięcia na początku
%%%%%%%% akapitu ustawią się na (nieco przydużą)
%%%%%%%% wartość 1.25cm 

\parindent=1cm



%%%%%%%%%%%%%%%%%%%%%%%%%%%%%%%%
%%%%% tu pewne poluzowanie rozmieszczenia elementów tabelek
%%%%% możecie sobie poeksperymentować, by dopasować do swych
%%%%% gustów, a przede wszystkim gustów promotorów (promotorek)
  \tabcolsep=4mm          
  %\renewcommand\arraystretch{1.3}
%%%%%%%%%%%%%%%%%%%%%%%%%%%%%%%%%%



%%%%%%%%% teraz żywa pagina (aka 'running headline') i~numerowanie stron
%%%%%%%%%%%%%%%%%%%%%%%%%%%%%%%%%%%%%%%%%%%%%%%%%%%%%%%%%%%%%%%%%%%%%%%%
%%%%%na górze mają być śródtytuły, na dole (po stronie zewneętrznej)
%%%%%numery stron. Poszedłem kapkę dalej i~na stronach ropoczynających
%%%%%rozdział nie ma paginy (górki).
%%%%% Oczywiście jeśli ostatnia strona
%%%%% jest pusta (uzupełnia jeno parzystość) to tam żadnej stopki ani 
%%%%% górki byc mnie może - ma być pusta.
%%%%%%%%%%%%%%%%%%%%%%%%%%%%
\pagestyle{fancy}
\fancyhead{}% oczyszczenie
\fancyhead[RO]{\rightmark} %% na nieparzystych 'podległe' śródtytuły
\fancyhead[LE]{\leftmark} %% na parzystych 'ważniejsze'
\fancyfoot{}% oczyszczenie
\fancyfoot[RO,LE]{\arabic{page}}  %% numer na dole (po prawej na
%% nieparzystych, po lewej na parzystych)
\renewcommand\headrulewidth{0.4pt} %%% cienka hrulka oddzielająca paginę
                                    %%% od kolumny tekstu
\fancypagestyle{closing}{%%%%%% to styl dla stron zamykających rozdział
\fancyhead{}% oczyszczenie
\fancyhead[RO]{\rightmark} %% na nieparzystych 'podległe'
\fancyhead[LE]{\leftmark} %% na parzystych 'ważniejsze'
\fancyfoot{}% oczyszczenie
\fancyfoot[RO,LE]{\arabic{page}}  %% numer na dole (po prawej na
                                  %% powyższą linijkę usuń jeśli nie
				  %% chcesz numerów na niepełnych
				  %% kolumnach (zamykających rozdział)
\renewcommand\headrulewidth{0.4pt}
}
\fancypagestyle{opening}{%%% styl stron rozpoczynających rozdział
\fancyhead{}% oczyszczenie
\fancyfoot{}% oczyszczenie
\fancyfoot[RO,LE]{\arabic{page}}  %% numer na dole (po prawej na
\renewcommand\headrulewidth{0pt}
}
\fancypagestyle{plain}{%%%% styl zwykły, niektóre konstrukcje
                       %%%% (typu \titlepage, którego ja tu nie używam
                       %%%% ale może są jakieś inne o których nawet nie chce 
                       %%% mi się myśleć, więc dla spokoju robię to po swojemu
\fancyhead{}% oczyszczenie
\fancyfoot{}% oczyszczenie
\fancyfoot[RO,LE]{\arabic{page}}  %% numer na dole (po prawej na
\renewcommand\headrulewidth{0pt}
}

%%%%%%%%%%%%%%%%%%%%%%%%%%%%%%%%%5
%%%%%%%%%%%%%%%%%%%%%%%%%%%%%%%%%%
%%% lekka modyfikcja 'markow' do paginy
%%% uznalem, ze jesli ktos nie da \section (np we wstepnie czy
%%% podsumowaniu to niech na obu sronach w~paginie pojawia sie tytuł
%%% chaptera, bo standardowo, to na nieparzystej stronie w takiej sytuacji
%%% nad górną linią ziałaby pustka, co mogłoby wprowadzać konsternację
\makeatletter
    \def\chaptermark#1{%
      \markboth{%
        \ifHeadingNumbered
     \if@mainmatter
     \@chapapp\
            \thechapter.\enspace
          \fi
        \fi
        #1}{%
        \ifHeadingNumbered
     \if@mainmatter
     \@chapapp\
            \thechapter.\enspace
          \fi
        \fi
        #1%
	}}%
    \def\sectionmark#1{%
      \markright{%
        \ifHeadingNumbered \thesection.\enspace \fi
        #1}}
%%%%%%%%%%%%%%%%%%%%%%%%%%%%%%%%%%%%%%%%%%%%%%%
%%%%%%%%%%%%%%%%%%%%%%%%%%%%%%%%%%%%%%%%%%%%%%%%
%%%%%%%%%%%% wielkości czcionek dla chapter i~section
%%%%%%%%%%%% 16 dla rozdziału, 14 dla podrozdziału - te domyślne
%%%%%%%%%%%% w klasie mwbk były całkiem ładne, ale żeby nie było
%%%%%%%%%%%% że nie potrafię ustawić
%%%%%%%%%%%%%%%%%%%%%%%%%%%%%%%%%%%%%%%%%%%%%%%%%%%
\SetSectionFormatting[breakbefore,wholewidth]{chapter}
        {0\p@}
        {\FormatRigidChapterHeading{6.4\baselineskip}{12\p@}%
	{\large\@chapapp\space}{\fontsize{16}{19}\selectfont}}
        {1.6\baselineskip}
\SetSectionFormatting{section}
        {24\p@\@plus5\p@\@minus2\p@}
	{\FormatHangHeading{\fontsize{14}{16}\selectfont}}
        {10\p@\@plus3\p@}
\makeatother	



%%%%%%%%%%%%%%%%%%%%%%%%%%%%%%%%%%%%%%%%%%%%%%
%%%%%%%%%%%%%%%%%%%%%%%%%%%%%%%%%%%%%%%%%%%%%%
%%%%%%%%%%%%%% jakies inne pomocnicze definicje, ja na przykład lubię
% \R
%%%%%%%%%%%%%%%%%%%%%%%5
%%%%%%%%%%%%%%%%%%%%%%%
%%%% tak naprawdę są t potrzebne tylko po to
%%%% by zadziałały przykłady poniżej w tekście
%%%% które w sposób dość losowy zostały 
%%%% pobrane z jakichś moich starych plików
%%%%%%%%%%%%%%%%%%%%%%%%%%%%%%%%%%
%%%%%%%%%%%%%%%%%%%%%%%%%%%%%%%%%%%
%%%% w realnej pracy te poniższe śmieci możecie oczywiście
%%%% usunąć
%%%%%%%%%%%%%%%%%%%%%%%%%%%%
\newcommand\R{\mathbb{R}}
\newcommand{\ff}{\mathbf{f}}
\newcommand{\hh}{\mathbf{h}}
\newcommand{\xx}{\mathbf{x}}
\newcommand{\yy}{\mathbf{y}}
\newcommand{\zz}{\mathbf{z}}
\newcommand{\gggg}{\mathbf{g}}
\newcommand{\skalar}[2]{\pmb{\langle}#1,#2\pmb{\rangle}}
%%%%%%%%%%%% koniec tych dodatkowych definicji

%%%%%% trocę więcej ``luzu'' przy rozmieszczaniu {fgur} i~{table}

 \renewcommand{\topfraction}{0.9}	% max fraction of floats at top
    \renewcommand{\bottomfraction}{0.8}	% max fraction of floats at bottom
    %   Parameters for TEXT pages (not float pages):
    \setcounter{topnumber}{2}
    \setcounter{bottomnumber}{2}
    \setcounter{totalnumber}{4}     % 2 may work better
    \setcounter{dbltopnumber}{2}    % for 2-column pages
    \renewcommand{\dbltopfraction}{0.9}	% fit big float above 2-col. text
    \renewcommand{\textfraction}{0.07}	% allow minimal text w. figs
    %   Parameters for FLOAT pages (not text pages):
    \renewcommand{\floatpagefraction}{0.7}	% require fuller float pages
    % N.B.: floatpagefraction MUST be less than topfraction !!
    \renewcommand{\dblfloatpagefraction}{0.7}	% require fuller float pages
    % remember to use [htp] or [htpb] for placement

    
%%% DWA proste polecenia służące do ujednolicenia podawania źródeł przy rysunkach i~tabelkach    
    
    \newcommand\zrodlo[1]{\par\vspace{-3mm}{\small\textit{Źródło: }#1 }}
    \newcommand\zrodlotab[1]{{\par\vspace{2mm}\small\textit{Źródło: }#1 }}

\raggedbottom   %%% to znaczy, że nie będzie siłowego wyrównywania typowych
                %%     stron do jednakowej wysokości

\linespread{1.3}





\newcommand{\D}[3]{
  \begin{defi}
    \textbf{#2} --- #3
    \label{#1}
    \end{defi}
}


\newcommand{\TODO}[1]{\textcolor{white}{\colorbox{red}{TODO: \texttt{#1}}}}