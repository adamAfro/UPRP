  \section{Metryki i statystyki przestrzenne}


\subsection{Autokorelacja przestrzenna}

Autokorelacja przestrzenna mówi o braku losowości w położeniu punktów:
jeśli autokorelacja jest statystycznie istotna to punkty są skupione w 
globalne klastry przestrzenne\cite{Se-Ar-Da-Wo-20}.
W kontekście analizy patentowej, globalność odnosi się do poziomu krajowego.

\D{spat-weight}{Wagi przestrzenne}
{ macierz liczbowych wag przestrzennych $W$,
  która określa siłę zależności przestrzenne między obiektami.
  Indeks rzędu macierzy $W$ odnosi się do obiektu przestrzennego,
  a indeks kolumny do obiektu co do którego występuje zależność.}

Zależność wag przestrzennych w poniżeszej analizie dotyczy 
konkretnie tego jak blisko siebie są obiekty przestrzenne ---
wynika wyłącznie z położenia i sąsiedztwa regionów.
Sąsiedztwo jest określane zgodnie z metodą \foreign{queen contiguity}
--- znaczy to tyle, że dwa regiony sąsiadują ze sobą
jeśli mają wspólną granicę lub punkt.

\D{lag}{Lag}{
  Macierz $Y_{l}$ jest iloczynem wag przestrzennych $W$ i wektora cechy $Y$
  dla każdej obserwacji przestrzennej. Lag jest miarą zależności przestrzennej
  w sąsiedztwie.
  \begin{math}
    Y_{l} = W\cdot Y
  \end{math}}

\D{Moran-I}{Statystyka I Morana}{
\begin{math}
  I = (n / \sum_i \sum_j w_{ij})\cdot (\sum_i \sum_j w_{ij} z_i z_j / \sum_i z_i^2)
\end{math} \TODO{wyjaśnienia}}