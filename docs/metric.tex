  \section{Metryki i statystyki przestrzenne}


\subsection{Średni dystans}

Średni dystans jest miarą odległości między wszystkimi punktami w przestrzeni.
Każdy punkt ma przyporządkowaną wagę równą ilości osób pełniących role patentowe,
które meldowały się w danym punkcie podczas składania aplikacji patentowej.

\D{mean-dist}{Średni dystans}{
  \begin{math}
    \bar{d} = \frac{1}{n(n-1)} \sum_{i=1}^{n} \sum_{j=1}^{n} d_{ij}
  \end{math}
  gdzie $d_{ij}$ to odległość między punktami $i$ i $j$.
}

Średni dystans jest wskaźnikiem centralności danego punktu względem
wszystkich innych punktów. Co za tym idzie jest to miara o poziomie krajowym.
Aby wyznaczyć inne poziomy centralności warto ograniczyć ją do maksymalnego
promienia --- wtedy otrzymamy średni dystans do innych punktów w danym promieniu.
Świadczy on jak bliskości danego punktu do innych w ograniczonym obszarze.

Obszary o wysokiej wartości średniego dystansu określamy jako peryferyjne krajowo,
a te o niskiej jako centralne. W przypadku ograniczonych promienii metryki mamy
do czynienie z bardziej szczegółową informacją o centralności danego punktu.
Punkty o wysokiej wartości średniego dystansu w ograniczonym promieniu określamy
zwyczajnie, jako peryferyjne, z racji nie są w bliskim sąsiedztwie z innymi punktami;
z kolei punkty o niskiej wartości średniego dystansu w ograniczonym promieniu
określamy jako centralne lokalnie.



\subsection{Autokorelacja przestrzenna}

Autokorelacja przestrzenna mówi o braku losowości w położeniu punktów:
jeśli autokorelacja jest statystycznie istotna to punkty są skupione w 
globalne klastry przestrzenne\cite{Se-Ar-Da-Wo-20}.
W kontekście analizy patentowej, globalność odnosi się do poziomu krajowego.

\D{spat-weight}{Wagi przestrzenne}
{ macierz liczbowych wag przestrzennych $W$,
  która określa siłę zależności przestrzenne między obiektami.
  Indeks rzędu macierzy $W$ odnosi się do obiektu przestrzennego,
  a indeks kolumny do obiektu co do którego występuje zależność.}

Zależność wag przestrzennych w poniżeszej analizie dotyczy 
konkretnie tego jak blisko siebie są obiekty przestrzenne ---
wynika wyłącznie z położenia i sąsiedztwa regionów.
Sąsiedztwo jest określane zgodnie z metodą \foreign{queen contiguity}
--- znaczy to tyle, że dwa regiony sąsiadują ze sobą
jeśli mają wspólną granicę lub punkt.

\D{lag}{Lag}{
  Macierz $Y_{l}$ jest iloczynem wag przestrzennych $W$ i wektora cechy $Y$
  dla każdej obserwacji przestrzennej. Lag jest miarą zależności przestrzennej
  w sąsiedztwie.
  \begin{math}
    Y_{l} = W\cdot Y
  \end{math}}

\D{Moran-I}{Statystyka I Morana}{
\begin{math}
  I = (n / \sum_i \sum_j w_{ij})\cdot (\sum_i \sum_j w_{ij} z_i z_j / \sum_i z_i^2)
\end{math}}


  \subsection
{Udział klasyfikacji \ac{IPC} w profilu punktu}\label{udział-klasyfikacji}

Udział klasyfikacji odnosi się do tego jak wiele osób dostawało 
ochronę patentową w danej sekcji ze wskazanego punktu, 
bądź współpracowało przy takim dokumencie z innymi osobami.