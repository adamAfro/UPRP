Raporty \ac{PDF} nie posiadają adnotacji tekstowych. 
Znaczy to tyle, że dane są zawarte w sposób czytelny
jedynie dla człowieka i nie są dostępne dla urządzeń w sposób
ustruktoryzowany inny niż ciąg binarny pikseli.
Rozwiązaniem jest proces \ac{OCR}, który obrazy zawierające tekst
przekształca na kod binarny, które można przetwarzać na komputerze
jako ciągi znaków odpowiadające prawdziwemu tekstowi. Pierwszą
czynnością w tym procesie jest zastosowanie pakietu \textit{paddle}.
Zastosowanie modułu pozwala na pozyskanie linijek tekstu z przypisaniem
do ich pozycji. Wynika z tego problem taki, że nie brak jest informacji
o tym gdzie zaczyna i kończy się tekst dotyczący wskazanej obserwacji.
W związku z tym nie sposób jest przypisać tekstu do odpowiednich
wpisów. Dodatkowo dochodzą problemy wynikające z błędów w procesie
skanowania samych dokumentów - zniekształcenia, zaciemnienia, czy
rotacje kartek sprawiają, że proces \ac{OCR} nie jest idealny.
Dodatkowo samo formatowanie nierzadko jest wadliwe co wynika
z wprowadzania danych jeszcze na etapie tworzenia dokumentów.



\subsubsection{Zastosowanie dużego modelu językowego}

Do skutecznego pozyskania danych z dokumentów kluczowe było zastosowanie
dużych modeli językowych z multimodalnymi wejściami. Stan tej technologii
na dzień procesu wyciągania danych był na tyle zaawansowany, że
aspekty techniczne ograniczają się do zastosowania zewnętrznego \ac{API}
dla modelu \textit{openai} \textit{GPT4o}. Model ten w wystarczający
sposób był w stanie przetworzyć obrazy zawierające tekst na ustruktoryzowany
zbiór wpisów tekstowych.

Mimo, że model \textit{paddle} nie dawał wyników pozwalających na
poprawną dalszą analizę to pozwolił na ograniczenie kosztów. Znalezienie
słów kluczowych nagłówków i stopek tabeli z informacjami było wystarczające
aby przyciąć zdjęcia do obszarów zainteresowania.
