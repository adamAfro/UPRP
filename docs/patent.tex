\section{Dane patentowe}

W danych ze zbioru \ac{UPRP} można wyróżnić kilka kategorii informacji
związanych z badanym tematem. Są to daty związane z procesem urzędowym
patentu, klasyfikacje patentowe, dane przestrzenne oraz dane osobowe
osób związanych z patentami.

\subsection{Dane przestrzenne}

Dane przestrzenne odnoszą się do miejsc z jakimi są powiązane
osoby albo organizacje związane z patentami. Pozostawia to więc 
różne możliwości analizy przestrzennej:

\begin{enumerate}
\item[$A$:] przypisanie każdego patentu do pojedynczej lokalizacji;
\item[$B$:] przypisanie patentu do wielu lokalizacji.
\end{enumerate}

W przypadku $A$ powstaje problem przypisania głównej lokalizacji.
Jest to kwestia $A_1$ priorytetowania organizacji ponad osoby, bądź
odwrotnie, oraz $A_2$ wyboru głównej osoby/organizacji.
Problem $A_1$ wiąże się z potencjalnymi różnicami w modelu zależnie
od wybranego podejścia. Problem $A_2$ może być niejednoznaczny
w rozwiązaniu z powodu zbyt małego zakresu informacji zawartych w danych.
Wymagałoby to dodatkowych danych z samego procesu powstawania wynalazku,
co jest poza zakresem tej pracy.

Dalsza analiza odnosi się wyłącznie do podejścia $B$.
Patent może mieć więc kilka lokalizacji, żadna nie jest określona jako główna.

Problemem jest także to, że dane zawierają wyłącznie nazwy miast, 
więc ich umieszczenie na mapie wiąże się z wadami: 
nazwy nie są unikalne --- w takim przypadku używany jest algorytm minimalizacji 
odległości, tak żeby wybrać optymalną kombinację. 
Wynika z tego oczywiste obicążenie. 
Po za tym nazwy mniejszych miejscowości mogą być duplikatami
nazw miast. Pewne powiązanie jest w takim przypadku niemożliwe
w wykonalny sposób na podstawie patentowych informacji.
Co za tym idzie rozważane są wyłącznie lokacje uważane za polskie miasta.
\todonote{przypis}


\newpage

Opis tego jak uzupełnione są braki danych znajduje się w kolejnej sekcji.
Obok znajduje się wykres ilustrujący ilość patentów z geolokalizacjiami,
zgodnie z tym jak zostały określone.

\fig{../fig/rgst/F-geoloc-eval-appl.png}
{ Stan uzupełnienia informacji o geolokalizacjach, w Polsce, 
  osób i organizacji  pełniących role patentowe
  w aplikacjach patentowych}

\fig{../fig/rgst/F-geoloc-eval-grant.png}
{ Stan uzupełnienia informacji o geolokalizacjach, w Polsce, 
  osób i organizacji  pełniących role patentowe
  w aplikacjach patentowych, które otrzymały ochronę}

\newpage

\figside{../fig/endo/M.png}
{Mapa rozrzutu geolokalizacji osób pełniących role patentowe}

\newpage

\subsection{Rejestr dat związanych z patentami}

Poszczególne czynności związane z ochroną patentów są rejestrowane.
Każde wydarzenie jest powiązane z konkretną datą kalendarzową.
Można wyróżnić kilka typów wydarzeń związanych z patentami:

\begin{itemize}
\item publiczne ujawnienie \foreign{ang}{exhibition}
\item roszczenia z pierwszeństwa \foreign{ang}{priority claim} --- 
      data rozszczenia sprzed rozpoczęcia procesu patentowania dla wybranego urzędu
\item regionalna deklaracja \foreign{ang}{regional filing}\todonote{potrz. wyjaśniene}
\item deklaracja \foreign{ang}{filing}\todonote{potrz. wyjaśniene}
\item aplikacja \foreign{ang}{application} --- data złożenia aplikacji
\item przyznanie ochrony \foreign{ang}{grant}
\item decyzja urzędowa
\item publikacja
\end{itemize}
\todonote{potrz. źródło}

\newpage

\figpage{0.8}{../fig/patt/F-UPRP-event.png}
  {Wydarzenia związane z patentami w kolejnych latach}

Wykres obrazuje kolejne lata i to jakie
działania podejmował urząd w stosunku do składanych patentów.
\todonote{analiza wykresu}

\newpage



\newpage
\subsection{Klasyfikacje patentów}

Klasyfikacje patentowe to systemy, które pozwalają na przypisanie
patentów do odpowiednich dziedzin.

\subsubsection{Międzynarodowa Klasyfikacja Patentów}

W Polsce funkcjonuje klasyfikacja
\ac{MKP}, czyli \ac{IPC}. Zapis klasyfikacji w tym systemie to ciąg
cyfrowo-literowy składający się z 4 części:

\begin{enumerate}
    \item Dział - najwyższa hierarchia złożona z 8 kategorii
    \begin{itemize}
        \item ma tytuł informacyjny
        \item każdy tytuł działu ma swój symbol: A, B, C, D, E, F, G albo H
        \begin{itemize}
            \item A – podstawowe potrzeby ludzkie
            \item B – różne procesy przemysłowe; transport
            \item C – chemia; metalurgia
            \item D – włókiennictwo; papiernictwo
            \item E – budownictwo; górnictwo
            \item F – budowa maszyn; oświetlenie; ogrzewanie; uzbrojenie; technika minerska
            \item G – fizyka
            \item H – elektrotechnika
        \end{itemize}
        \item poddział - każdy dział może zawierać poddział, który nie jest oznaczany symbolem
    \end{itemize}
    \item Klasa - drugi poziom hierarchii
    \begin{itemize}
        \item ma tytuł informacyjny
        \item oznaczana przez liczbę 2-cyfrową
        \item zakres klasy - skrótowa informacja o treści klasy
    \end{itemize}
    \item Podklasa - trzeci poziom hierarchii
    \begin{itemize}
        \item ma tytuł informacyjny
        \item oznaczana dużą literą
        \item ma zakres i tytuł pomocniczy
    \end{itemize}
    \item Grupa - czwarty poziom hierarchii
    \begin{itemize}
        \item 2 zestawy cyfr oddzielone ukośnikiem
        \begin{itemize}
            \item zestaw pierwszy składa się od 1 do 3 cyfr i określa grupę główną
            \item zestaw drugi składa się z 2 cyfr i określa grupę pomocniczą, grupa główna jest oznaczana 00
        \end{itemize}
        \item grupa ma tytuł informacyjny, podgrupa ma bardziej szczegółowe hasło
    \end{itemize}
\end{enumerate}



\subsubsection{Wspólna Klasyfikacja Patentów}

\ac{CPC} to klasyfikacja stworzona przez \ac{EPO} i \ac{USPTO}
jako wspólny wysiłek w kierunku standaryzacji klasyfikacji.

\begin{enumerate}
    \item Sekcja - najwyższa hierarchia złożona z 9 kategorii
    \begin{itemize}
        \item ma tytuł informacyjny
        \item każda sekcja ma swój symbol: A, B, C, D, E, F, G, H albo Y
        \begin{itemize}
            \item A – Potrzeby ludzkie
            \item B – Różne operacje przemysłowe; Transport
            \item C – Chemia; Metalurgia
            \item D – Tekstylia; Papier
            \item E – Budownictwo; Górnictwo
            \item F – Mechanika; Oświetlenie; Ogrzewanie; Broń; Wybuchy
            \item G – Fizyka
            \item H – Elektryczność
            \item Y – Ogólne technologie lub zastosowania specyficzne dla nowych lub pojawiających się dziedzin technologii
        \end{itemize}
    \end{itemize}
    \item Klasa - drugi poziom hierarchii
    \begin{itemize}
        \item ma tytuł informacyjny
        \item oznaczana przez liczbę 2-cyfrową
        \item zakres klasy - skrótowa informacja o treści klasy
    \end{itemize}
    \item Podklasa - trzeci poziom hierarchii
    \begin{itemize}
        \item ma tytuł informacyjny
        \item oznaczana dużą literą
        \item ma zakres i tytuł pomocniczy
    \end{itemize}
    \item Grupa - czwarty poziom hierarchii
    \begin{itemize}
        \item 2 zestawy cyfr oddzielone ukośnikiem
        \begin{itemize}
            \item zestaw pierwszy składa się od 1 do 3 cyfr i określa grupę główną
            \item zestaw drugi składa się z 2 cyfr i określa grupę pomocniczą, grupa główna jest oznaczana 00
        \end{itemize}
        \item grupa ma tytuł informacyjny, podgrupa ma bardziej szczegółowe hasło
    \end{itemize}
\end{enumerate}



\subsubsection{Rewizja \ac{IPC}}

\ac{IPCR} powstał przez rewizję systemu \ac{IPC}. Zawiera dodatkowe
symbole dodające prezycji do klasyfikacji.



\begin{acronym}
  \acro{UPRP}{Urząd Patentowy Rzeczypospolitej Polskiej}
  \acro{WUP}{Wiadomości Urzędu Patentowego}
  \acro{BUP}{Biuletyn Urzędu Patentowego}
  \acro{EPO}{European Patent Office}
  \acro{WIPO}{World Intellectual Property Organization}
  \acro{MKP}{Międzynarodowa Klasyfikacja Patentów}
  \acro{IPC}{International Patent Classification}
  \acro{IPCR}{International Patent Classification Revision}
  \acro{API}{Application Programming Interface}
  \acro{URI}{Uniform Resource Identifier}
  \acro{URL}{Uniform Resource Locator}
  \acro{OCR}{Optical Character Recognition}
  \acro{XML}{Extensible Markup Language}
  \acro{CPC}{Cooperative Patent Classification}
  \acro{USPTO}{United States Patent and Trademark Office}
  \acro{USPG}{United States Patent (and Trademark Office) Grants}
  \acro{USPA}{United States Patent (and Trademark Office) Applications}
  \acro{PGC}{Patents.Google.com}
  \acro{PLO}{Patenty Lens.org}
\end{acronym}