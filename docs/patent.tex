\section{Dane patentowe}

W danych ze zbioru \ac{UPRP} można wyróżnić kilka kategorii informacji
związanych z badanym tematem. Są to daty związane z procesem urzędowym
patentu, klasyfikacje patentowe, dane przestrzenne oraz dane osobowe
osób związanych z patentami.

\subsection{Dane przestrzenne}

Dane przestrzenne odnoszą się do miejsc z jakimi są powiązane
osoby albo organizacje związane z patentami. Pozostawia to więc 
różne możliwości analizy przestrzennej:

\begin{enumerate}
\item[$A$:] przypisanie każdego patentu do pojedynczej lokalizacji;
\item[$B$:] przypisanie patentu do wielu lokalizacji.
\end{enumerate}

W przypadku $A$ powstaje problem przypisania głównej lokalizacji.
Jest to kwestia $A_1$ priorytetowania organizacji ponad osoby, bądź
odwrotnie, oraz $A_2$ wyboru głównej osoby/organizacji.
Problem $A_1$ wiąże się z potencjalnymi różnicami w modelu zależnie
od wybranego podejścia. Problem $A_2$ może być niejednoznaczny
w rozwiązaniu z powodu zbyt małego zakresu informacji zawartych w danych.
Wymagałoby to dodatkowych danych z samego procesu powstawania wynalazku,
co jest poza zakresem tej pracy.

Dalsza analiza odnosi się wyłącznie do podejścia $B$.
Patent może mieć więc kilka lokalizacji, żadna nie jest określona jako główna.

Problemem jest także to, że dane zawierają wyłącznie nazwy miast, 
więc ich umieszczenie na mapie wiąże się z wadami: 
nazwy nie są unikalne --- w takim przypadku używany jest algorytm minimalizacji 
odległości, tak żeby wybrać optymalną kombinację. 
Wynika z tego oczywiste obicążenie. 
Po za tym nazwy mniejszych miejscowości mogą być duplikatami
nazw miast. Pewne powiązanie jest w takim przypadku niemożliwe
w wykonalny sposób na podstawie patentowych informacji.
Co za tym idzie rozważane są wyłącznie lokacje uważane za polskie miasta.
\todonote{przypis}

Mimo powyższych uproszczeń braki danych ciągle są obecne. Ich uzupełnianie
jest opisane w dalszej części pracy. Poniżej znajdują się 2 wykresy. 
Pierwszy przedstawiaja ilość danych dotyczących osób, dla których 
umieszczono informacje o lokalizacji. Widać w nim datę graniczną początkową,
która świadczy o momencie rozpoczęcia rejestrowania danych lokalizacyjnych.
Interesujący jest również spadek w roku 2004. Należy przypuszczać, że wynika
on ze zmiany metodologii zbierania danych osobowych po przystąpieniu
Polski do Unii Europejskiej.
Dla lat 2013-2022, czyli okresu zainteresowania analizy,
widać względną stabilność ilości danych z widocznym
spadkiem ich przyrostu rocznego. 



\begin{figure}[H]\centering
\includegraphics[width=\textwidth]{../registry/NA-loc-geo.png}
\caption{Stan~uzupełnienia~informacji
         o~geolokalizacjach,
         w~Polsce,
         osób~i~organizacji 
         pełniących~role~patentowe 
         w~aplikacjach~patentowych}
\srcuprp
\end{figure}



\Needspace{15\baselineskip}
Opis tego jak uzupełnione są braki danych znajduje się w kolejnej sekcji.
Poniżej znajduje się wykres ilustrujący ilość patentów z geolokalizacjiami,
zgodnie z tym jak zostały określone. Zawiera też informacje o tym, kogo
dotyczyczą te lokalizacje, zawartą na osi poziomej.

\begin{figure}[H]\centering
\includegraphics[width=\textwidth]{../subject/NA-geo-role.png}
\caption{Stan~uzupełnienia~informacji
         o~geolokalizacjach,
         w~Polsce,
         osób~i~organizacji 
         pełniących~role~patentowe 
         w~aplikacjach~patentowych}
\srcuprp
\end{figure}

\Needspace{20\baselineskip}
Poniżej jest mapa tych lokalizacji z liniami 
estymacji gęstości jądrowej. Rozmiar jest zaznaczony kolorystyką 
oraz rozmiarem punktów, gdzie wielkość jest proporcjonalna 
do ilości patentów w danej geolokalizacji.

\begin{figure}[H]\centering
\includegraphics[width=\textwidth]{../subject/map.png}
\caption{Mapa~rozrzutu~geolokalizacji~osób
         pełniących~role~patentowe
         z liniami estymacji gęstości jądrowej}
\srcuprp
\end{figure}

Z wykresu możemy odczytać, że w Polsce istnieją dwa skupiska źródeł
aplikacji patentowych. Jedno w Warszawie, drugie w aglomeracji Katowic oraz
jej okolicach. Większe miasta Polski również charakteryzyją się większą ilością
aplikacji patentowych od reszty kraju. Przykładowo Lublin jest największym miastem
w swoim otoczeniu. Podobnie Kraków, czy Poznań i inne duże miasta.







\subsection{Rejestr dat związanych z patentami}

Poszczególne czynności związane z ochroną patentów są rejestrowane.
Każde wydarzenie jest powiązane z konkretną datą kalendarzową.
Można wyróżnić kilka typów wydarzeń związanych z patentami:

\begin{itemize}
\item publiczne ujawnienie \foreign{ang}{exhibition}
\item roszczenia z pierwszeństwa \foreign{ang}{priority claim} --- 
      data rozszczenia sprzed rozpoczęcia procesu patentowania dla wybranego urzędu
\item regionalna deklaracja \foreign{ang}{regional filing}\todonote{potrz. wyjaśniene}
\item deklaracja \foreign{ang}{filing}\todonote{potrz. wyjaśniene}
\item aplikacja \foreign{ang}{application} --- data złożenia aplikacji
\item publikacja przed przyznaniem ochrony \foreign{ang}{unexamined-printed-without-grant}
\item przyznanie ochrony \foreign{ang}{grant}
\item decyzja urzędowa
\item publikacja
\end{itemize}
\todonote{potrz. źródło}

\begin{figure}[H]\centering
\includegraphics[width=\textwidth]{../api.uprp.gov.pl/event/series.png}
\caption{Wydarzenia związane z patentami w kolejnych latach}
\srcuprp
\label{fig:NA-loc-geo.png}
\end{figure}

Powyższy wykres (\ref{fig:NA-loc-geo.png}) obrazuje kolejne lata i to jakie
działania podejmował urząd w stosunku do składanych patentów. Urząd funkcjonuje
ponad 100 lat\todonote{od którego roku lepiej nap.} i widać to także na wykresie.
Widać na nim historyczne zdarzenia takie jak 2 Wojna Światowa (spadek w latach 39-45) 
oraz upadek ustroju komunistycznego w Polsce w latach (nagłwy wzrost 89-90) oraz
przystąpienie Polski do Unii Europejskiej (spadek w 2004).
Zmiany ustrojowe wiążą się także ze zmianem rozkładu rejestrowanych wydarzeń.
Przed latami 90 rejestrowane były wyłącznie daty aplikacji i grantów.
Następnie zaczęto rejestrować także inne, wcześniej wymienione, 
wydarzenia związane z patentami.


Zmienność danych geolokalizacyjnych oraz datowych jest niewidoczna, co obrazuje
poniższy wykres. W ciągu lat 2013-2022 nie widać znaczących zmian w kolejnych latach.
Można wyróżnić liczności z roku 2015 roku jako względnie wysokie dla Warszawy oraz 
okręgu Katowic. Jak wcześniej zauważono są to główne ośrodki patentowe w Polsce.

\fig{../subject/map-periods.png}{
  Mapa~rozrzutu~geolokalizacji~osób
  pełniących~role~patentowe
  z~liniami~estymacji~gęstości~jądrowej
  w~różnych~okresach~czasu\newline\srcuprp
}

\subsection{Klasyfikacje patentów}

Klasyfikacje patentowe to systemy, które pozwalają na przypisanie
patentów do odpowiednich dziedzin.



\newpage
\subsubsection{Średni dystans do innych osób}

Obszary peryferyjne to miejsca odległe od centrów, w tym przypadku
głównych źródeł aplikacji patentowych. Wyznaczamy je poprzez obliczenie
średniego dystansu do innych osób pełniących role patentowe.
Oprócz dystansu do każdego innego punktu, warto jest ograniczyć
tę statystykę do pewnego obszaru. Jak wcześniej zaznaczono, Polska
nie jest jednorodna przestrzennie pod tym względem, stąd peryferyjność
nie może być rozumiana tylko na krajowym poziomie, ale również lokalnym.

\begin{uwaga}
Średnia odległość po między geolokalizacjami patentowymi jest ważona, gdzie
ilość patentów pochodzących z danej lokalizacji jest wagą.
\end{uwaga}

\fig{../subject/map-geostats-761.png}{
  Średni~dystans~do~innych~osób~pełniących~role~patentowe w~Polsce\newline\srcuprp
\newline Rozmiar --- ilość patentów; kolor --- średni dystans.
}

Rysunek \ref{fig:../subject/map-geostats-761.png} przedstawia średni dystans
do innych osób pełniących role patentowe w Polsce. Widać, że Warszawa
jest najbardziej centralnym punktem, co jest zgodne z wcześniejszymi
obserwacjami. To samo dotyczy okręgu Katowic. Widać jednak, że oprócz
tych aglomeracji istnieją też 2 obszary wyróżniające się od reszty kraju.
Są to obszar łączony Wielkopolski i Dolnego Śląska.\todonote{potrzeb. mapa
z granic. woj.} oraz miasto Łódź. Na podstawie tej statystyki należy przypuszczać,
że są to regiony o względnej bliskości do innych źródeł patentowych na poziomie krajowym.

\fig{../subject/map-geostats-100.png}{
  Średni~dystans~do~innych~osób~pełniących~role~patentowe
  w~Polsce~w~okręgu~o~promieniu~100~kilometrów\newline\srcuprp
\newline Rozmiar --- ilość patentów; kolor --- średni dystans.
}

Na rysunku \ref{fig:../subject/map-geostats-100.png} przedstawiono ten sam
wskaźnik, ale ograniczony do obszaru o promieniu 100 kilometrów.
Widać na nim dużo wyraźniej, że obszary Wielkopolski i Dolnego Śląska
oraz obszar Katowic mają względnie wysoką bliskość. Linie estymacji gęstości
jądrowej pokazują, że inaczej może być dla Warszawy. Mimo liczności patentów
pochodzących z miasta i okolic nie jest ono tak centralne jak mogłoby się
wydawać. 





\subsubsection{Międzynarodowa Klasyfikacja Patentów}

W Polsce funkcjonuje klasyfikacja
\ac{MKP}, czyli \ac{IPC}. Zapis klasyfikacji w tym systemie to ciąg
cyfrowo-literowy składający się z 4 części:

\begin{enumerate}
    \item Dział - najwyższa hierarchia złożona z 8 kategorii
    \begin{itemize}
        \item ma tytuł informacyjny
        \item każdy tytuł działu ma swój symbol: A, B, C, D, E, F, G albo H
        \begin{itemize}
            \item A – podstawowe potrzeby ludzkie
            \item B – różne procesy przemysłowe; transport
            \item C – chemia; metalurgia
            \item D – włókiennictwo; papiernictwo
            \item E – budownictwo; górnictwo
            \item F – budowa maszyn; oświetlenie; ogrzewanie; uzbrojenie; technika minerska
            \item G – fizyka
            \item H – elektrotechnika
        \end{itemize}
        \item poddział - każdy dział może zawierać poddział, który nie jest oznaczany symbolem
    \end{itemize}
    \item Klasa - drugi poziom hierarchii
    \begin{itemize}
        \item ma tytuł informacyjny
        \item oznaczana przez liczbę 2-cyfrową
        \item zakres klasy - skrótowa informacja o treści klasy
    \end{itemize}
    \item Podklasa - trzeci poziom hierarchii
    \begin{itemize}
        \item ma tytuł informacyjny
        \item oznaczana dużą literą
        \item ma zakres i tytuł pomocniczy
    \end{itemize}
    \item Grupa - czwarty poziom hierarchii
    \begin{itemize}
        \item 2 zestawy cyfr oddzielone ukośnikiem
        \begin{itemize}
            \item zestaw pierwszy składa się od 1 do 3 cyfr i określa grupę główną
            \item zestaw drugi składa się z 2 cyfr i określa grupę pomocniczą, grupa główna jest oznaczana 00
        \end{itemize}
        \item grupa ma tytuł informacyjny, podgrupa ma bardziej szczegółowe hasło
    \end{itemize}
\end{enumerate}



\subsubsection{Wspólna Klasyfikacja Patentów}

\ac{CPC} to klasyfikacja stworzona przez \ac{EPO} i \ac{USPTO}
jako wspólny wysiłek w kierunku standaryzacji klasyfikacji.

\begin{enumerate}
    \item Sekcja - najwyższa hierarchia złożona z 9 kategorii
    \begin{itemize}
        \item ma tytuł informacyjny
        \item każda sekcja ma swój symbol: A, B, C, D, E, F, G, H albo Y
        \begin{itemize}
            \item A – Potrzeby ludzkie
            \item B – Różne operacje przemysłowe; Transport
            \item C – Chemia; Metalurgia
            \item D – Tekstylia; Papier
            \item E – Budownictwo; Górnictwo
            \item F – Mechanika; Oświetlenie; Ogrzewanie; Broń; Wybuchy
            \item G – Fizyka
            \item H – Elektryczność
            \item Y – Ogólne technologie lub zastosowania specyficzne dla nowych lub pojawiających się dziedzin technologii
        \end{itemize}
    \end{itemize}
    \item Klasa - drugi poziom hierarchii
    \begin{itemize}
        \item ma tytuł informacyjny
        \item oznaczana przez liczbę 2-cyfrową
        \item zakres klasy - skrótowa informacja o treści klasy
    \end{itemize}
    \item Podklasa - trzeci poziom hierarchii
    \begin{itemize}
        \item ma tytuł informacyjny
        \item oznaczana dużą literą
        \item ma zakres i tytuł pomocniczy
    \end{itemize}
    \item Grupa - czwarty poziom hierarchii
    \begin{itemize}
        \item 2 zestawy cyfr oddzielone ukośnikiem
        \begin{itemize}
            \item zestaw pierwszy składa się od 1 do 3 cyfr i określa grupę główną
            \item zestaw drugi składa się z 2 cyfr i określa grupę pomocniczą, grupa główna jest oznaczana 00
        \end{itemize}
        \item grupa ma tytuł informacyjny, podgrupa ma bardziej szczegółowe hasło
    \end{itemize}
\end{enumerate}



\subsubsection{Rewizja \ac{IPC}}

\ac{IPCR} powstał przez rewizję systemu \ac{IPC}. Zawiera dodatkowe
symbole dodające prezycji do klasyfikacji.

\begin{figure}[H]\centering
\includegraphics[draft, width=\textwidth]{api.uprp.gov.pl/classify/dist.png}
\caption{Ilość patentów w danych klasyfikacjach}
\end{figure}


\begin{uwaga}
Rejestr kategorii nie jest równoznaczny z patentem. Patenty
mogą mieć wiele kategorii, a co za tym idzie, ich rejestrów.
\end{uwaga}


W niniejszej analizie pod uwagę brana jest wyłacznie klasyfikacja \ac{IPC}.

\fig{../subject/NA-IPC-geo.png}{}

To jak są uzupełniane braki geolokalizacji jest w innej sekcji.
Poniżej pokazane są tego efekty oraz rozkład lokalizacji z ich przyporządkowaniem
geolokalizacyjnym. Na następnej stronie jest również przekrój roczny dla badanych
lat. Szczególnie doktliwe braki pozostają w sekcjach E oraz F.
To jak uzupełnione zostały dane ma relatywnie równomierny rozkład po między
sekcjami.

\newpage
\begin{multicols}{2}

  \halffig{../subject/NA-IPC-loc-A.png}{A}

  \halffig{../subject/NA-IPC-loc-B.png}{B}

  \halffig{../subject/NA-IPC-loc-C.png}{C}

  \halffig{../subject/NA-IPC-loc-D.png}{D}

  \halffig{../subject/NA-IPC-loc-E.png}{E}

  \halffig{../subject/NA-IPC-loc-F.png}{F}

  \halffig{../subject/NA-IPC-loc-G.png}{G}

  \halffig{../subject/NA-IPC-loc-H.png}{H}

\end{multicols}
\newpage

\fig{../subject/map-IPC-reg.png}{}

Mapa rozkładu geograficznego sekcji klasyfikacji powyżej obrazuje
duże skupienie patentów z sekcji D --- włókiennictwo; papiernictwo.
Fakt ten może wynikać z samej liczności patentów o tej klasyfikacji
(patrz: \cref{fig:../subject/NA-IPC-loc-D.png}).
Zauważalna jest również duża ilość patentów w regionach, które zawierają
duże miasta. Wśród nich wyróżnia się Warszawa. W oczy rzuca się też
Wrocław jako mocny ośrodek patentów chemicznych (sekcja C klasyfikacji).



\fig{../subject/map-IPC-A-periods.png}{A w kolejnych latach}

\fig{../subject/map-IPC-B-periods.png}{B w kolejnych latach}

\fig{../subject/map-IPC-C-periods.png}{C w kolejnych latach}

\fig{../subject/map-IPC-D-periods.png}{D w kolejnych latach}

\fig{../subject/map-IPC-E-periods.png}{E w kolejnych latach}

\fig{../subject/map-IPC-F-periods.png}{F w kolejnych latach}

\fig{../subject/map-IPC-G-periods.png}{G w kolejnych latach}

\fig{../subject/map-IPC-H-periods.png}{H w kolejnych latach}