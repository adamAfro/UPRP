\section{Dane patentowe}



\subsection{Rejestr dat związanych z patentami}

Poszczególne czynności związane z ochroną patentów są rejestrowane
przez urzędy i są dostępne w wymienionych zbiorach danych.
Takie wydarzenia są powiązane z konkretną datą kalendarzową,
co pozwala na wyznaczenie czasu pojawienia się danej informacji
w systemie. Można wyróżnić 9 typów dat:

\begin{itemize}
\item publiczne ujawnienie \foreign{ang}{exhibition}
\item roszczenia z pierwszeństwa \foreign{ang}{priority claim} --- 
      data rozszczenia sprzed rozpoczęcia procesu patentowania dla wybranego urzędu
\item regionalna deklaracja \foreign{ang}{regional filing}\todonote{potrz. wyjaśniene}
\item deklaracja \foreign{ang}{filing}\todonote{potrz. wyjaśniene}
\item aplikacja \foreign{ang}{application} --- data złożenia aplikacji
\item publikacja przed przyznaniem ochrony \foreign{ang}{unexamined-printed-without-grant}
\item przyznanie ochrony \foreign{ang}{grant}
\item decyzja urzędowa
\item publikacja
\end{itemize}
\todonote{potrz. źródło}



\subsection{Klasyfikacje patentów}

Klasyfikacje patentowe to systemy, które pozwalają na przypisanie
patentów do odpowiednich dziedzin.



\section{Dane przestrzenne}

Dane przestrzenne odnoszą się do miejsc z jakimi są powiązane
osoby albo organizacji związane z patentami. Patent może mieć więc
kilka lokalizacji, żadna nie jest określona jako główna.
\todonote{w sumie można by wyznaczyć główną lokację
          jako miast organizacji jeśli jakaś jest}

Dane zawierają wyłącznie nazwy miast, więc ich umieszczenie na
mapie wiąże się z wadami: nazwy nie są unikalne --- w takim
przypadku używany jest algorytm minimalizacji odległości, tak
żeby wybrać optymalną kombinację. Wynika z tego oczywiste obicążenie.
Po za tym nazwy mniejszych miejscowości mogą być duplikatami
nazw miast. Pewne powiązanie jest w takim przypadku niemożliwe
w wykonalny sposób na podstawie patentowych informacji.



\subsubsection{Międzynarodowa Klasyfikacja Patentów}

W Polsce funkcjonuje klasyfikacja
\ac{MKP}, czyli \ac{IPC}. Zapis klasyfikacji w tym systemie to ciąg
cyfrowo-literowy składający się z 4 części:

\begin{enumerate}
    \item Dział - najwyższa hierarchia złożona z 8 kategorii
    \begin{itemize}
        \item ma tytuł informacyjny
        \item każdy tytuł działu ma swój symbol: A, B, C, D, E, F, G albo H
        \begin{itemize}
            \item A – podstawowe potrzeby ludzkie
            \item B – różne procesy przemysłowe; transport
            \item C – chemia; metalurgia
            \item D – włókiennictwo; papiernictwo
            \item E – budownictwo; górnictwo
            \item F – budowa maszyn; oświetlenie; ogrzewanie; uzbrojenie; technika minerska
            \item G – fizyka
            \item H – elektrotechnika
        \end{itemize}
        \item poddział - każdy dział może zawierać poddział, który nie jest oznaczany symbolem
    \end{itemize}
    \item Klasa - drugi poziom hierarchii
    \begin{itemize}
        \item ma tytuł informacyjny
        \item oznaczana przez liczbę 2-cyfrową
        \item zakres klasy - skrótowa informacja o treści klasy
    \end{itemize}
    \item Podklasa - trzeci poziom hierarchii
    \begin{itemize}
        \item ma tytuł informacyjny
        \item oznaczana dużą literą
        \item ma zakres i tytuł pomocniczy
    \end{itemize}
    \item Grupa - czwarty poziom hierarchii
    \begin{itemize}
        \item 2 zestawy cyfr oddzielone ukośnikiem
        \begin{itemize}
            \item zestaw pierwszy składa się od 1 do 3 cyfr i określa grupę główną
            \item zestaw drugi składa się z 2 cyfr i określa grupę pomocniczą, grupa główna jest oznaczana 00
        \end{itemize}
        \item grupa ma tytuł informacyjny, podgrupa ma bardziej szczegółowe hasło
    \end{itemize}
\end{enumerate}



\subsubsection{Wspólna Klasyfikacja Patentów}

\ac{CPC} to klasyfikacja stworzona przez \ac{EPO} i \ac{USPTO}
jako wspólny wysiłek w kierunku standaryzacji klasyfikacji.

\begin{enumerate}
    \item Sekcja - najwyższa hierarchia złożona z 9 kategorii
    \begin{itemize}
        \item ma tytuł informacyjny
        \item każda sekcja ma swój symbol: A, B, C, D, E, F, G, H albo Y
        \begin{itemize}
            \item A – Potrzeby ludzkie
            \item B – Różne operacje przemysłowe; Transport
            \item C – Chemia; Metalurgia
            \item D – Tekstylia; Papier
            \item E – Budownictwo; Górnictwo
            \item F – Mechanika; Oświetlenie; Ogrzewanie; Broń; Wybuchy
            \item G – Fizyka
            \item H – Elektryczność
            \item Y – Ogólne technologie lub zastosowania specyficzne dla nowych lub pojawiających się dziedzin technologii
        \end{itemize}
    \end{itemize}
    \item Klasa - drugi poziom hierarchii
    \begin{itemize}
        \item ma tytuł informacyjny
        \item oznaczana przez liczbę 2-cyfrową
        \item zakres klasy - skrótowa informacja o treści klasy
    \end{itemize}
    \item Podklasa - trzeci poziom hierarchii
    \begin{itemize}
        \item ma tytuł informacyjny
        \item oznaczana dużą literą
        \item ma zakres i tytuł pomocniczy
    \end{itemize}
    \item Grupa - czwarty poziom hierarchii
    \begin{itemize}
        \item 2 zestawy cyfr oddzielone ukośnikiem
        \begin{itemize}
            \item zestaw pierwszy składa się od 1 do 3 cyfr i określa grupę główną
            \item zestaw drugi składa się z 2 cyfr i określa grupę pomocniczą, grupa główna jest oznaczana 00
        \end{itemize}
        \item grupa ma tytuł informacyjny, podgrupa ma bardziej szczegółowe hasło
    \end{itemize}
\end{enumerate}



\subsubsection{Rewizja \ac{IPC}}

\ac{IPCR} powstał przez rewizję systemu \ac{IPC}. Zawiera dodatkowe
symbole dodające prezycji do klasyfikacji.