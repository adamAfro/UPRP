\section{Identyfikacja osób}

Głownym problemen danych jest niejednoznaczność w kontekście identyfikacji osób.
W danych patentowych, osoby rozróżnia się za pomocą imienia, nazwiska
oraz nazwy miejscowości. Jak wiadomo wiele osób może mieć te same imię i nazwisko,
także w jednym miejscu. Jest to duże ograniczenie wynikające z samego zbioru danych.
Należy także wspomnieć o drobnych niespójnościach danych w zapisie imion i nazwisk
(\cref{def:drobne-niespójności}) --- występowanie diaktryk i akcentów w zapisie
nie jest gwarantowane, a jednocześnie nie jest wykluczone.

Kolejną niejednoznacznością jest podobne zjawisko dla nazw miejscowości.
W Polsce jest wiele miejscowości o identycznych nazwach, a rejestry nie oferują
nic po za samą nazwą. Tutaj także występuje problem z diaktrykami i akcentami.

Ponadto występują też 2 inne problemy. Pierwszym jest 
niespójność fragmentacji danych (\cref{def:niespójność-fragmentacji}).
W przypadku tabeli z danymi osobowymi wynalazców są do dyspozycji ich
imiona i nazwiska. W przypadku pozostałych osób związanych z patentem
są to najczęściej ciągi imion i nazwisk. Nie jest jednak gwarantowane,
że dotyczą one osób fizycznych. Drugim problemem jest niespójność 
typów danych (\cref{def:niespójność-typów}). Część danych oznaczonych
jako imiona dotyczy nazw firm lub instytucji. Oznaczenie tego faktu
istnieje tylko w niektórych przypadkach, dużo częściej jest to pominięte.



\subsection{Niespójność typów}

Rozwiązaniem problemu pomieszania nazw organizacji o raz imion osób
jest wykorzystanie danych o odpowiednim typowaniu i fragmentacji.
Wyróżniamy w nich pojedyncze słowa albo ciągi i przyjmujemy, 
że są charakterystyczne dla danego typu. W przypadku nazw organizacji
są to ciągi, a w przypadku imion i nazwisk --- pojedyncze słowa.
Dodatkowo testujemy podpisy na zawartość ustalonego zbioru słów
kluczowych, które są charakterystyczne dla nazw organizacji.
Po utworzeniu zbiorów słów i ciągów kluczowych imiona i nazwy
są klasyfikowane na podstawie ich zawartości. W ten sposób
możemy zidentyfikować, czy dany wpis dotyczy osoby fizycznej
czy organizacji.

\begin{uwaga}
Duża część nazw i imion nie ulega klasyfikacji w wyniku powyższego
algorytmu. Dla uproszczenia zakładamy, że dotyczą one wtedy imion
osób fizycznych.
\end{uwaga}



\subsection{Wyszukiwanie podobieństw}

\begin{defi}
Wpis osobowy $w_i$ --- pojedynczy obiekt przypisany danemu patentowi. Zawiera
informacje o osobie powiązanej z patentem. Jeden paten może zawierać
wiele wpisów osobowych. Każdy składa się z: imienia i nazwiska albo
ciągu imienniczego (\cref{def:ciąg-imienniczy}); nazwy miejscowości zameldowania.
\end{defi}

\begin{defi}\label{def:ciąg-imienniczy}
Ciąg imienniczy $N_k$ --- ciąg imion oraz nazwisk przypisany danej osobie.
Nazwy podwójne rozdzielone znakami interpunkcyjnymi są traktowane jako
oddzielne imiona.
\end{defi}

Każdy element ciągu imienniczego podlega normalizacji. Wszystkie jego 
znaki są traktowane jako wielkie litery, a znaki diaktryczne oraz akcenty 
są zastępowane ich odpowiednikami tych wyróżnień piśmienniczych.
Wynika to z faktu, że ich obecność nie jest pewna w danych.

\begin{uwaga}
To czy dany element $n,\ n\in N_k$ jest imieniem, czy nazwiskiem nie zawsze
jest jednoznaczne.
\end{uwaga}

\begin{uwaga}
Dane w zbiorach uwzględniają różną szczegółowość w zapisie imion i nazwisk.
Niektóre zawierają drugie imię, niektóre wyłącznie literę drugiego imienia.
Przypadków jest wiele. Poniższe podejście pomija tę ambiwalencję.
\todonote{można tego użyć jako dodatk. deter. podob.}
\end{uwaga}

\begin{defi}
Główna para imiennicza $\hat N_k$ --- zbiór 2-elementowy pierwszego 
i ostatniego słowa ciągu imienniczego
\end{defi}

\begin{defi}\label{defi:zgodność-nazw}
Zgodność nazewnicza 2 wpisów $w_i,w_j$ występuje pod warunkiem, że
elementy suma zbiorów ich głównych par imienniczych jest im równa:
$$
\varphi(w_i, w_j) = \begin{cases}
  1 & \text{jeśli } \hat N_i = (\hat N_i \cap \hat N_j) = \hat N_j\\
  0 & \text{w przeciwnym przypadku}
\end{cases}
$$
\end{defi}

\begin{defi}\label{defi:podobieństwo-af-nazw}
Podobieństwo afiliacyjno-nazewnicze $\tilde \varphi$ --- dotyczy wpisów,
które są związane z patentami badanej pary wpisów $w_i, w_j$.
Zbiór $N_i$ jest zbiorem zbiorów głównych par imienniczych wpisów
dotyczących patentu zawierającego wpis $w_i$. Analogicznie jest
dla $N_j$:

$$
\tilde \varphi(p_i, p_j) = | \tilde N_i \cap \tilde N_j | \ge 0,\quad
\tilde N_i = \{ \hat N_k \mid w_k \in W_i \land k \ne i \}
$$
\end{defi}

\begin{uwaga}
Zgodność $\varphi$ (\cref{defi:zgodność-nazw}) oraz podobieństwo $\tilde \varphi$ 
(\cref{defi:podobieństwo-af-nazw}) pomija rozróżnienie na imiona i nazwiska.
\todonote{dodać sposób korzys. z rozróżn. na imię, nazwi.}
\end{uwaga}

\begin{defi}\label{defi:zgodność-geolokalizacyjna}
Zgodność geolokalizacyjna:
$$
\gamma(p_i, p_j) = \begin{cases}
  1 & \text{jeśli } G_i = G_j\\
  0 & \text{w przeciwnym przypadku}
\end{cases}
$$
gdzie $G_i$ to zbiór geolokalizacji przypisanej danej osobie.
\end{defi}

\begin{defi}
Zbiór afiliacyjno-nazewniczy $k$-osoby $\tilde N_k$ - zbiór imion, 
nazwisk oraz słów, które mogą być imieniem lub nazwą; zawiera
wyżej wymienione elementy, którymi identyfiują się osoby będące
współautorami patentów razem z $k$-osobą.
$$\tilde N_k \subset N_0$$
\end{defi}

\begin{uwaga}
Wpisy zawierają nazwy miejscowości zameldowania, jednak wyszukwianie za ich
pomocą odbywa się po geolokalizacji patentowej\todonote{potrzeb. odnies.}.
W związku z tym, mimo identycznej nazwy miejscowości może nie dojść do
zgodności geolokalizacyjnej.
\end{uwaga}

\begin{defi}\label{defi:podobieństwo-af-geo}
Podobieństwo afiliacyjno-geolokalizacyjne $\tilde\gamma$ --- ilość identycznych geolokalizacji
dla dwóch osób: $p_i$ oraz $p_j$:

$$
\tilde\gamma(p_i, p_j) = | \tilde G_i \cap \tilde G_j | \ge 0,
$$

gdzie $\tilde G_i$ to zbiór geolokalizacji osób w relacji współautorstwa z $p_i$.
\end{defi}

Kolejnym determinantem jest klasyfikacja. \Cref{wniosek:klasyfikacje-deter-1}
pokazuje, że klasyfikacje nie są dobrym determinantem jeśli opierać by się wyłącznie 
na nich. Warto jednak zauważyć, że klasyfikacje mogą być dobrym uzupełnieniem
dla pozostałych determinantów.

\begin{uwaga}
Aktualnie jedyną klasyfikacją wziętą pod uwagę jest \ac{IPC}, które
nie występuje dla wszystkich obserwacji.\todonote{usunąć to po uwzgl.
innych klasyf.}
\end{uwaga}

\begin{defi}\label{defi:zgodność-clsf}
Zgodność klasyfikacyjna $\eta$ --- ilość identycznych sekcji klasyfikacji,
w których znajdują się aplikacje patentowe dwóch osób: $p_i$ oraz $p_j$:
$$\eta(p_i, p_j) = | C_i \cap C_j | \ge 0,$$
gdzie $C_i$ to zbiór sekcji klasyfikacji dla osoby $p_i$.
\end{defi}

\subsubsection{Przebieg wyszukiwania}

\begin{uwaga}
Przyjmujemy uproszczenie: zakładamy, że dana osoba w różnych patentach 
jest podpisana w jednolity sposób.
\end{uwaga}

Pierwszym etapem wyszukiwania jest zawężenie zakresu wyłącznie do relacji
spełniających warunek $\varphi(p_i, p_j) = 1$.
W efekcie otrzymujemy zbiór $W_0$:

$$W_0 = \{ (p_i, p_j)\mid \varphi(p_i, p_j) = 1 \}$$

Kolejnym etapem jest wyznaczenie wartości zgodności klasyfikacji oraz 
podobieństw afiliacyjnych.

$$W_1 = \{ ( \eta(p_i, p_j), \tilde \gamma(p_i, p_j), \tilde \varphi(p_i, p_j) ) \mid (p_i, p_j) \in W_0 \}$$

Zbiór $W_1$ dzielimy na dwa podzbiory zgodnie z wartościami zgodności lokalizacyjnej:

$$
W_2 = \{ ( w \mid w \in W_1, \gamma(p_i, p_j) = 0 \}\qquad 
W_3 = \{ ( w \mid w \in W_1, \gamma(p_i, p_j) = 1 \}
$$

\begin{uwaga}
Zgodność lokalizacyjną można zastąpić miarą odległości geograficznej,
jednak na potrzeby uproszczenia jest to wartość binarna.
\end{uwaga}

Oba zbiory rozważamy jako oddzielne przypadki ze względu na ich
gruntownie różną naturę. Przyjmując pewne stałe jako wartości
graniczne dla zgodności klasyfikacji oraz podobieństw afiliacyjnych
możemy podjąć decyzję o zgodności dwóch wpisów pod względem
opisywania jednej osoby. Na podstawie grafu tych zgodności
identyfikujemy spójne składowe. Każda taka część grafu to
zbiór wpisów, które opisują tę samą osobę.



\subsection{Uzupełnianie braków geolokalizacji za pomocą innych danych}

Brak danych dotyczących położenia osób związanych z danym patentem jest
istotnym problemem w analizie dyfuzji przestrzennej. Pominięcie obserwacji
z powodu braku danych może prowadzić do błędnych wniosków. 
Determinacja położenia osób za pomocą innych danych patentowych jest
więc kluczowa.

W danych można wyróżnić 3 przypadki dostępności geolokalizacji.

\begin{przyp}\label{przyp:brak-geo-0}
Geolokalizacja jest dostępna dla każdej osoby związanej z patentem.
\end{przyp}

\begin{przyp}\label{przyp:brak-geo-n}
Geolokalizacja jest dostępna dla części osób zwiazanych z patentem.
\end{przyp}

\begin{przyp}\label{przyp:brak-geo-N}
Geolokalizacja nie jest dostępna dla żadnej osoby związanej z patentem.
\end{przyp}

W przypadku \ref{przyp:brak-geo-0} nie ma potrzeby uzupełniania danych.
W przypadkach \ref{przyp:brak-geo-n} i \ref{przyp:brak-geo-N} 
rozważamy sytuację po identyfiakcji za pomocą podobieństw. 
Jeśli dany wpis osoby jest przypisane osobie, 
która w innych wpisach ma geolokalizacje, to jest ona uzupełniana
najczęściej powtarzającą sie geolokalizacją.

Efektem jest zmiana części przypadków z \ref{przyp:brak-geo-N} na
\ref{przyp:brak-geo-n}, potencjalnie także na \ref{przyp:brak-geo-0}.
W takiej sytuacji należy powtórzyć czynność wyszukiwania podobieństw ---
dane zostały uzupełnione o geolokalizację, więc potencjalnie można
oczekiwać znalezienia nowych relacji identyczności.
Te 2 kroki należy powtarzać dopóki dają efekt. Ostatecznie
przypadki, które pozostały przypadkami \ref{przyp:brak-geo-N}
nie mogą zostać uzupełnione wiarygodnymi metodami.
W przypadkach \ref{przyp:brak-geo-n} można zastosować wyróżnianie
najczęściej powtarzanego miejsca związania pośród innych osób i przyjąć
je za geolokalizację osób bez niej.