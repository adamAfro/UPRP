\section{Wprowadzenie do tematu dyfuzji wiedzy}

Dyfuzja wiedzy innowacyjnej jest przedmiotem zainteresowania
wielu badaczy. Przegląd dzieł związanych z tematem od różnych
autorów pozwala na zrozumienie tego jakie są kluczowe elementy
procesu dyfuzji wiedzy innowacyjnej i czym właściwie ona jest.

Nonaka \cite{No-98} rozdziela informację od wiedzy na podstawie
tego jak została zaadaptowana przez podmioty i w jakim
kontekście jest umieszczona. Pogląd rozdziału informacji
od wiedzy powielają także Morone i Taylor \cite{Mo-Ta-09}.

Nonaka \cite{No-98} rozdziela także wiedzę na jawna i ukrytą.
Pathirage \cite{Pa-08} wskazuje, że taki rozdział jest dominujący 
w literaturze przedmiotu. Alwis i Harmann \cite{Al-Ha-08} dają dobry 
wgląd w to czym jest wiedza ukryta poprzez zagłębienie się 
w literaturę. Wskazują, że wiedza ukryta jest kluczowym 
elementem innowacyjności firm i jest ściśle powiązana z 
procesem dyfuzji wiedzy innowacyjnej.

Częstym cytatem w literaturze przedmiotu jest stwierdzenie
Michaela Polanyiego: \textit{Wiemy więcej niż potrafimy powiedzieć},
gdzie z faktu, że \textit{wiemy więcej} wnioskujemy o istnieniu czegoś
ponad to czym jest wiedza jawna - tego \textit{co potrafimy powiedzieć}.

Dużą rolę w zachowaniu konkurencyjności firm przypisuje się
zarządzaniu wiedzą, a szczególnie wiedzą ukrytą \cite{No-98}.
Nonaka, jak inni \cite{Mo-16}, \cite{Ga-Th-14}, definiuje wiedzę ukrytą
jako zaprzeczenie wiedzy jawnej. Wiedza jawna to taka
przechowywana na wszelkich nośnikach, ale też możliwa w artykulacji
do innych osób. Wiedza ukryta, jako jej przeciwieństwo,
jest trudna albo niemożliwa do wyrażenia słowami albo symbolami.
Jest nabywana podczas praktyki albo obserwacji \cite{No-98}.

Nonaka \cite{No-98} definiuje model procesu tworzenia wiedzy w firmie, 
jako 4-etapową spiralę, w której kolejne kroki to: nabywanie wiedzy
ukrytej, jej synteza w wiedzę jawną, standaryzacja wiedzy jawnej
i adaptacja wiedzy jawnej przez pracowników do wiedzy ukrytej.
Podobny pogląd na powstawanie wiedzy przedstawiaja Morone i 
Taylor \cite{Mo-Ta-09} stwierdzając, że wiedza zawsze jest początkowo
wiedzą ukrytą, a dopiero w procesie jej artykulacji staje się
wiedzą jawną.

Istotą przepływu wiedzy ukrytej jest to jakie warunki panują
w firmie. Nonaka \cite{No-98} wskazuje jako modelowe, firmy japońskie, w
których panuje redundancja informacji, co sprawia, że pracownicy
posiadają podobny zestaw wiedzy ukrytej. Alwis i Hartmann \cite{Al-Ha-08}
także przypisują jakość dyfuzji wiedzy ukrytej do organizacji
firm twierdząc, że likwidacja barier wewnętrznych w firmie jest
kluczowa dla efektywnego przepływu wiedzy.

Bathelt i Feldman \cite{Ba-Fe-11} także rozważają powstawanie wiedzy
innowacyjnej jako proces przerabiania aktualnej wiedzy na nową.
W nieścisły sposób łączy się to z modelem Nonaka \cite{No-98}, gdzie
wiedza także ulega ciągłej transformacji. 

Dalej \cite{Ba-Fe-11}, na podstawie literatury, 
stwierdzają o ograniczeniach przestrzennych jakie wiążą się z
rozprzestrzenianiem się wiedzy. Wynikają one między innymi z
przywiązaniem ludzi do ich miejsca zamieszkania, czy ulokowaniem
środków firm, czy całego sektora w jednym regionie.

Samo rozprzestrzenianie się wiedzy można podzielić na 2 typy:
dyfuzję i wymianę \cite{Mo-Ta-09}. Wymiana polega na
przepływnie wiedzy między podmiotami na podstawie uzgodnionych
oraz świadomych działań w trakcie których następuje symbiotyczna
interakcja, w której jedna strona zyskuje wiedzę, a druga
wiedzę inną albo korzyści niezwiązane z samą wiedzą. W kontrze
do wymiany, dyfuzja to proces nieświadomego przepływu samej wiedzy.
Morone i Taylor wskazują, że w procesie dyfuzji, podmioty będące
odbiorcami mogą wykorzystywać mimowolne przepływy wiedzy na
swoją korzyść. Dalej zastrzegają, że takiemu procesowi dyfuzji
podlega wiedza ukryta, a jej przyswojenie wymaga zdolności 
(pojemności) absorpcyjnych (ang. \textit{absortive capacity}).

Taylor i Morone \cite{Mo-Ta-09} rozkładają dyfuzję na 3 procesy:
rozlewanie \foreign{ang}{spillover}, transfer oraz integrację. Rozlewanie
i transfer to procesy podobne z tą różnicą, że transfer jest
określony jako przepływ z pierwotną intencją jego zaistnienia.
Oba te procesy wymagają istniejącej wcześniej wiedzy, która
umożliwia absorbcję nowej wiedzy. Integracja z kolei to proces,
w który istniejąca wiedza jest aplikowana w innym kontekście.

Klarl \cite{Kl-14} twierdzi, że dyfuzja wiedzy to proces społeczny,
który napędzają, po pierwsze: więzi między ludźmi i grupami,
a po drugie cechy indywidualne ludzi. To jak silne są więzi
w grupie oraz między grupami ma znaczący wpływ na szybkość
rozprzestrzeniania się wiedzy. Klarl twierdzi, że można
wyróżnić w populacji grupy o różnych miarach dyfuzji wewnątrz,
jak i po między grupami oraz z zewnątrz. Dodatkowo zakłada
ograniczenia przestrzenne związane z dyfuzją wiedzy - sieć
jest tym słabsza im bardziej jest rozproszona w przestrzeni.

Boone, Ganeshan, i Hicks \cite{Bo-Ga-Hi-08} przedstawiają znane już zjawisko
starzenia się wiedzy jako proces, w którym wiedza nagromadzona
w organizacji przekłada się na coraz mniejszą wartość dla firmy.
Mimo, że innowacje są efektem nadbudowywania wiedzy to należy
się domyślić, że jej starzenie skutecznie temu zapobiega. 
Zaznaczają też, że szybkość starzenia zależy od branży, ale też
działań jakie są podejmowane aby mu zapobiegać.

Z podobną tezą występują Grubler i Nemet \cite{Gr-Ne-13}, określając
zjawisko starzenia się jako \textit{oduczanie} - przeciwieństwo nauki.
Podają ku temu 2 przyczyny: zanikanie wiedzy związane z rotacją
pracowników oraz zanikanie wiedzy związane z innowacjami, które
następują na tyle szybko, że nie sposób dostosować do nich starej
wiedzy. Zanikanie związane z rotacją wiąże się z tym, że
przeniesienie, czy zwolnienie pracownika sprawia, że ten,
posiadający wiedzę ukrytą, nie używa jej już w organizacji.
Grubler i Nemet zaznaczają, że to właśnie wiedza ukryta jest
szczególnie wrażliwa na starzenie.

Frenken, Van Ooor i Verburg \cite{Fr-Oo-Ve-07} opisują pojęcie pokrewnej i
niepokrewnej różnorodności wiedzy. Różnorodność pokrewna dotyczy
wiedzy z jednej dziedziny, a niepokrewna z różnych dziedzin.
Badacze wskazują, że różnorodność pokrewna sprzyja bardziej
dyfuzji wiedzy niż różnorodność niepokrewna.



\section{Patenty w literaturze przedmiotu}

Patenty są istotnym wskaźnikiem dyfuzji wiedzy innowacyjnej
między środowiskami akademickimi, a rynkiem \cite{Lo-Br-08}.
Dają one informacje o tym jak wiedza zdobywana na uczelniach
przenika do firm i jak jest wykorzystywana w praktyce.

Sorenson i Fleming \cite{So-Fl-04} wskazują w swojej pracy na związek
między cytowaniami w patentach, a tym jak te same patenty są
cytowane. W badaniu wykazali, że patenty, które cytują częściej
są także częściej cytowane. Wynika z tego, że przepływ wiedzy
jest przyśpieszany przez uwzględnienie wiedzy jaka została
już wcześniej zgromadzona.

Acs, Anselin i Varga \cite{Ac-An-Va-02} określają związek między patentami,
a innowacjami jako zależność nieidealną, gorszą od literatury
badawczej ale porównywalną z nią. Wskazują na zaletę jaką może
być uwzględnienie przez patenty innowacji z mniejszych organizacji,
co często jest pomijane w opracowaniach naukowych.

Verspagen i Schoenmakers \cite{Ve-Sc-00} podają 2 powody, dla których
patenty mogą być solidnym wyznacznikiem dyfuzji w ekonomii:
w kontekście dużych firm zakładają istnienie zespołów R\&D do
analizy patentów na rynku, a co za tym idzie nabywania wiedzy
o nich. Drugi powód to cytowania patentowe, które wskazują
na związki między wynalazcami - duże podobieństwo jest miarą
tego jak wiedza ulega dyfuzji na tym wymiarze.


\section{Podsumowanie}

Na podstawie wcześniejszego przeglądu literatury można wyprowdzić
pewne ogólne definicje, co do których będą stawiane tezy.

Jednym z podstawowych założeń jest rozdział wiedzy od samej informacji.
Logicznie można też stwierdzić, że wiedza jest podzbiorem informacji,
która wyróżnia się tym, że jest interpretowana przez człowieka w danym
kontekście.

\Needspace{5\baselineskip}
\begin{defi}
Informacja ($I$) --- ogólny termin odnoszący się do wszelkich form wiedzy, 
umiejętności, doświadczeń, przekonań przechowywanych na różnych nośnikach 
albo w ludzkiej pamięci - świadomie lub nieświadomie.
\end{defi}

\begin{defi}
Wiedza ($W$) --- informacja zinterpretowana przez człowieka w danym kontekście.
\begin{center}
\begin{math}
W \subset I
\end{math}
\end{center}
\end{defi}

To czym jest informacja można odnieść zarówno do danych
przechowywanych na serwerach, tez zawartych w dziełach naukowych,
jak i umiejętności praktycznych wszelkiego rodzaju, np.
balans ciałem podczas jazdy na rowerze. W odróżnieniu od informacji, 
wiedza jest zawsze związana z jakimś kontekstem oraz tym 
jak jest ona interpretowana przez daną jednostkę.

\Needspace{5\baselineskip}
\begin{defi}
Wiedza jawna ($W_j$) \foreign{ang}{explicit knowledge} --- wiedza
zapisywana na nośnikach informacji i przekazywana za pomocą
języka, symboli, obrazów, dźwięków.
\end{defi}

Wiedza jawna jest ogólnie dostępna z samej jej definicji - przekazywana
w słowach ulega upowszechnieniu w postaci rozmów i wszelkich mediów: 
książek, artykułów, prezentacji, filmów, itp. 

\Needspace{5\baselineskip}
\begin{defi}
Wiedza ukryta $W_u$ --- przeciwieństwo wiedzy jawnej:
wiedza, której nie da się uchwyciś słowami czy symbolami,
nabywana podczas praktyki.
\begin{center}
\begin{math}
W_u = W \setminus W_j
\end{math}
\end{center}
\end{defi}

W anglosaskiej literaturze określana jest jako \textit{tacit knowledge}.
Nazwa pochodzi od łacińskiego \textit{tacitus} - \textit{milczący}.
Dobrze oddaje naturę tej wiedzy, bo oprócz tego, że jest ona
trudna do uchwycenia, to jej specyfika wynika właśnie z niemożności
do jej artykulacji w języku. Jej specyfikę podsumowuje popularne 
w literaturze stwierdzenie \textit{Wiemy więcej niż potrafimy powiedzieć}, 
Michael Polanyi.

\Needspace{5\baselineskip}
\begin{defi}
Wiedza innowacyjna $W_i$ --- wiedza rozszerzająca dotychczasową
wiedzę, która prowadzi do rozwiązania aktualnych problemów.
\begin{center}
\begin{math}
W_i \subset W\qquad
W + W_i = W' \supset W
\end{math}
\end{center}
\end{defi}

Wiedza innowacyjna jest zawsze w jakimś stopniu nowa, ale
niekoniecznie musi być to coś zupełnie nowego. Może to być
również modyfikacja istniejącej wiedzy i najczęściej tak
właśnie powstaje - poprzez nadbudowywanie na istniejącej wiedzy.

\begin{defi}
Dyfuzja wiedzy innowacyjnej --- proces czasowy przepływu 
wiedzy innowacyjnej między podmiotami, występujący szczególnie
w lokalnej przestrzeni.
\end{defi}

Jest to \textbf{mimowolna} transmisja wiedzy w czasie i przestrzeni,
która wiąże się z przepływem wiedzy z ośrodków o dużej
koncentracji do sąsiadujących z nim podmiotów o mniejszej
koncentracji wiedzy.

Kluczowa w dyfuzji wiedzy innowacyjnej jest wymiana wiedzy
twarzą w twarz. Wielu autorów podkreśla, że wiedza ukryta,
która jest kluczowym elementem innowacyjności. Nie ulega
ona transmisji poprzez media, ponieważ jest ona niewerbalna.
W konsekwencji do jej rozprzestrzeniania potrzebne są spotkania
i interakcje między ludźmi podczas jej praktyki. Publikacje
naukowe zwracają też uwagę na istotę zaangażowania w rozszerzaniu
się tego rodzaju wiedzy, które jest łatwiej zainicjować w
sytuacjach socjalnych niż rozgrywających się na polu wirtualnym,
czy w postaci innych środków komunikacji.

Ograniczenia występowania tego rodzaju sytuacji są dość oczywiste i
wynikają między innymi z przestrzeni w jakiej operują ludzie.
Zazwyczaj są to ograniczenia do przestrzeni lokalnej, co sprawia,
że dyfuzja wiedzy innowacyjnej jest ograniczona do pewnego obszaru.
Nawet w lokalnych warunkach mogą dochodzdić dodatkowe obciążenia
takie jak bariery fizyczne, kulturowe czy wynikające z hierarchii
organizacyjnej albo struktury społecznej.


\Needspace{15\baselineskip}
\subsection{Patent jako narzędzie ochrony wiedzy innowacyjnej}

\begin{acronym}
  \acro{UPRP}{Urząd Patentowy Rzeczypospolitej Polskiej}
  \acro{WUP}{Wiadomości Urzędu Patentowego}
  \acro{BUP}{Biuletyn Urzędu Patentowego}
  \acro{EPO}{European Patent Office}
  \acro{WIPO}{World Intellectual Property Organization}
  \acro{MKP}{Międzynarodowa Klasyfikacja Patentów}
  \acro{IPC}{International Patent Classification}
  \acro{IPCR}{International Patent Classification Revision}
  \acro{API}{Application Programming Interface}
  \acro{URI}{Uniform Resource Identifier}
  \acro{URL}{Uniform Resource Locator}
  \acro{OCR}{Optical Character Recognition}
  \acro{XML}{Extensible Markup Language}
  \acro{CPC}{Cooperative Patent Classification}
  \acro{USPTO}{United States Patent and Trademark Office}
  \acro{USPG}{United States Patent (and Trademark Office) Grants}
  \acro{USPA}{United States Patent (and Trademark Office) Applications}
  \acro{PGC}{Patents.Google.com}
  \acro{PLO}{Patenty Lens.org}
\end{acronym}

\begin{defi}
\textbf{Patent} --- forma ochrony prawnej udzielana na nieużywane, 
albo nieopatentowane wcześniej wynalazki, które wykazują 
zastosowanie w przemyśle. Powinny też być nieoczywiste dla 
ekspertów dziedzinowych.
\end{defi}

W Polsce przydzielaniem i ochroną patentów zajmuje się \ac{UPRP}, 
na poziomie europejskim jest to \ac{EPO}, a światowa organizacja 
to \ac{WIPO}.

Wartość patentów jako wskaźników dyfuzji wiedzy innowacyjnej
wynika z charakterystyki patentów jako narzędzi ochrony wiedzy.
Z definicji prawnej, patenty muszą reprezentować wiedzę innowacyjną
i być nowe, nieoczywiste oraz nadające się do praktycznego
zastosowania. Z tego właśnie powodu można przypuszczać, że
w pewnym stopniu oddają one stan faktyczny w dziedzinie innowacji.
Kolejnym atutem patentów jest ich klasyfikacja.
Patenty są klasyfikowane według międzynarodowej klasyfikacji
patentowej \ac{IPC}, dzięki czemu analizować je pod kątem dziedzinowym.
Dane dotyczące patentów są dostępne publicznie, co daje 
szerokie możliwości do ich analizowania oraz publikowania efektów
prac. Ponadto każdy patent jest przypisany do właściciela, oraz
na potrzeby formalne zawiera informacje o autorach, co pozwala
na budowanie sieci współpracy między podmiotami.

Mimo tych zalet, patenty mają swoje ograniczenia. Przede wszystkim
nie wszystkie wynalazki są opatentowane. Te opatentowane z kolei
mogą nie być wykorzystywane w praktyce, co sprawia, że nie
wszystkie innowacje są reprezentowane w danych patentowych.
Ponadto, patenty są zazwyczaj zgłaszane przez duże firmy, co
może prowadzić do zniekształcenia obrazu innowacyjności w
danej dziedzinie.