% Szablon, v. 3.9 p.wlaz@pollub.pl
% http://mat.pol.lublin.pl/~pwlaz/dyplom-latex

\documentclass[12pt, withmarginpar]{mwbk}
\usepackage[a4paper,twoside,top=2.6cm,bottom=2.6cm,inner=3cm,outer=2.6cm]{geometry}

\usepackage[utf8]{inputenc}            % kodowanie znaków
\usepackage[polish]{babel}             % język polski
\usepackage[T1]{polski}                % j.w.

\usepackage[font=small,labelfont=bf,justification=centering]{caption}




\input glyphtounicode.tex              % Zaznaczanie tekstu w programach
\pdfgentounicode = 1                   % j.w.

\usepackage[font=footnotesize,labelfont=bf,justification=raggedright,singlelinecheck=false]{caption}

\usepackage{lmodern}


\usepackage{xcolor}
\usepackage{tabularx}
\usepackage{float}
\usepackage{needspace}
\usepackage{fancyhdr}
\usepackage{graphicx}
\usepackage{multicol}
\usepackage{amsmath}
\usepackage{amsthm}
\usepackage{amssymb}
\usepackage{url}
\usepackage{longtable}
\usepackage{array,hhline}
\usepackage{acronym}                   % skróty literowe
\usepackage{fancyvrb}                  % kod komputerowy
\usepackage{hyperref}                  % hiperłącza na komputerach
\usepackage{cleveref}
\usepackage{todonotes}                 % notatki na marginesie
\newcommand\sidenote[1]{\todo[linecolor=lightgray,backgroundcolor=lightgray!25,bordercolor=lightgray]{\raggedright\footnotesize{#1}}}
\newcommand\todonote[1]{\todo[linecolor=red!50,backgroundcolor=lightgray!25,bordercolor=red!50]{\raggedright\footnotesize{#1}}}
\newcommand{\foreign}[2]{(#1. \textit{#2})}

\usepackage{booktabs} \heavyrulewidth=1.5bp \lightrulewidth=0.5bp

\newcommand\allsource[0]{źródło: opracowanie własne; 
                         dane: UPRP, patents.google.com, Lens.org, USPTO}

\let\leq\leqslant\let\le\leq\let\geq\geqslant\let\ge\geq 
                                       % sic! ucywilizowanie znaków niewiększości


\theoremstyle{plain}
\newtheorem{twier}{Twierdzenie}[chapter] % pierwsze to nazwa środowiska,
                                      %drugie to wyświetlana nazwa
				% to trzecie w~nawiasie kwadratowym
				% wskazuje numer dolepiony z~lewej do
				% numeru twierdzenia (tu numer
				% 'chapter', 
\newtheorem{lemat}{Lemat}[chapter]

\crefname{figure}{rys.}{rys.}
\Crefname{figure}{Rys.}{Rys.}

\crefname{table}{tab.}{tab.}
\Crefname{table}{Tab.}{Tab.}

\crefname{section}{sekcja}{sekcje}
\Crefname{section}{Sekcja}{Sekcje}

\theoremstyle{definition}
\newtheorem{defi}{Definicja}[chapter]
\crefname{defi}{def.}{def.}
\Crefname{defi}{Def.}{Def.}

\newcommand{\crefpairconjunction}{ oraz }
\newcommand{\creflastconjunction}{ oraz }

\theoremstyle{definition}
\newtheorem{przyp}{Przypadek}[chapter]

\theoremstyle{definition}
\newtheorem{przykład}{Przykład}[chapter]

\crefname{przyp}{przypadek}{przypadki}
\Crefname{przyp}{Przypadek}{Przypadki}

\theoremstyle{remark}
\newtheorem{uwaga}{Uwaga}[chapter]
\newtheorem{wniosek}{Wniosek}[chapter]


\newcommand{\chart}[2]{
\begin{figure}[H]
\centering
\includegraphics[scale=.42]{#1}
\caption{#2}
\label{#1}
\end{figure}}

\newcommand{\chartside}[3]{
\begin{multicols}{2}
\begin{figure}[H]
\centering
\includegraphics[scale=.42]{#1}
\caption{#2}
\label{#1}
\end{figure}
\columnbreak
#3
\end{multicols}}

\newcommand{\charttripled}[6]{
\begin{multicols}{3}

  \begin{figure}[H]\centering
    \includegraphics[scale=.42]{#1}\caption{#2}\label{#1}
    \end{figure}
  \begin{figure}[H]\centering
    \includegraphics[scale=.42]{#3}\caption{#4}\label{#3}
    \end{figure}
  \begin{figure}[H]\centering
    \includegraphics[scale=.42]{#5}\caption{#6}\label{#5}
    \end{figure}

  \end{multicols}}

% #1 - nazwa pliku
% #2 - opis
% #3 - źródło
% #4 - źródło danych
\newcommand{\fig}[2]{
\begin{figure}[H]
\includegraphics[width=\textwidth]{#1}
\caption{#2}
\label{#1}
\end{figure}}

\newcommand{\figpage}[3]{
\begin{figure}[H]\centering
\includegraphics[height=#1\textheight]{#2}
\caption{#3}
\label{#2}
\end{figure}}

\newcommand{\figside}[3]{
\begin{multicols}{2}
  \begin{figure}[H]
  \includegraphics[width=0.5\textwidth]{#1}
  \caption{#2}
  \label{#1}
  \end{figure}
  \columnbreak
  #3
\end{multicols}}

\newcommand{\figsides}[4]{
\begin{multicols}{2}
  \begin{figure}[H]
  \includegraphics[width=0.5\textwidth]{#1}
  \caption{#2}
  \label{#1}
  \end{figure}
  \columnbreak
  \begin{figure}[H]
  \includegraphics[width=0.5\textwidth]{#3}
  \caption{#4}
  \label{#3}
  \end{figure}
\end{multicols}}

\newcommand{\figsidesTri}[6]{
\begin{multicols}{3}
  \begin{figure}[H]
  \includegraphics[width=0.33\textwidth]{#1}
  \caption{#2}
  \label{#1}
  \end{figure}
  \columnbreak

  \begin{figure}[H]
  \includegraphics[width=0.33\textwidth]{#3}
  \caption{#4}
  \label{#3}
  \end{figure}
  \columnbreak

  \begin{figure}[H]
  \includegraphics[width=0.33\textwidth]{#5}
  \caption{#6}
  \label{#5}
  \end{figure}

\end{multicols}}

\newcommand{\tblside}[3]{
\begin{multicols}{2}
  \begin{table}[H]
  \includegraphics[width=0.5\textwidth]{#1}
  \caption{#2}
  \label{#1}
  \end{table}
  \columnbreak
  #3
\end{multicols}}

\newcommand{\tblsides}[4]{
\begin{multicols}{2}
  \begin{table}[H]
  \includegraphics[width=0.5\textwidth]{#1}
  \caption{#2}
  \label{#1}
  \end{table}
  \columnbreak
  \begin{table}[H]
  \includegraphics[width=0.5\textwidth]{#3}
  \caption{#4}
  \label{#3}
  \end{table}
\end{multicols}}



%%%%% więcej możliwości w~dokumentacji amsthm



%%%%%%%%%%%%%%%%%%%%%%%%%%%%%%%%%%%%%%%%%5
%%%%%%%%%%%%%%%%%%%%%%%%%%%%%%%%%%%%%%%%%%
%%%%%%%%% wcięcie akapitowe %%%%%%%%%%%%%%
%%%%%%%%%%%%%%%%%%%%%%%%%%%%%%%%%%%%%%%%%%
%%%%%% ustawić w~zaleceń i~gustu %%%%%%%%%
%%%%%%%%%%%%%%%%%%%%%%%%%%%%%%%%%%%%%%%%%%
%%%%%%%% zalecenie na stronie wydziałowej
%%%%%%%% było 1.25cm i wyglądało jakoś 
%%%%%%%% śmiesznie duże, więc spłoszony nieco
%%%%%%%% wpisałem 1cm, ale uważny czytelnik już
%%%%%%%% zapewne się domyśli, że podmiana napisu 
%%%%%%%% =1cm na =1.25cm sprawi, że wcięcia na początku
%%%%%%%% akapitu ustawią się na (nieco przydużą)
%%%%%%%% wartość 1.25cm 

\parindent=1cm



%%%%%%%%%%%%%%%%%%%%%%%%%%%%%%%%
%%%%% tu pewne poluzowanie rozmieszczenia elementów tabelek
%%%%% możecie sobie poeksperymentować, by dopasować do swych
%%%%% gustów, a przede wszystkim gustów promotorów (promotorek)
  \tabcolsep=4mm          
  %\renewcommand\arraystretch{1.3}
%%%%%%%%%%%%%%%%%%%%%%%%%%%%%%%%%%



%%%%%%%%% teraz żywa pagina (aka 'running headline') i~numerowanie stron
%%%%%%%%%%%%%%%%%%%%%%%%%%%%%%%%%%%%%%%%%%%%%%%%%%%%%%%%%%%%%%%%%%%%%%%%
%%%%%na górze mają być śródtytuły, na dole (po stronie zewneętrznej)
%%%%%numery stron. Poszedłem kapkę dalej i~na stronach ropoczynających
%%%%%rozdział nie ma paginy (górki).
%%%%% Oczywiście jeśli ostatnia strona
%%%%% jest pusta (uzupełnia jeno parzystość) to tam żadnej stopki ani 
%%%%% górki byc mnie może - ma być pusta.
%%%%%%%%%%%%%%%%%%%%%%%%%%%%
\pagestyle{fancy}
\fancyhead{}% oczyszczenie
\fancyhead[RO]{\rightmark} %% na nieparzystych 'podległe' śródtytuły
\fancyhead[LE]{\leftmark} %% na parzystych 'ważniejsze'
\fancyfoot{}% oczyszczenie
\fancyfoot[RO,LE]{\arabic{page}}  %% numer na dole (po prawej na
%% nieparzystych, po lewej na parzystych)
\renewcommand\headrulewidth{0.4pt} %%% cienka hrulka oddzielająca paginę
                                    %%% od kolumny tekstu
\fancypagestyle{closing}{%%%%%% to styl dla stron zamykających rozdział
\fancyhead{}% oczyszczenie
\fancyhead[RO]{\rightmark} %% na nieparzystych 'podległe'
\fancyhead[LE]{\leftmark} %% na parzystych 'ważniejsze'
\fancyfoot{}% oczyszczenie
\fancyfoot[RO,LE]{\arabic{page}}  %% numer na dole (po prawej na
                                  %% powyższą linijkę usuń jeśli nie
				  %% chcesz numerów na niepełnych
				  %% kolumnach (zamykających rozdział)
\renewcommand\headrulewidth{0.4pt}
}
\fancypagestyle{opening}{%%% styl stron rozpoczynających rozdział
\fancyhead{}% oczyszczenie
\fancyfoot{}% oczyszczenie
\fancyfoot[RO,LE]{\arabic{page}}  %% numer na dole (po prawej na
\renewcommand\headrulewidth{0pt}
}
\fancypagestyle{plain}{%%%% styl zwykły, niektóre konstrukcje
                       %%%% (typu \titlepage, którego ja tu nie używam
                       %%%% ale może są jakieś inne o których nawet nie chce 
                       %%% mi się myśleć, więc dla spokoju robię to po swojemu
\fancyhead{}% oczyszczenie
\fancyfoot{}% oczyszczenie
\fancyfoot[RO,LE]{\arabic{page}}  %% numer na dole (po prawej na
\renewcommand\headrulewidth{0pt}
}

%%%%%%%%%%%%%%%%%%%%%%%%%%%%%%%%%5
%%%%%%%%%%%%%%%%%%%%%%%%%%%%%%%%%%
%%% lekka modyfikcja 'markow' do paginy
%%% uznalem, ze jesli ktos nie da \section (np we wstepnie czy
%%% podsumowaniu to niech na obu sronach w~paginie pojawia sie tytuł
%%% chaptera, bo standardowo, to na nieparzystej stronie w takiej sytuacji
%%% nad górną linią ziałaby pustka, co mogłoby wprowadzać konsternację
\makeatletter
    \def\chaptermark#1{%
      \markboth{%
        \ifHeadingNumbered
     \if@mainmatter
     \@chapapp\
            \thechapter.\enspace
          \fi
        \fi
        #1}{%
        \ifHeadingNumbered
     \if@mainmatter
     \@chapapp\
            \thechapter.\enspace
          \fi
        \fi
        #1%
	}}%
    \def\sectionmark#1{%
      \markright{%
        \ifHeadingNumbered \thesection.\enspace \fi
        #1}}
%%%%%%%%%%%%%%%%%%%%%%%%%%%%%%%%%%%%%%%%%%%%%%%
%%%%%%%%%%%%%%%%%%%%%%%%%%%%%%%%%%%%%%%%%%%%%%%%
%%%%%%%%%%%% wielkości czcionek dla chapter i~section
%%%%%%%%%%%% 16 dla rozdziału, 14 dla podrozdziału - te domyślne
%%%%%%%%%%%% w klasie mwbk były całkiem ładne, ale żeby nie było
%%%%%%%%%%%% że nie potrafię ustawić
%%%%%%%%%%%%%%%%%%%%%%%%%%%%%%%%%%%%%%%%%%%%%%%%%%%
\SetSectionFormatting[breakbefore,wholewidth]{chapter}
        {0\p@}
        {\FormatRigidChapterHeading{6.4\baselineskip}{12\p@}%
	{\large\@chapapp\space}{\fontsize{16}{19}\selectfont}}
        {1.6\baselineskip}
\SetSectionFormatting{section}
        {24\p@\@plus5\p@\@minus2\p@}
	{\FormatHangHeading{\fontsize{14}{16}\selectfont}}
        {10\p@\@plus3\p@}
\makeatother	



%%%%%%%%%%%%%%%%%%%%%%%%%%%%%%%%%%%%%%%%%%%%%%
%%%%%%%%%%%%%%%%%%%%%%%%%%%%%%%%%%%%%%%%%%%%%%
%%%%%%%%%%%%%% jakies inne pomocnicze definicje, ja na przykład lubię
% \R
%%%%%%%%%%%%%%%%%%%%%%%5
%%%%%%%%%%%%%%%%%%%%%%%
%%%% tak naprawdę są t potrzebne tylko po to
%%%% by zadziałały przykłady poniżej w tekście
%%%% które w sposób dość losowy zostały 
%%%% pobrane z jakichś moich starych plików
%%%%%%%%%%%%%%%%%%%%%%%%%%%%%%%%%%
%%%%%%%%%%%%%%%%%%%%%%%%%%%%%%%%%%%
%%%% w realnej pracy te poniższe śmieci możecie oczywiście
%%%% usunąć
%%%%%%%%%%%%%%%%%%%%%%%%%%%%
\newcommand\R{\mathbb{R}}
\newcommand{\ff}{\mathbf{f}}
\newcommand{\hh}{\mathbf{h}}
\newcommand{\xx}{\mathbf{x}}
\newcommand{\yy}{\mathbf{y}}
\newcommand{\zz}{\mathbf{z}}
\newcommand{\gggg}{\mathbf{g}}
\newcommand{\skalar}[2]{\pmb{\langle}#1,#2\pmb{\rangle}}
%%%%%%%%%%%% koniec tych dodatkowych definicji

%%%%%% trocę więcej ``luzu'' przy rozmieszczaniu {fgur} i~{table}

 \renewcommand{\topfraction}{0.9}	% max fraction of floats at top
    \renewcommand{\bottomfraction}{0.8}	% max fraction of floats at bottom
    %   Parameters for TEXT pages (not float pages):
    \setcounter{topnumber}{2}
    \setcounter{bottomnumber}{2}
    \setcounter{totalnumber}{4}     % 2 may work better
    \setcounter{dbltopnumber}{2}    % for 2-column pages
    \renewcommand{\dbltopfraction}{0.9}	% fit big float above 2-col. text
    \renewcommand{\textfraction}{0.07}	% allow minimal text w. figs
    %   Parameters for FLOAT pages (not text pages):
    \renewcommand{\floatpagefraction}{0.7}	% require fuller float pages
    % N.B.: floatpagefraction MUST be less than topfraction !!
    \renewcommand{\dblfloatpagefraction}{0.7}	% require fuller float pages
    % remember to use [htp] or [htpb] for placement

    
%%% DWA proste polecenia służące do ujednolicenia podawania źródeł przy rysunkach i~tabelkach    
    
    \newcommand\zrodlo[1]{\par\vspace{-3mm}{\small\textit{Źródło: }#1 }}
    \newcommand\zrodlotab[1]{{\par\vspace{2mm}\small\textit{Źródło: }#1 }}

\raggedbottom   %%% to znaczy, że nie będzie siłowego wyrównywania typowych
                %%     stron do jednakowej wysokości

\linespread{1.3}





\newcommand{\D}[3]{
  \begin{defi}
    \textbf{#2} --- #3
    \label{#1}
    \end{defi}
}


\newcommand{\TODO}[1]{\footnotesize\textcolor{white}{\colorbox{red}{TODO: \texttt{#1}}}}

\begin{document}

\thispagestyle{empty}%brak numeracji


%%%%%%%%%%%%%%%%%%%%%%%%%%%%%%%%%%%%%%%%%%%%%%%%%%%%%%%%%%%%%%%
%%%%%tytuły definiuje jako makrodefinicje, gdyż zamierzam je%%%
%%%%%powtórzyć na stronie ze streszczeniami, to nic nie boli%%%
%%%%%a gwarantuje, że będą one takie same, i~tak ma być.%%%%%%%
%%%%%%%%%%%%%%%%%%%%%%%%%%%%%%%%%%%%%%%%%%%%%%%%%%%%%%%%%%%%%%%P.Wlaź
\newcommand\tytul{Dyfuzja wiedzy innowacyjnej w czasie i przestrzeni
                  na podstawie patentów z ostatnich dziesięciu lat}

\newcommand\tytulangielski{Diffusion of innovative knowledge 
                           in time and space based on patents 
                           from the last ten years}

\noindent
\hspace*{-3mm}\includegraphics[width=8.67cm]{logo.pdf}
\fontfamily{qhv}\fontsize{12pt}{15pt}\selectfont

\vfil 
\noindent Katedra Matematyki Stosowanej

\vfil\vfil\vfil\vfil


\fontsize{40pt}{50pt}\selectfont
\noindent Praca inżynierska

\fontsize{12pt}{15pt}\selectfont


\vfil
\noindent
na kierunku \emph{inżynieria i~analiza danych}
\vfil\vfil

\vspace{2cm}
\fontsize{16pt}{18pt}\selectfont
\noindent \tytul

\vspace{1cm}
\fontsize{16pt}{18pt}\selectfont
\noindent \tytulangielski

\vfil\vfil\vfil\vfil
\fontsize{16pt}{20pt}\selectfont

\noindent 
Adam Jakubczak

\vfil
\fontsize{12pt}{15pt}\selectfont
\noindent
numer albumu: 098750


\vfil

\noindent
promotor: dr inż. Korneliusz Pylak

\vfil\vfil\vfil

\fontsize{9pt}{12pt}\selectfont

\noindent
Lublin 2025

\normalsize \rm

\tableofcontents

\chapter*{Wstęp}

Dyfuzja wiedzy innowacyjnej w czasie i przestrzeni to proces
rozprzestrzeniania się nowych technologii pomiędzy podmiotami
w lokalnym otoczeniu.

Wiedza innowacyjna jest kluczowym elementem rozwoju gospodarczego.
To jej przypisuje się siłę napędową dla wzrostu gospodarczego
w nowoczesnych gospodarkach. Zrozumienie procesów dyfuzji 
wiedzy innowacyjnej może pozwolić na budowanie lepszych 
warunków dla rozwoju gospodarczego.

Patenty są dobrym wskaźnikiem tego zjawiska z 2 powodów:
po pierwsze dotyczą wyłacznie innowacji --- takie jest ich zadanie,
po drugie zawierają informacje na temat ich twórców, co pozwala
na stwierdzenie o ich lokalizacji, która jest kluczowym elementem
w procesie dyfuzji wiedzy.

\todonote{wstęp do poprawy na koniec}

\chapter{Przegląd literatury}\label{ch:intro}

\section{Wprowadzenie do tematu dyfuzji wiedzy}

Dyfuzja wiedzy innowacyjnej jest przedmiotem zainteresowania
wielu badaczy. Przegląd dzieł związanych z tematem od różnych
autorów pozwala na zrozumienie tego jakie są kluczowe elementy
procesu dyfuzji wiedzy innowacyjnej i czym właściwie ona jest.

Nonaka \cite{No-98} rozdziela informację od wiedzy na podstawie
tego jak została zaadaptowana przez podmioty i w jakim
kontekście jest umieszczona. Pogląd rozdziału informacji
od wiedzy powielają także Morone i Taylor \cite{Mo-Ta-09}.

Nonaka \cite{No-98} rozdziela także wiedzę na jawna i ukrytą.
Pathirage \cite{Pa-08} wskazuje, że taki rozdział jest dominujący 
w literaturze przedmiotu. Alwis i Harmann \cite{Al-Ha-08} dają dobry 
wgląd w to czym jest wiedza ukryta poprzez zagłębienie się 
w literaturę. Wskazują, że wiedza ukryta jest kluczowym 
elementem innowacyjności firm i jest ściśle powiązana z 
procesem dyfuzji wiedzy innowacyjnej.

Częstym cytatem w literaturze przedmiotu jest stwierdzenie
Michaela Polanyiego: \textit{Wiemy więcej niż potrafimy powiedzieć},
gdzie z faktu, że \textit{wiemy więcej} wnioskujemy o istnieniu czegoś
ponad to czym jest wiedza jawna - tego \textit{co potrafimy powiedzieć}.

Dużą rolę w zachowaniu konkurencyjności firm przypisuje się
zarządzaniu wiedzą, a szczególnie wiedzą ukrytą \cite{No-98}.
Nonaka, jak inni \cite{Mo-16}, \cite{Ga-Th-14}, definiuje wiedzę ukrytą
jako zaprzeczenie wiedzy jawnej. Wiedza jawna to taka
przechowywana na wszelkich nośnikach, ale też możliwa w artykulacji
do innych osób. Wiedza ukryta, jako jej przeciwieństwo,
jest trudna albo niemożliwa do wyrażenia słowami albo symbolami.
Jest nabywana podczas praktyki albo obserwacji \cite{No-98}.

Nonaka \cite{No-98} definiuje model procesu tworzenia wiedzy w firmie, 
jako 4-etapową spiralę, w której kolejne kroki to: nabywanie wiedzy
ukrytej, jej synteza w wiedzę jawną, standaryzacja wiedzy jawnej
i adaptacja wiedzy jawnej przez pracowników do wiedzy ukrytej.
Podobny pogląd na powstawanie wiedzy przedstawiaja Morone i 
Taylor \cite{Mo-Ta-09} stwierdzając, że wiedza zawsze jest początkowo
wiedzą ukrytą, a dopiero w procesie jej artykulacji staje się
wiedzą jawną.

Istotą przepływu wiedzy ukrytej jest to jakie warunki panują
w firmie. Nonaka \cite{No-98} wskazuje jako modelowe, firmy japońskie, w
których panuje redundancja informacji, co sprawia, że pracownicy
posiadają podobny zestaw wiedzy ukrytej. Alwis i Hartmann \cite{Al-Ha-08}
także przypisują jakość dyfuzji wiedzy ukrytej do organizacji
firm twierdząc, że likwidacja barier wewnętrznych w firmie jest
kluczowa dla efektywnego przepływu wiedzy.

Bathelt i Feldman \cite{Ba-Fe-11} także rozważają powstawanie wiedzy
innowacyjnej jako proces przerabiania aktualnej wiedzy na nową.
W nieścisły sposób łączy się to z modelem Nonaka \cite{No-98}, gdzie
wiedza także ulega ciągłej transformacji. 

Dalej \cite{Ba-Fe-11}, na podstawie literatury, 
stwierdzają o ograniczeniach przestrzennych jakie wiążą się z
rozprzestrzenianiem się wiedzy. Wynikają one między innymi z
przywiązaniem ludzi do ich miejsca zamieszkania, czy ulokowaniem
środków firm, czy całego sektora w jednym regionie.

Samo rozprzestrzenianie się wiedzy można podzielić na 2 typy:
dyfuzję i wymianę \cite{Mo-Ta-09}. Wymiana polega na
przepływnie wiedzy między podmiotami na podstawie uzgodnionych
oraz świadomych działań w trakcie których następuje symbiotyczna
interakcja, w której jedna strona zyskuje wiedzę, a druga
wiedzę inną albo korzyści niezwiązane z samą wiedzą. W kontrze
do wymiany, dyfuzja to proces nieświadomego przepływu samej wiedzy.
Morone i Taylor wskazują, że w procesie dyfuzji, podmioty będące
odbiorcami mogą wykorzystywać mimowolne przepływy wiedzy na
swoją korzyść. Dalej zastrzegają, że takiemu procesowi dyfuzji
podlega wiedza ukryta, a jej przyswojenie wymaga zdolności 
(pojemności) absorpcyjnych (ang. \textit{absortive capacity}).

Taylor i Morone \cite{Mo-Ta-09} rozkładają dyfuzję na 3 procesy:
rozlewanie \foreign{ang}{spillover}, transfer oraz integrację. Rozlewanie
i transfer to procesy podobne z tą różnicą, że transfer jest
określony jako przepływ z pierwotną intencją jego zaistnienia.
Oba te procesy wymagają istniejącej wcześniej wiedzy, która
umożliwia absorbcję nowej wiedzy. Integracja z kolei to proces,
w który istniejąca wiedza jest aplikowana w innym kontekście.

Klarl \cite{Kl-14} twierdzi, że dyfuzja wiedzy to proces społeczny,
który napędzają, po pierwsze: więzi między ludźmi i grupami,
a po drugie cechy indywidualne ludzi. To jak silne są więzi
w grupie oraz między grupami ma znaczący wpływ na szybkość
rozprzestrzeniania się wiedzy. Klarl twierdzi, że można
wyróżnić w populacji grupy o różnych miarach dyfuzji wewnątrz,
jak i po między grupami oraz z zewnątrz. Dodatkowo zakłada
ograniczenia przestrzenne związane z dyfuzją wiedzy - sieć
jest tym słabsza im bardziej jest rozproszona w przestrzeni.

Boone, Ganeshan, i Hicks \cite{Bo-Ga-Hi-08} przedstawiają znane już zjawisko
starzenia się wiedzy jako proces, w którym wiedza nagromadzona
w organizacji przekłada się na coraz mniejszą wartość dla firmy.
Mimo, że innowacje są efektem nadbudowywania wiedzy to należy
się domyślić, że jej starzenie skutecznie temu zapobiega. 
Zaznaczają też, że szybkość starzenia zależy od branży, ale też
działań jakie są podejmowane aby mu zapobiegać.

Z podobną tezą występują Grubler i Nemet \cite{Gr-Ne-13}, określając
zjawisko starzenia się jako \textit{oduczanie} - przeciwieństwo nauki.
Podają ku temu 2 przyczyny: zanikanie wiedzy związane z rotacją
pracowników oraz zanikanie wiedzy związane z innowacjami, które
następują na tyle szybko, że nie sposób dostosować do nich starej
wiedzy. Zanikanie związane z rotacją wiąże się z tym, że
przeniesienie, czy zwolnienie pracownika sprawia, że ten,
posiadający wiedzę ukrytą, nie używa jej już w organizacji.
Grubler i Nemet zaznaczają, że to właśnie wiedza ukryta jest
szczególnie wrażliwa na starzenie.

Frenken, Van Ooor i Verburg \cite{Fr-Oo-Ve-07} opisują pojęcie pokrewnej i
niepokrewnej różnorodności wiedzy. Różnorodność pokrewna dotyczy
wiedzy z jednej dziedziny, a niepokrewna z różnych dziedzin.
Badacze wskazują, że różnorodność pokrewna sprzyja bardziej
dyfuzji wiedzy niż różnorodność niepokrewna.



\section{Patenty w literaturze przedmiotu}

Patenty są istotnym wskaźnikiem dyfuzji wiedzy innowacyjnej
między środowiskami akademickimi, a rynkiem \cite{Lo-Br-08}.
Dają one informacje o tym jak wiedza zdobywana na uczelniach
przenika do firm i jak jest wykorzystywana w praktyce.

Sorenson i Fleming \cite{So-Fl-04} wskazują w swojej pracy na związek
między cytowaniami w patentach, a tym jak te same patenty są
cytowane. W badaniu wykazali, że patenty, które cytują częściej
są także częściej cytowane. Wynika z tego, że przepływ wiedzy
jest przyśpieszany przez uwzględnienie wiedzy jaka została
już wcześniej zgromadzona.

Acs, Anselin i Varga \cite{Ac-An-Va-02} określają związek między patentami,
a innowacjami jako zależność nieidealną, gorszą od literatury
badawczej ale porównywalną z nią. Wskazują na zaletę jaką może
być uwzględnienie przez patenty innowacji z mniejszych organizacji,
co często jest pomijane w opracowaniach naukowych.

Verspagen i Schoenmakers \cite{Ve-Sc-00} podają 2 powody, dla których
patenty mogą być solidnym wyznacznikiem dyfuzji w ekonomii:
w kontekście dużych firm zakładają istnienie zespołów R\&D do
analizy patentów na rynku, a co za tym idzie nabywania wiedzy
o nich. Drugi powód to cytowania patentowe, które wskazują
na związki między wynalazcami - duże podobieństwo jest miarą
tego jak wiedza ulega dyfuzji na tym wymiarze.


\section{Podsumowanie}

Na podstawie wcześniejszego przeglądu literatury można wyprowdzić
pewne ogólne definicje, co do których będą stawiane tezy.

Jednym z podstawowych założeń jest rozdział wiedzy od samej informacji.
Logicznie można też stwierdzić, że wiedza jest podzbiorem informacji,
która wyróżnia się tym, że jest interpretowana przez człowieka w danym
kontekście.

\Needspace{5\baselineskip}
\begin{defi}
Informacja ($I$) --- ogólny termin odnoszący się do wszelkich form wiedzy, 
umiejętności, doświadczeń, przekonań przechowywanych na różnych nośnikach 
albo w ludzkiej pamięci - świadomie lub nieświadomie.
\end{defi}

\begin{defi}
Wiedza ($W$) --- informacja zinterpretowana przez człowieka w danym kontekście.
\begin{center}
\begin{math}
W \subset I
\end{math}
\end{center}
\end{defi}

To czym jest informacja można odnieść zarówno do danych
przechowywanych na serwerach, tez zawartych w dziełach naukowych,
jak i umiejętności praktycznych wszelkiego rodzaju, np.
balans ciałem podczas jazdy na rowerze. W odróżnieniu od informacji, 
wiedza jest zawsze związana z jakimś kontekstem oraz tym 
jak jest ona interpretowana przez daną jednostkę.

\Needspace{5\baselineskip}
\begin{defi}
Wiedza jawna ($W_j$) \foreign{ang}{explicit knowledge} --- wiedza
zapisywana na nośnikach informacji i przekazywana za pomocą
języka, symboli, obrazów, dźwięków.
\end{defi}

Wiedza jawna jest ogólnie dostępna z samej jej definicji - przekazywana
w słowach ulega upowszechnieniu w postaci rozmów i wszelkich mediów: 
książek, artykułów, prezentacji, filmów, itp. 

\Needspace{5\baselineskip}
\begin{defi}
Wiedza ukryta $W_u$ --- przeciwieństwo wiedzy jawnej:
wiedza, której nie da się uchwyciś słowami czy symbolami,
nabywana podczas praktyki.
\begin{center}
\begin{math}
W_u = W \setminus W_j
\end{math}
\end{center}
\end{defi}

W anglosaskiej literaturze określana jest jako \textit{tacit knowledge}.
Nazwa pochodzi od łacińskiego \textit{tacitus} - \textit{milczący}.
Dobrze oddaje naturę tej wiedzy, bo oprócz tego, że jest ona
trudna do uchwycenia, to jej specyfika wynika właśnie z niemożności
do jej artykulacji w języku. Jej specyfikę podsumowuje popularne 
w literaturze stwierdzenie \textit{Wiemy więcej niż potrafimy powiedzieć}, 
Michael Polanyi.

\Needspace{5\baselineskip}
\begin{defi}
Wiedza innowacyjna $W_i$ --- wiedza rozszerzająca dotychczasową
wiedzę, która prowadzi do rozwiązania aktualnych problemów.
\begin{center}
\begin{math}
W_i \subset W\qquad
W + W_i = W' \supset W
\end{math}
\end{center}
\end{defi}

Wiedza innowacyjna jest zawsze w jakimś stopniu nowa, ale
niekoniecznie musi być to coś zupełnie nowego. Może to być
również modyfikacja istniejącej wiedzy i najczęściej tak
właśnie powstaje - poprzez nadbudowywanie na istniejącej wiedzy.

\begin{defi}
Dyfuzja wiedzy innowacyjnej --- proces czasowy przepływu 
wiedzy innowacyjnej między podmiotami, występujący szczególnie
w lokalnej przestrzeni.
\end{defi}

Jest to \textbf{mimowolna} transmisja wiedzy w czasie i przestrzeni,
która wiąże się z przepływem wiedzy z ośrodków o dużej
koncentracji do sąsiadujących z nim podmiotów o mniejszej
koncentracji wiedzy.

Kluczowa w dyfuzji wiedzy innowacyjnej jest wymiana wiedzy
twarzą w twarz. Wielu autorów podkreśla, że wiedza ukryta,
która jest kluczowym elementem innowacyjności. Nie ulega
ona transmisji poprzez media, ponieważ jest ona niewerbalna.
W konsekwencji do jej rozprzestrzeniania potrzebne są spotkania
i interakcje między ludźmi podczas jej praktyki. Publikacje
naukowe zwracają też uwagę na istotę zaangażowania w rozszerzaniu
się tego rodzaju wiedzy, które jest łatwiej zainicjować w
sytuacjach socjalnych niż rozgrywających się na polu wirtualnym,
czy w postaci innych środków komunikacji.

Ograniczenia występowania tego rodzaju sytuacji są dość oczywiste i
wynikają między innymi z przestrzeni w jakiej operują ludzie.
Zazwyczaj są to ograniczenia do przestrzeni lokalnej, co sprawia,
że dyfuzja wiedzy innowacyjnej jest ograniczona do pewnego obszaru.
Nawet w lokalnych warunkach mogą dochodzdić dodatkowe obciążenia
takie jak bariery fizyczne, kulturowe czy wynikające z hierarchii
organizacyjnej albo struktury społecznej.


\Needspace{15\baselineskip}
\subsection{Patent jako narzędzie ochrony wiedzy innowacyjnej}

\begin{defi}
\textbf{Patent} --- forma ochrony prawnej udzielana na nieużywane, 
albo nieopatentowane wcześniej wynalazki, które wykazują 
zastosowanie w przemyśle. Powinny też być nieoczywiste dla 
ekspertów dziedzinowych.
\end{defi}

W Polsce przydzielaniem i ochroną patentów zajmuje się \ac{UPRP}, 
na poziomie europejskim jest to \ac{EPO}, a światowa organizacja 
to \ac{WIPO}.

Wartość patentów jako wskaźników dyfuzji wiedzy innowacyjnej
wynika z charakterystyki patentów jako narzędzi ochrony wiedzy.
Z definicji prawnej, patenty muszą reprezentować wiedzę innowacyjną
i być nowe, nieoczywiste oraz nadające się do praktycznego
zastosowania. Z tego właśnie powodu można przypuszczać, że
w pewnym stopniu oddają one stan faktyczny w dziedzinie innowacji.
Kolejnym atutem patentów jest ich klasyfikacja.
Patenty są klasyfikowane według międzynarodowej klasyfikacji
patentowej \ac{IPC}, dzięki czemu analizować je pod kątem dziedzinowym.
Dane dotyczące patentów są dostępne publicznie, co daje 
szerokie możliwości do ich analizowania oraz publikowania efektów
prac. Ponadto każdy patent jest przypisany do właściciela, oraz
na potrzeby formalne zawiera informacje o autorach, co pozwala
na budowanie sieci współpracy między podmiotami.

Mimo tych zalet, patenty mają swoje ograniczenia. Przede wszystkim
nie wszystkie wynalazki są opatentowane. Te opatentowane z kolei
mogą nie być wykorzystywane w praktyce, co sprawia, że nie
wszystkie innowacje są reprezentowane w danych patentowych.
Ponadto, patenty są zazwyczaj zgłaszane przez duże firmy, co
może prowadzić do zniekształcenia obrazu innowacyjności w
danej dziedzinie.

\begin{acronym}
  \acro{UPRP}{Urząd Patentowy Rzeczypospolitej Polskiej}
  \acro{EPO}{European Patent Office}
  \acro{WIPO}{World Intellectual Property Organization}
  \acro{MKP}{Międzynarodowa Klasyfikacja Patentów}
  \acro{IPC}{International Patent Classification}
  \acro{API}{Application Programming Interface}
  \acro{URI}{Uniform Resource Identifier}
  \acro{URL}{Uniform Resource Locator}
  \acro{OCR}{Optical Character Recognition}
  \acro{XML}{Extensible Markup Language}
\end{acronym}

\chapter{Źródła danych patentowych}\label{ch:data}

Jak już wspomniano, w Polsce centralnym organem odpowiedzialnym
za przyznawanie patentów jest \acf{UPRP}. Oprócz ochroną patentową
oraz publikowaniem informacji o patentach, urząd prowadzi
bazę danych patentów, która jest dostępna publicznie przy użyciu \ac{API}.
Pozwala to na automatyczne pobieranie danych przy pomocy skryptów.

\section{\ac{UPRP}}\label{sec:UPRP}

Identyfikacja patentów w tym systemie to przyporządkowanie
każdego patentu do 6-cyfrowego numeru, który jest unikalny dla
każdego zgłoszenia. Pobieranie danych dotyczących wszystkich patentów 
można więc przeprowadzić wysyłając zapytanie do interfejsu o każdy
patent po kolei. W odpowiedzi otrzymuje się dane w formacie \ac{XML}.

\bigskip
\begin{figure}
\centering
\begin{tikzpicture}
	\draw[draw=black, thin, solid] (-3.00,3.00) rectangle (-1.00,2.00);
	\draw[draw=black, thin, solid] (0.00,3.00) rectangle (2.00,2.00);
	\node[black, anchor=south west] at (-3.06,2.25) {skrypt};
	\node[black, anchor=south west] at (-0.06,2.25) {\ac{API}};
	\draw[draw=black, thin, solid] (-2.00,2.00) -- (-2.00,-3.00);
	\draw[draw=black, thin, solid] (1.00,2.00) -- (1.00,-3.00);
	\node[black, anchor=south west] at (-2.06,0.25) {kod 6-cyfrowy};
	\draw[draw=black, -latex, thin, solid] (-2.00,0.00) -- (1.00,0.00);
	\node[black, anchor=south west] at (-1.06,-1.75) {plik XML};
	\draw[draw=black, -latex, thin, solid] (1.00,-2.00) -- (-2.00,-2.00);
	\draw[draw=black, thin, solid] (-3.00,-3.00) rectangle (-1.00,-3.50);
	\draw[draw=black, thin, solid] (0.00,-3.00) rectangle (2.00,-3.50);
\end{tikzpicture}
\caption{Schemat pobierania danych z \ac{API} \ac{UPRP}}
\end{figure}

\subsection{Dane w formie surowej}
\label{sec:profilowanie-UPRP}

Wstępna struktura to formacja zagnieżdzonych obiektów z parametrami 
oraz list obiektów innego typu. Zagnieżdżenie oznacza, że obiekt znajduje się
w innym obiekcie, a parametr to wartość skalarna przyporządkowana danej obserwacji.
Dodatkowo każdy dokument traktujemy także jako obiekt.
Ze względów praktycznych dane wymagają wstępnej obróbki.
Pożądana struktura to listy obiektów ustalonego typu, z parametrami, 
bez żadnych zagnieżdżeń.

\bigskip

\begin{figure}[H]
\begin{tikzpicture}
	\draw[draw=black, thin, solid] (-6.00,5.00) rectangle (8.00,-6.00);
	\draw[draw=black, thin, solid] (3.00,-2.00) rectangle (0.00,-5.00);
	\draw[draw=black, thin, solid] (-1.00,-2.00) rectangle (-4.00,-5.00);
	\draw[draw=black, thin, solid] (-4.00,3.00) rectangle (-1.00,0.00);
	\draw[draw=black, thin, solid] (0.00,3.00) rectangle (3.00,0.00);
	\draw[draw=black, thin, solid] (-5.00,4.00) rectangle (7.00,-1.00);
	\node[black, anchor=south west] at (-6.06,5.25) {$A_1$};
	\node[black, anchor=south west] at (-0.06,3.25) {$C_2$};
	\node[black, anchor=south west] at (-4.06,3.25) {$C_1$};
	\node[black, anchor=south west] at (-4.06,-1.75) {$B_2$};
	\node[black, anchor=south west] at (-5.06,4.25) {$B_1$};
	\node[black, anchor=south west] at (-0.06,-1.75) {$G$};
	\node[black, anchor=south west] at (-3.56,2.25) {$c_1: x_1$};
	\node[black, anchor=south west] at (-3.56,1.25) {$c_b: x_2$};
	\node[black, anchor=south west] at (-3.56,0.25) {$c_r: x_3$};
	\node[black, anchor=south west] at (0.44,2.25) {$c_1: x_4$};
	\node[black, anchor=south west] at (0.44,0.25) {$c_3: x_5$};
	\node[black, anchor=south west] at (4.44,2.25) {$b_1: x_6$};
	\node[black, anchor=south west] at (-3.56,-2.75) {$b_1: x_7$};
	\node[black, anchor=south west] at (-3.56,-3.75) {$b_2: x_8$};
	\node[black, anchor=south west] at (-3.56,-4.75) {$b_3: x_9$};
	\node[black, anchor=south west] at (0.44,-2.75) {$g_1: x_{10}$};
	\node[black, anchor=south west] at (0.44,-3.75) {$g_2: x_{11}$};
	\node[black, anchor=south west] at (4.44,-2.75) {$a_1: x_{12}$};
	\node[black, anchor=south west] at (4.44,-3.75) {$a_2: x_{13}$};
\end{tikzpicture}
    \centering
    \caption{Przykładowy schemat danych zagnieżdżonych}
    \label{fig:przykład-danych}
\end{figure}

W powyższym przykładzie zaprezentowany jest dokument $A_1$ o nazwie $A$.
Dla dokumentów jest przyjęte, że jest to relatywna ścieżka lokacji,
w której się znajdują na urządzeniu.

Obiekt $A_1$ zawiera 2 parametry i 3 obiekty:

\begin{itemize}
  \item parametry mają klucze $a_1$ oraz $a_2$ z wartościami, odpowiednio, 
        $x_{12}$ oraz $x_{13}$;
  \item 2 obiekty $B_1$ oraz $B_2$ o nazwie $B$;
  \item obiekt typu $G_1$ nazywa się $G$.
\end{itemize}

\Needspace{7\baselineskip}
Ścieżki to ciągi nazw obiektów i parametrów, które trzeba przejść, 
aby dotrzeć do wartości. Dla wyżej wymienionych właśności będą to odpowiednio:

\begin{itemize}
  \item $a_1\colon\quad(A, a_1)$
  \item $a_2\colon\quad(A, a_2)$
  \item $B_1\colon\quad(A, B)$
  \item $B_2\colon\quad(A, B)$
  \item $G_1\colon\quad(A, G)$
\end{itemize}

Dla wartości $c_3$ w obiekcie $C$ ścieżką będzie $c'_3=(A, B, C, c_3)$.

Z faktu, że istnieją 2 obiekty jednego typu $B$ wnosimy, że jest 
to lista obiektów i na tej podstawie definiujemy typ $T_B$.
Wszystkie obiekty o identycznej ścieżce $(A, B)$. Schemat
przedstawia sytuację, w której mimo takiej samej ścieżki, co za tym
idzie, identycznego typu jest istotna różnica w strukturze obu obiektów.
W takiej sytuacji przyjmowane jest, że to większy zbiór kluczy definiuje
parametry obiektu, a różnice są przyjmowane jako braki w poszczególnych
obiektach: $B_1 = \{ b'_1 = x_7; b'_2 = x_8; b'_3 = x_9 \}$,
$B_2 = \{ b'_1 = x_6; b'_2 = \emptyset; b'_3 = \emptyset \}$.

Jak widać w strukturach $B_1$ oraz $B_2$ nie wymieniamy faktu,
że mogą zawierać obiekty o nazwie $C$. To $C_1$ oraz $C_2$ są
powiązane do obiektów nadrzędnych poprzez nadanie im nowych
parametrów, które zawierają unikalny klucz obiektu nadrzędnego.

Każdy obiekt ma generowany własny unikalny klucz.

Jeszcze jedną wartą uwagi sytuacją jest to jak traktowany jest 
obiekt $G_1$. Jak widać jest on jedyny w obiekcie $A_1$, stąd
nie będzie traktowany jako obiekt w danych wyjściowych. Jego
parametry będą parametrami obiektu $A_1$ co da obraz obiektu $A_1$
jako $\{a'_1 = x_{12}, a'_2 = x_{13}, g'_1 = x_{10}, g'_2 = x_{11} \}$.

W sytacji, w której inny dokument $A$ miałby przynajmniej dwa obiekty
o nazwie $G$ sytuacja była by analogiczna do sytuacji obiektów $C_1; C_2$



\subsubsection{Zasada profilowania danych}

Głównym problemem jest fakt, że dane są rozległe oraz różnią się 
w schemacie w zależności od czasu dodania oraz wolumenu i jakości informacji
jakie zostały wprowadzone w każdym przypadku.

Aby wyróżnić typy obiektów przyjmujemy, że każdy parametr ma dokładnie 1 obiekt,
a każdy obiekt ma dokładnie 1 parametr o danej ścieżce. Ścieżka to ciąg nazw
obiektów jakie należy przejść, aby dotrzeć do wskazanego obiektu w zagnieżdżonej
strukturze. Jeśli w obiekcie istnieją 2 obiekty o identycznej ścieżce to przyjmujemy,
że znajdują się w liście obiektów tego samego typu. Właśnie tak wyróżniamy typ.
Jeśli taka sytuacja dzieje się tylko dla części dokumentów, to i tak 
przyjmujemy je jako elementy wyróżnionego typu, a nie parametry obiektu
aby zapewnić homogeniczność danych. Podobnie dzieje się w przypadku braków:
jeśli część obiektów tego samego typu ma pewien parametr, a część go nie ma
to przyjmujemy, że i tak ten parametr jest charakterystyczny dla tego typu.
Część obiektów po prostu nie ma tej informacji.

Implementacja polega na zastosowaniu 2 etapów: wyszukiwania powtarzających się
ścieżek i wyróżniania ich jako typy obiektów, oraz etap drugi: zbieranie danych
do tabel zgodnie z wyróżnionymi typami obiektów.

W trakcie tych procesów występują także inne działania, takie jak określanie
unikalnych numerów identyfikacyjnych każdego obiektu, wiązanie obiektów ze sobą
za pomocą umieszczania identyfikatorów obiektów nadrzędnych w tych podrzęndnych,
każda tabela zawiera numer dokumentu, z którego pochodzi, numeracja obiektów i
dokumentów jest generowana na bieżąco i nie ma związku z danymi po za tym, że
to na podstawie ich steuktury powstaje, nazwy tabel i parametrów to ścieżki
jakimi są podpisane.

\Needspace{30\baselineskip}
Po zastosowaniu procesu, dla wcześniej wymienionego przykładu 
\cref{fig:przykład-danych} należy oczekiwać następujących niżej danych, 
patrz: \cref{fig:przykład-danych-po-profilowaniu}.

\begin{uwaga}
W tym przypadku zakładamy, że obiekt $G$ nigdy nie 
występuje wielokrotnie w dokumencie. To jak zachowuje się obiekt
występujący wielokrotnie przedstawia $B$ oraz $C$.
\end{uwaga}

\begin{figure}[H]\centering
\begin{tikzpicture}
\node[black, anchor=south west] at (-4.56,0.75) {$A$};
\node[black, anchor=south west] at (-4.06,-0.75) {
\begin{math}
\text{id}: 1,\quad
\text{doc}: 1,\quad
a'_1: x_{12},\quad
a'_2: x_{13},\quad
g'_1: x_{10},\quad
g'_2: x_{11}
\end{math}
};
\draw[draw=black, thin, solid] (-4.00,-1.00) rectangle (8.00,0.00);
\draw[draw=black, thin, solid] (-4.50,0.50) rectangle (8.50,-1.50);
\draw[draw=black, thin, solid] (-4.50,-2.50) rectangle (8.50,-6.00);
\draw[draw=black, thin, solid] (-4.00,-3.00) rectangle (8.00,-4.00);
\draw[draw=black, thin, solid] (-4.00,-4.50) rectangle (8.00,-5.50);
\node[black, anchor=south west] at (-4.56,-2.25) {$B$};
\node[black, anchor=south west] at (-4.06,-3.75) {
\begin{math}
\text{id}: 2,\quad
\text{doc}: 1,\quad
\text\&{A}: 1,\quad
b'_1: x_{6},\quad
b'_2: \emptyset,\quad
b'_3: \emptyset
\end{math}
};
\node[black, anchor=south west] at (-4.06,-5.25) {
\begin{math}
\text{id}: 3,\quad
\text{doc}: 1,\quad
\text\&{A}: 1,\quad
b'_1: x_{7},\quad
b'_2: x_{8},\quad
b'_3: x_{9}
\end{math}
};
\draw[draw=black, thin, solid] (-4.50,-7.00) rectangle (8.50,-10.50);
\draw[draw=black, thin, solid] (-4.00,-7.50) rectangle (8.00,-8.50);
\draw[draw=black, thin, solid] (-4.00,-9.00) rectangle (8.00,-10.00);
\node[black, anchor=south west] at (-4.56,-6.75) {$C$};
\node[black, anchor=south west] at (-4.06,-8.25) {
\begin{math}
\text{id}: 4,\quad
\text{doc}: 1,\quad
\text\&{B}: 2,\quad
c'_1: x_{1},\quad
c'_2: x_{2},\quad
c'_3: x_{3}
\end{math}
};
\node[black, anchor=south west] at (-4.06,-9.75) {
\begin{math}
\text{id}: 5,\quad
\text{doc}: 1,\quad
\text\&{B}: 2,\quad
c'_1: x_{4},\quad
c'_2: \emptyset,\quad
c'_3: x_{5}
\end{math}
};
\end{tikzpicture}
\caption{Schemat danych zagnieżdżonych (patrz: \cref{fig:przykład-danych}) po profilowaniu}
\label{fig:przykład-danych-po-profilowaniu}
\end{figure}



\subsubsection{Tworzenie uproszczonych nazw tabel i kolumn}

Ostatecznie otrzymujemy kilka tabel, które są powiązane między sobą identyfikatorami.
Ich nazwy oraz nazwy parametrów to ścieżki w oryginalnej strukturze. Aby ułatwić
czytelność danych wymagany jest kolejny etap aliasowania. Polega on na przypisaniu
unikalnej nazwy. Nazwy powstają ze ścieżek jako ich podciagi. Generowanie opiera się
o stworznie grafu drzewa, w którym każdy wierzchołek to fragment ścieżki. Wierzchołki
zawierają informacje, które ścieżki używają fragmentów przypisanych do nich.
Nazwy powstają przez pobranie ostatniego wierzchołka; dalej jeśli taka nazwa
jest unikalna zostaje zapisana, jeśli nie dodawany jest kolejny wierzchołek, który
nie został wcześniej użyty, jeśli to niemożliwe dodawany jest jakikolwiek wierzchołek,
a później sytuacja się powtarza do osiągnięcia unikalności. W przypadku porażki dodawany
jest iteracyjnie numer, który i tak zapewnia unikalność. Oryginalne ścieżki
wraz z nazwami są zapisane jako słownik, aby nie tracić informacji. Natomiast
tabele i ich kolumny przyjmują nowe unikalne nazwy prostsze w obróce.

\needspace{20\baselineskip}
Dla przykładu \ref{fig:przykład-danych} należy oczekiwać następujących ścieżek:

\begin{itemize}
\item $(A) \to (A)$
\item $(A, B) \to (A, B)$
\item $(A, B, b_1) \to (A, b_1)$
\item $(A, B, b_2) \to (A, b_2)$
\item $(A, B, b_3) \to (A, b_3)$
\item $(A, B, C) \to (A, C)$
\item $(A, B, C, c_1) \to (A, c_1)$
\item $(A, B, C, c_2) \to (A, c_2)$
\item $(A, B, C, c_3) \to (A, c_3)$
\item $(A, G, g_1) \to (A, g_1)$
\item $(A, G, g_2) \to (A, g_2)$
\end{itemize}

\needspace{20\baselineskip}
Częstą sytuacją jest to, że nazwy parametrów są identyczne,
zakładając, że $b_1 = g_1$ otrzymalibyśmy dla nich następujące
ścieżki o zapewnionej unikalności:

\begin{itemize}
\item $(A) \to (A)$
\item $(A, B) \to (A, B)$
\item $(A, B, b_1) \to (A, b_1)$
\item $(A, B, b_2) \to (A, b_2)$
\item $(A, B, b_3) \to (A, b_3)$
\item $(A, B, C) \to (A, C)$
\item $(A, B, C, c_1) \to (A, c_1)$
\item $(A, B, C, c_2) \to (A, c_2)$
\item $(A, B, C, c_3) \to (A, c_3)$
\item $(A, G, g_1) \to (A, G, g_1)$ --- 
      dodatkowa wartość $G$ w ścieżce
\item $(A, G, g_2) \to (A, g_2)$
\end{itemize}


\newpage
\footnotesize
\input{ex-storage/input}
\normalsize
\newpage




\subsection{Przypisywanie ról dla danych}

Po wyciągnięciu danych dla każdego rodzaju wartości można 
przypisać rolę. W toku dalszej analizy, role są używane do
wyciągania danych o różnych formatach z wielu tabel jednocześnie
do homogenicznych struktur. Struktury składają się z wyciągnietych
wartości oraz ich mapowania do oryginalnych danych.
To podejście zapewnia względną jednorodność oczekiwanych danych
dzięki zastosowaniu powtarzalnej metody ich wyciągania;
mapowanie daje wgląd do wewnętrznych źródeł danych; dane oryginalne
po profilowaniu nigdy nie zmieniają formy.

\sidenote{czy potrzebne wyjaśn.?}

\section{Dane patentowe}



\subsection{Role patentowe w aplikacjach patentowych}

\begin{figure}[H]
\centering
\begin{tikzpicture}
	\draw[draw=black, fill=lightgray, thin, solid] (-2.00,2.00) rectangle (-1.00,0.50);
	\node[black, anchor=south west] at (-3.06,2.25) {patent};
	\draw[draw=black, thin, solid] (-1.00,1.50) -- (1.00,4.00);
	\draw[draw=black, thin, solid] (-1.00,1.00) -- (1.00,-2.00);
	\node[black, anchor=south west] at (1.94,-2.25) {wynalazek};
	\node[black, anchor=south west] at (1.94,-0.25) {wynalazca};
	\draw[draw=black, thin, solid] (-1.00,1.50) -- (1.00,2.00);
	\draw[draw=black, thin, solid] (-1.00,1.50) -- (1.00,0.00);
	\node[black, anchor=south west] at (1.94,1.75) {aplikant};
	\node[black, anchor=south west] at (1.94,3.75) {właściciel};
	\draw[draw=black, thin, solid] (1.50,4.00) ellipse (0.50 and 0.50);
	\draw[draw=black, thin, solid] (1.50,2.00) ellipse (0.50 and 0.50);
	\draw[draw=black, thin, solid] (1.50,0.00) ellipse (0.50 and 0.50);
	\draw[draw=black, fill=black, thin, solid] (-1.00,1.50) circle (0.1);
	\draw[draw=black, fill=black, thin, solid] (-1.00,1.00) circle (0.1);
	\draw[draw=black, fill=black, thin, solid] (1.00,-1.50) rectangle (2.00,-2.50);
	\node[black, anchor=south west] at (-5.06,3.25) {biuro};
	\draw[draw=black, thin, solid] (-1.50,4.00) ellipse (0.50 and -0.50);
	\draw[draw=black, thin, solid] (-5.00,3.00) rectangle (-4.00,2.00);
	\node[black, anchor=south west] at (-1.2,4.5) {pełnomocnik};
	\draw[draw=black, thin, solid] (-1.50,2.00) -- (-1.50,3.50);
	\draw[draw=black, fill=black, thin, solid] (-1.50,2.00) circle (0.1);
	\draw[draw=black, fill=black, thin, solid] (-2.00,1.50) circle (0.1);
	\draw[draw=black, thin, solid] (-2.00,1.50) -- (-4.00,2.50);
	\draw[draw=black, thin, solid] (-4.50,0.00) ellipse (0.50 and -0.50);
	\node[black, anchor=south west] at (-5.06,0.75) {urzędnik};
	\node[black, anchor=south west] at (-3.06,-3.25) {raport};
	\draw[draw=black, fill=gray, thin, solid] (-4.00,-2.00) rectangle (-3.00,-3.50);
	\draw[draw=black, thin, solid] ([shift=(90:0.50 and -1.25)]-5.00,1.25) arc (90:270:0.50 and -1.25);
	\draw[draw=black, thin, solid] (-4.50,-0.50) -- (-4.00,-2.00);
	\draw[draw=black, fill=black, thin, solid] (-4.00,-2.00) circle (0.1);
	\draw[draw=black, thin, dotted] (-4.00,0.00) -- (-2.00,1.50);
	\draw[draw=black, thin, solid] (-3.00,-2.00) -- (-2.00,1.00);
	\draw[draw=black, fill=black, thin, solid] (-2.00,1.00) circle (0.1);
	\draw[draw=black, fill=black, thin, solid] (-3.00,-2.00) circle (0.1);
\end{tikzpicture}
\caption{Struktura powiązań patentu}
\label{fig:struktura-patentowa}
\end{figure}

\begin{defi}
Wynalazca --- osoba podająca się za autora bądź współautora nowej
wiedzy technicznej.
\end{defi}

\begin{defi}
Wynalazek --- nowa wiedza techniczna, która jest opatentowana.
\end{defi}

\begin{defi}
\label{defi:wynalazca}
Wynalazca --- osoba podająca się za autora bądź współautora nowej
wiedzy technicznej.
\end{defi}

\begin{defi}
\label{defi:aplikant}
Aplikant --- osoba składająca wniosek patentowy na podstawie autorstwa,
albo innych przesłanek do własności nad patentem; przykładowo patent
może być efektem pracy w organizacji w zatrudnieniu --- wtedy to
organizacja może być składać wniosek patentowy.
\end{defi}

\begin{defi}
Właściciel --- osoba posiadająca prawo do patentu; może je utrzymać
na przykład w wyniku sprzedaży.
\end{defi}

\begin{defi}
Pełnomocnik --- osoba wykonująca czynności urzędowe związane z
utrzymaniem patentu w mocy; może to być wyznaczona osoba niepowiązana z 
patentem, ale posiadająca uprawenienia wymagane przez urząd, albo
osoba fizyczna współuprawniona bądź z bliskiej rodziny.
\end{defi}

\begin{defi}
Biuro --- instytucja zajmująca się przyznawaniem patentów.
\end{defi}

\begin{defi}
Urzędnik --- tutaj: pracownik biura wykonujący raport o stanie
techniki dla danego patentu.
\end{defi}



\subsection{Dane przestrzenne}

Dane przestrzenne odnoszą się do miejsc z jakimi są powiązane
osoby albo organizacje związane z patentami. Pozostawia to więc 
różne możliwości analizy przestrzennej:

\begin{enumerate}
\item[$A$:] przypisanie każdego patentu do pojedynczej lokalizacji;
\item[$B$:] przypisanie patentu do wielu lokalizacji.
\end{enumerate}

W przypadku $A$ powstaje problem przypisania głównej lokalizacji.
Jest to kwestia $A_1$ priorytetowania organizacji ponad osoby, bądź
odwrotnie, oraz $A_2$ wyboru głównej osoby/organizacji.
Problem $A_1$ wiąże się z potencjalnymi różnicami w modelu zależnie
od wybranego podejścia. Problem $A_2$ może być niejednoznaczny
w rozwiązaniu z powodu zbyt małego zakresu informacji zawartych w danych.
Wymagałoby to dodatkowych danych z samego procesu powstawania wynalazku,
co jest poza zakresem tej pracy.

Dalsza analiza odnosi się wyłącznie do podejścia $B$.
Patent może mieć więc kilka lokalizacji, żadna nie jest określona jako główna.

\figside{../fig/endo/M.png}
{Mapa rozrzutu geolokalizacji osób pełniących role patentowe}
{ Problemem jest także to, że dane zawierają wyłącznie nazwy miast, 
  więc ich umieszczenie na mapie wiąże się z wadami: 
  nazwy nie są unikalne --- w takim przypadku używany jest algorytm minimalizacji 
  odległości, tak żeby wybrać optymalną kombinację. 
  Wynika z tego oczywiste obicążenie. 
  Po za tym nazwy mniejszych miejscowości mogą być duplikatami
  nazw miast. Pewne powiązanie jest w takim przypadku niemożliwe
  w wykonalny sposób na podstawie patentowych informacji.
  Co za tym idzie rozważane są wyłącznie lokacje uważane za polskie miasta. }


\newpage

Opis tego jak uzupełnione są braki danych znajduje się w kolejnej sekcji.
Obok znajduje się wykres ilustrujący ilość patentów z geolokalizacjiami,
zgodnie z tym jak zostały określone.

\fig{../fig/rgst/F-geoloc-eval-appl.png}
{ Stan uzupełnienia informacji o geolokalizacjach, w Polsce, 
  osób i organizacji  pełniących role patentowe
  w aplikacjach patentowych}

\fig{../fig/rgst/F-geoloc-eval-grant.png}
{ Stan uzupełnienia informacji o geolokalizacjach, w Polsce, 
  osób i organizacji  pełniących role patentowe
  w aplikacjach patentowych, które otrzymały ochronę}

\newpage









  \newpage\subsection
{Rejestr dat związanych z patentami}

Poszczególne czynności związane z ochroną patentów są rejestrowane.
Każde wydarzenie jest powiązane z konkretną datą kalendarzową.
Można wyróżnić kilka typów wydarzeń związanych z patentami:

  \begin{itemize}

\item
publiczne ujawnienie \foreign{ang}{exhibition};


\item
roszczenia z pierwszeństwa \foreign{ang}{priority claim} --- 
data rozszczenia sprzed rozpoczęcia procesu patentowania dla wybranego urzędu;


\item
regionalna deklaracja \foreign{ang}{regional filing};


\item
deklaracja \foreign{ang}{filing};


\item
aplikacja \foreign{ang}{application} --- data złożenia aplikacji;


\item
przyznanie ochrony \foreign{ang}{grant};


\item
decyzja urzędowa;


\item
publikacja.
\end{itemize}



  \newpage
\figpage{0.8}{../fig/patt/F-UPRP-event.png}
{Wydarzenia związane z patentami w kolejnych latach}

Wykres obrazuje kolejne lata i to jakie
działania podejmował urząd w stosunku do składanych patentów.
\todonote{analiza wykresu}



  \newpage
\figsides
{../fig/rgst/F-grant-delay.png}
{Okres po między złożeniem aplikacji, a przyznaniem ochrony w Polsce}
{../fig/rgst/F-grant-delay.png}
{ Okres po między złożeniem aplikacji, a przyznaniem ochrony w Polsce 
  w latach 2013-2022 }

\figside
{../fig/rgst/F-application-grant.png}
{ Lata składania aplikacji dla patentów, które otrzymały ochrone w Polsce
  w latach 2013-2022 }
{ W analizie czasowej opartej o patenty istotny jest fakt, że przyznawanie ochrony
  nie jest natychmiastowe. W Polsce średni czas oczekiwania na ochronę wynosi
  5.5 roku, chociaż najczęściej nie przekracza on okresu 5 lat (mediana).
  W związku z tym wszelkie wnioski dotyczące innowacji w Polsce są istotnie
  opóźnione w stosunku do powstawania wynalazków, przeciętnie o niespełna 5 lat. }


  \newpage
\figside
{../fig/subj/F-geoloc-eval.png}
{ Stan uzupełnienia informacji o geolokalizacjach, w Polsce, 
  osób i organizacji pełniących role patentowe
  w aplikacjach patentowych w zależności od roku}








  \newpage\subsection
{Klasyfikacje patentów}

Klasyfikacje patentowe to systemy, które pozwalają na przypisanie
patentów do odpowiednich dziedzin.



  \subsubsection
{Międzynarodowa Klasyfikacja Patentów}
\label{IPC}

W Polsce funkcjonuje klasyfikacja
\ac{MKP}, czyli \ac{IPC}. Zapis klasyfikacji w tym systemie to ciąg
cyfrowo-literowy składający się z 4 części:



  \begin{enumerate}

\item Dział - najwyższa hierarchia złożona z 8 kategorii

  \begin{itemize}
\item ma tytuł informacyjny
\item każdy tytuł działu ma swój symbol: A, B, C, D, E, F, G albo H

  \begin{itemize}
\item A – podstawowe potrzeby ludzkie
\item B – różne procesy przemysłowe; transport
\item C – chemia; metalurgia
\item D – włókiennictwo; papiernictwo
\item E – budownictwo; górnictwo
\item F – budowa maszyn; oświetlenie; ogrzewanie; uzbrojenie; technika minerska
\item G – fizyka
\item H – elektrotechnika
\end{itemize}

\item poddział - każdy dział może zawierać poddział, który nie jest oznaczany symbolem
\end{itemize}



\item Klasa - drugi poziom hierarchii

  \begin{itemize}
\item ma tytuł informacyjny
\item oznaczana przez liczbę 2-cyfrową
\item zakres klasy - skrótowa informacja o treści klasy
\end{itemize}



\item Podklasa - trzeci poziom hierarchii

  \begin{itemize}
\item ma tytuł informacyjny
\item oznaczana dużą literą
\item ma zakres i tytuł pomocniczy
\end{itemize}



\item Grupa - czwarty poziom hierarchii

  \begin{itemize}
\item 2 zestawy cyfr oddzielone ukośnikiem

  \begin{itemize}
\item zestaw pierwszy składa się od 1 do 3 cyfr i określa grupę główną
\item zestaw drugi składa się z 2 cyfr i określa grupę pomocniczą, grupa główna jest oznaczana 00
\end{itemize}

\item grupa ma tytuł informacyjny, podgrupa ma bardziej szczegółowe hasło
\end{itemize}

\end{enumerate}




\figside
{../fig/subj/F-geoloc-eval-clsf.png}
{ Stan uzupełnienia informacji o geolokalizacjach, w Polsce, 
  osób i organizacji pełniących role patentowe
  w aplikacjach patentowych w zależności od klasyfikacji patentu}




  \newpage\begin{acronym}

\acro
{UPRP}{Urząd Patentowy Rzeczypospolitej Polskiej}

\acro
{WUP}{Wiadomości Urzędu Patentowego}

\acro
{BUP}{Biuletyn Urzędu Patentowego}

\acro
{EPO}{European Patent Office}

\acro
{WIPO}{World Intellectual Property Organization}

\acro
{MKP}{Międzynarodowa Klasyfikacja Patentów}

\acro
{IPC}{International Patent Classification}

\acro
{IPCR}{International Patent Classification Revision}

\acro
{API}{Application Programming Interface}

\acro
{URI}{Uniform Resource Identifier}

\acro
{URL}{Uniform Resource Locator}

\acro
{OCR}{Optical Character Recognition}

\acro
{XML}{Extensible Markup Language}

\acro
{CPC}{Cooperative Patent Classification}

\acro
{USPTO}{United States Patent and Trademark Office}

\acro
{USPG}{United States Patent (and Trademark Office) Grants}

\acro
{USPA}{United States Patent (and Trademark Office) Applications}

\acro
{PGC}{Patents.Google.com}

\acro
{PLO}{Patenty Lens.org}

\end{acronym}

    \newpage\section
  {Obszary peryferyjne}

Obszary peryferyjne to miejsca odległe od centrów, w tym przypadku
głównych źródeł aplikacji patentowych. Wyznaczamy je poprzez obliczenie
średniego dystansu do innych osób pełniących role patentowe.
Oprócz dystansu do każdego innego punktu, warto jest ograniczyć
tę statystykę do pewnego obszaru. Polska nie jest jednorodna przestrzennie 
pod tym względem, stąd peryferyjność nie może być rozumiana tylko 
na krajowym poziomie, ale również lokalnym.

\begin{uwaga}
Średnia odległość po między geolokalizacjami patentowymi jest ważona ---
ilość patentów pochodzących z danej geolokalizacji jest wagą tego punktu.
\end{uwaga}

  \fig
{../fig/endo/M-meandist.png}
{ Mapa średniej odległości do innych osób pełniących role patentowe }

  \fig
{../fig/endo/M-meandist100.png}
{ Mapa średniej odległości do innych osób pełniących role patentowe do 100 km }

  \fig
{../fig/endo/M-meandist50.png}
{ Mapa średniej odległości do innych osób pełniących role patentowe do 50 km }


[TODO wnioski]




    \newpage\section
  {Połączenia opracowane npdst. raportów o stanie techniki}

[TODO wprowadzenie]

  \figside
{../fig/grph/F-rprt-meandist.png}
{ Histogram pionowy odległości między punktami, które są 
  połączone opracowanymi na podstawie raportów o stanie techniki }

Na następnej stronie znajdują się mapy z zaznaczonymi połączeniami.

\newpage

\figpage{0.9}
{../fig/grph/M-rprt-dist.png}
{ Połączenia opracowane npdst. raportów o stanie techniki }

\newpage

  \chartside
{../fig/grph/F-rprt-comp.png}
{ Statystyki grafu połączeń opracowanych na podstawie raportów o stanie techniki }


\subsection{[TODO statystyki grafu]}

\subsection{[TODO wnioski]}




    \newpage\subsection
  {Dodatkowe statystyki punktowe}

  \subsubsection
{Udział klasyfikacji \ac{IPC} w klastrach}\label{udział-klasyfikacji}
Udział klasyfikacji odnosi się do tego jak wiele osób dostawało 
ochronę patentową w danej sekcji ze wskazanego punktu, 
bądź współpracowało przy takim dokumencie z innymi osobami.

[TODO szczegółowy opis]

  \subsubsection
{[TODO Procentowy przyrost liczby patentów w danej sekcji 
  dla $k$-przeszłego roku z $n=10$ lat w punkcie]}\label{przyrost-patentów}

  \subsubsection
{[TODO Liczba krajowych i niekrajowych patentów wspomnianych
  w raportach o stanach techniki --- udział procentowy w danym punkcie]}\label{rprt-krajowy}

  \subsubsection
{[TODO Dystans do patentów wspomnianych w raportach 
  o stanie techniki]}\label{rprt-krajowy-dystans}




    \newpage\section
  {[TODO Autokorelacja przestrzenna]}




    \newpage\section
  {Klastry przestrzenne z uwzględnieniem klasyfikacji \ac{IPC}}

  \figside
{../fig/clst/M-kmeans.png}
{Rozrzut przestrzenny puntków w klastrach}
{
Metodą $k$-średnich dla $k=$ wyznaczamy klastry przestrzenne. 
Cechami branymi pod uwagę są współrzędne przestrzenne punktów oraz
udział klasyfikacji \ac{IPC} w klastrach (\cref{udział-klasyfikacji}). 
[TODO implementacja \cref{przyrost-patentów},\cref{rprt-krajowy},
\cref{rprt-krajowy-dystans}]
}

\tblside
{../fig/clst/T-kmeans-clsf.png}
{Udział klasyfikacji \ac{IPC} w klastrach}

\tblside
{../fig/clst/T-kmeans-meandist.png}
{Średnie odległości między punktami w klastrach}




    \newpage\section{Dynamika czasowa}

  \figside
{../fig/endo/F-pat-n-woj.png}
{ Liczba patentów, które otrzymały ochronę w poszczególnych województwach 
  w kolejnych latach }

  \figsides
{../fig/endo/M-woj-13.png}
{ Mapa rozrzutu geolokalizacji osób pełniących role patentowe 
  przy patentach, które otrzymały ochronę w 2013 roku}
{../fig/endo/M-woj-dt-14.png}
{ Mapa rozrzutu geolokalizacji osób pełniących role patentowe, 
  przy patentach, które otrzymały ochronę w 2014 roku}

  \newpage\figsides
{../fig/endo/M-woj-dt-15.png}{Mapa zmian w 2015}
{../fig/endo/M-woj-dt-16.png}{Mapa zmian w 2016}

  \figsidesTri
{../fig/endo/M-woj-dt-17.png}{2017}
{../fig/endo/M-woj-dt-18.png}{2018}
{../fig/endo/M-woj-dt-19.png}{2019}

  \figsidesTri
{../fig/endo/M-woj-dt-20.png}{2020}
{../fig/endo/M-woj-dt-21.png}{2021}
{../fig/endo/M-woj-dt-22.png}{2022}



  \newpage
[TODO obserwacje]



  \newpage\subsection
{Analiza trendu}



  \fig
{../fig/endo/F-Q.png}{Ilość patentów w zależności od kwartału przyznania ochrony}

  \fig
{../fig/endo/F-mo.png}{Ilość patentów w zależności od miesiąca przyznania ochrony}



  \subsubsection
{[TODO Stacjonarność]}




  \newpage\subsubsection
{Sezonowość}

  \figsides
{../fig/endo/F-pat-n-woj-Q.png}
{ Liczba patentów, które otrzymały ochronę w poszczególnych województwach 
  --- sumy kwartalne z lat 2013-2022 }
{../fig/endo/F-pat-n-woj-mo.png}
{ Liczba patentów, które otrzymały ochronę w poszczególnych województwach 
  --- sumy miesięczne z lat 2013-2022 }

[TODO testy]

  \subsubsection
{TODO Autokorelacja}

\section{Identyfikacja osób}

Głownym problemen danych jest niejednoznaczność w kontekście identyfikacji osób.
W danych patentowych, osoby rozróżnia się za pomocą imienia, nazwiska
oraz nazwy miejscowości. Jak wiadomo wiele osób może mieć te same imię i nazwisko,
także w jednym miejscu. Jest to duże ograniczenie wynikające z samego zbioru danych.
Należy także wspomnieć o drobnych niespójnościach danych w zapisie imion i nazwisk
(\cref{def:drobne-niespójności}) --- występowanie diaktryk i akcentów w zapisie
nie jest gwarantowane, a jednocześnie nie jest wykluczone.

Kolejną niejednoznacznością jest podobne zjawisko dla nazw miejscowości.
W Polsce jest wiele miejscowości o identycznych nazwach, a rejestry nie oferują
nic po za samą nazwą. Tutaj także występuje problem z diaktrykami i akcentami.

Ponadto występują też 2 inne problemy. Pierwszym jest 
niespójność fragmentacji danych (\cref{def:niespójność-fragmentacji}).
W przypadku tabeli z danymi osobowymi wynalazców są do dyspozycji ich
imiona i nazwiska. W przypadku pozostałych osób związanych z patentem
są to najczęściej ciągi imion i nazwisk. Nie jest jednak gwarantowane,
że dotyczą one osób fizycznych. Drugim problemem jest niespójność 
typów danych (\cref{def:niespójność-typów}). Część danych oznaczonych
jako imiona dotyczy nazw firm lub instytucji. Oznaczenie tego faktu
istnieje tylko w niektórych przypadkach, dużo częściej jest to pominięte.



\subsection{Niespójność typów}

Rozwiązaniem problemu pomieszania nazw organizacji o raz imion osób
jest wykorzystanie danych o odpowiednim typowaniu i fragmentacji.
Wyróżniamy w nich pojedyncze słowa albo ciągi i przyjmujemy, 
że są charakterystyczne dla danego typu. W przypadku nazw organizacji
są to ciągi, a w przypadku imion i nazwisk --- pojedyncze słowa.
Dodatkowo testujemy podpisy na zawartość ustalonego zbioru słów
kluczowych, które są charakterystyczne dla nazw organizacji.
Po utworzeniu zbiorów słów i ciągów kluczowych imiona i nazwy
są klasyfikowane na podstawie ich zawartości. W ten sposób
możemy zidentyfikować, czy dany wpis dotyczy osoby fizycznej
czy organizacji.

\begin{uwaga}
Duża część nazw i imion nie ulega klasyfikacji w wyniku powyższego
algorytmu. Dla uproszczenia zakładamy, że dotyczą one wtedy imion
osób fizycznych.
\end{uwaga}



\subsection{Wyszukiwanie podobieństw}

\begin{defi}
Wpis osobowy $w_i$ --- pojedynczy obiekt przypisany danemu patentowi. Zawiera
informacje o osobie powiązanej z patentem. Jeden paten może zawierać
wiele wpisów osobowych. Każdy składa się z: imienia i nazwiska albo
ciągu imienniczego (\cref{def:ciąg-imienniczy}); nazwy miejscowości zameldowania.
\end{defi}

\begin{defi}\label{def:ciąg-imienniczy}
Ciąg imienniczy $N_k$ --- ciąg imion oraz nazwisk przypisany danej osobie.
Nazwy podwójne rozdzielone znakami interpunkcyjnymi są traktowane jako
oddzielne imiona.
\end{defi}

Każdy element ciągu imienniczego podlega normalizacji. Wszystkie jego 
znaki są traktowane jako wielkie litery, a znaki diaktryczne oraz akcenty 
są zastępowane ich odpowiednikami tych wyróżnień piśmienniczych.
Wynika to z faktu, że ich obecność nie jest pewna w danych.

\begin{uwaga}
To czy dany element $n,\ n\in N_k$ jest imieniem, czy nazwiskiem nie zawsze
jest jednoznaczne.
\end{uwaga}

\begin{uwaga}
Dane w zbiorach uwzględniają różną szczegółowość w zapisie imion i nazwisk.
Niektóre zawierają drugie imię, niektóre wyłącznie literę drugiego imienia.
Przypadków jest wiele. Poniższe podejście pomija tę ambiwalencję.
\todonote{można tego użyć jako dodatk. deter. podob.}
\end{uwaga}

\begin{defi}
Główna para imiennicza $\hat N_k$ --- zbiór 2-elementowy pierwszego 
i ostatniego słowa ciągu imienniczego
\end{defi}

\begin{defi}\label{defi:zgodność-nazw}
Zgodność nazewnicza 2 wpisów $w_i,w_j$ występuje pod warunkiem, że
elementy suma zbiorów ich głównych par imienniczych jest im równa:
$$
\varphi(w_i, w_j) = \begin{cases}
  1 & \text{jeśli } \hat N_i = (\hat N_i \cap \hat N_j) = \hat N_j\\
  0 & \text{w przeciwnym przypadku}
\end{cases}
$$
\end{defi}

\begin{defi}\label{defi:podobieństwo-af-nazw}
Podobieństwo afiliacyjno-nazewnicze $\tilde \varphi$ --- dotyczy wpisów,
które są związane z patentami badanej pary wpisów $w_i, w_j$.
Zbiór $N_i$ jest zbiorem zbiorów głównych par imienniczych wpisów
dotyczących patentu zawierającego wpis $w_i$. Analogicznie jest
dla $N_j$:

$$
\tilde \varphi(p_i, p_j) = | \tilde N_i \cap \tilde N_j | \ge 0,\quad
\tilde N_i = \{ \hat N_k \mid w_k \in W_i \land k \ne i \}
$$
\end{defi}

\begin{uwaga}
Zgodność $\varphi$ (\cref{defi:zgodność-nazw}) oraz podobieństwo $\tilde \varphi$ 
(\cref{defi:podobieństwo-af-nazw}) pomija rozróżnienie na imiona i nazwiska.
\todonote{dodać sposób korzys. z rozróżn. na imię, nazwi.}
\end{uwaga}

\begin{defi}\label{defi:zgodność-geolokalizacyjna}
Zgodność geolokalizacyjna:
$$
\gamma(p_i, p_j) = \begin{cases}
  1 & \text{jeśli } G_i = G_j\\
  0 & \text{w przeciwnym przypadku}
\end{cases}
$$
gdzie $G_i$ to zbiór geolokalizacji przypisanej danej osobie.
\end{defi}

\begin{defi}
Zbiór afiliacyjno-nazewniczy $k$-osoby $\tilde N_k$ - zbiór imion, 
nazwisk oraz słów, które mogą być imieniem lub nazwą; zawiera
wyżej wymienione elementy, którymi identyfiują się osoby będące
współautorami patentów razem z $k$-osobą.
$$\tilde N_k \subset N_0$$
\end{defi}

\begin{uwaga}
Wpisy zawierają nazwy miejscowości zameldowania, jednak wyszukwianie za ich
pomocą odbywa się po geolokalizacji patentowej\todonote{potrzeb. odnies.}.
W związku z tym, mimo identycznej nazwy miejscowości może nie dojść do
zgodności geolokalizacyjnej.
\end{uwaga}

\begin{defi}\label{defi:podobieństwo-af-geo}
Podobieństwo afiliacyjno-geolokalizacyjne $\tilde\gamma$ --- ilość identycznych geolokalizacji
dla dwóch osób: $p_i$ oraz $p_j$:

$$
\tilde\gamma(p_i, p_j) = | \tilde G_i \cap \tilde G_j | \ge 0,
$$

gdzie $\tilde G_i$ to zbiór geolokalizacji osób w relacji współautorstwa z $p_i$.
\end{defi}

Kolejnym determinantem jest klasyfikacja. \Cref{wniosek:klasyfikacje-deter-1}
pokazuje, że klasyfikacje nie są dobrym determinantem jeśli opierać by się wyłącznie 
na nich. Warto jednak zauważyć, że klasyfikacje mogą być dobrym uzupełnieniem
dla pozostałych determinantów.

\begin{uwaga}
Aktualnie jedyną klasyfikacją wziętą pod uwagę jest \ac{IPC}, które
nie występuje dla wszystkich obserwacji.\todonote{usunąć to po uwzgl.
innych klasyf.}
\end{uwaga}

\begin{defi}\label{defi:zgodność-clsf}
Zgodność klasyfikacyjna $\eta$ --- ilość identycznych sekcji klasyfikacji,
w których znajdują się aplikacje patentowe dwóch osób: $p_i$ oraz $p_j$:
$$\eta(p_i, p_j) = | C_i \cap C_j | \ge 0,$$
gdzie $C_i$ to zbiór sekcji klasyfikacji dla osoby $p_i$.
\end{defi}

\subsubsection{Przebieg wyszukiwania}

\begin{uwaga}
Przyjmujemy uproszczenie: zakładamy, że dana osoba w różnych patentach 
jest podpisana w jednolity sposób.
\end{uwaga}

Pierwszym etapem wyszukiwania jest zawężenie zakresu wyłącznie do relacji
spełniających warunek $\varphi(p_i, p_j) = 1$.
W efekcie otrzymujemy zbiór $W_0$:

$$W_0 = \{ (p_i, p_j)\mid \varphi(p_i, p_j) = 1 \}$$

Kolejnym etapem jest wyznaczenie wartości zgodności klasyfikacji oraz 
podobieństw afiliacyjnych.

$$W_1 = \{ ( \eta(p_i, p_j), \tilde \gamma(p_i, p_j), \tilde \varphi(p_i, p_j) ) \mid (p_i, p_j) \in W_0 \}$$

Zbiór $W_1$ dzielimy na dwa podzbiory zgodnie z wartościami zgodności lokalizacyjnej:

$$
W_2 = \{ ( w \mid w \in W_1, \gamma(p_i, p_j) = 0 \}\qquad 
W_3 = \{ ( w \mid w \in W_1, \gamma(p_i, p_j) = 1 \}
$$

\begin{uwaga}
Zgodność lokalizacyjną można zastąpić miarą odległości geograficznej,
jednak na potrzeby uproszczenia jest to wartość binarna.
\end{uwaga}

Oba zbiory rozważamy jako oddzielne przypadki ze względu na ich
gruntownie różną naturę. Przyjmując pewne stałe jako wartości
graniczne dla zgodności klasyfikacji oraz podobieństw afiliacyjnych
możemy podjąć decyzję o zgodności dwóch wpisów pod względem
opisywania jednej osoby. Na podstawie grafu tych zgodności
identyfikujemy spójne składowe. Każda taka część grafu to
zbiór wpisów, które opisują tę samą osobę.



\subsection{Uzupełnianie braków geolokalizacji za pomocą innych danych}

Brak danych dotyczących położenia osób związanych z danym patentem jest
istotnym problemem w analizie dyfuzji przestrzennej. Pominięcie obserwacji
z powodu braku danych może prowadzić do błędnych wniosków. 
Determinacja położenia osób za pomocą innych danych patentowych jest
więc kluczowa.

W danych można wyróżnić 3 przypadki dostępności geolokalizacji.

\begin{przyp}\label{przyp:brak-geo-0}
Geolokalizacja jest dostępna dla każdej osoby związanej z patentem.
\end{przyp}

\begin{przyp}\label{przyp:brak-geo-n}
Geolokalizacja jest dostępna dla części osób zwiazanych z patentem.
\end{przyp}

\begin{przyp}\label{przyp:brak-geo-N}
Geolokalizacja nie jest dostępna dla żadnej osoby związanej z patentem.
\end{przyp}

W przypadku \ref{przyp:brak-geo-0} nie ma potrzeby uzupełniania danych.
W przypadkach \ref{przyp:brak-geo-n} i \ref{przyp:brak-geo-N} 
rozważamy sytuację po identyfiakcji za pomocą podobieństw. 
Jeśli dany wpis osoby jest przypisane osobie, 
która w innych wpisach ma geolokalizacje, to jest ona uzupełniana
najczęściej powtarzającą sie geolokalizacją.

Efektem jest zmiana części przypadków z \ref{przyp:brak-geo-N} na
\ref{przyp:brak-geo-n}, potencjalnie także na \ref{przyp:brak-geo-0}.
W takiej sytuacji należy powtórzyć czynność wyszukiwania podobieństw ---
dane zostały uzupełnione o geolokalizację, więc potencjalnie można
oczekiwać znalezienia nowych relacji identyczności.
Te 2 kroki należy powtarzać dopóki dają efekt. Ostatecznie
przypadki, które pozostały przypadkami \ref{przyp:brak-geo-N}
nie mogą zostać uzupełnione wiarygodnymi metodami.
W przypadkach \ref{przyp:brak-geo-n} można zastosować wyróżnianie
najczęściej powtarzanego miejsca związania pośród innych osób i przyjąć
je za geolokalizację osób bez niej.

\chapter{Raport o stanie techniki}\label{ch:data}

\section{Raporty o stanie techniki jako informacja o dyfuzji}

Na stronie \ac{UPRP} wyróżniono etapy w procesie patentowania.
Etapem po spełnieniu formalności jest 
\textit{sprawozdanie ze stanu techniki}\cite{UPRP-pat-prd}.
Słownik urzędu wskazuje, że \textit{stan techniki} to wszelka
wiedza dostępna do wskazanej daty powszechnie, albo taka,
która nie jest publiczna, ale została już ogłoszona 
w określony sposób\cite{UPRP-dict}.

Sprawozdanie jest realizowane przez urzędników i składa się z
klasyfikacji zgłoszenia, informacji o innych klasyfikacjach,
w których prowadzono poszukiwania, wykaz baz komputerowych,
użytych w trakcie procesu oraz tabelę zawierającą odniesienia
do innych prac. Wśród odniesień można wyróżnić odniesienia do
patentów, artykułów naukowych, książek, stron internetowych,
a także do innych zgłoszeń patentowych.
Przykłady takich odniesień są przedstawione na następnej stronie.

\newpage
\begin{figure}[H]\centering
\label{fig:raport-biblio-ex}
\includegraphics[width=0.8\textwidth]{ex-img/bilblio-ex-1.jpg}
\caption{Przykład odniesienia do literatury technicznej 
         w raporcie o stanie techniki.}
\end{figure}
\begin{figure}[H]\centering
\label{fig:raport-pat-ex}
\includegraphics[width=0.8\textwidth]{ex-img/pat-ex-1.jpg}
\caption{Przykład odniesienia do innych patentów 
         w raporcie o stanie techniki.}
\end{figure}
\begin{figure}[H]\centering
\label{fig:raport-url-ex}
\includegraphics[width=0.8\textwidth]{ex-img/url-ex-1.jpg}
\caption{Przykład odniesienia do strony internetowej 
         w raporcie o stanie techniki.}
\end{figure}
\newpage

Z danych zebranych z \ac{API} można wyodrębnić tabelę 
\textit{other-documents}, która zawiera listę adresów internetowych
z plikami związanymi z danym zgłoszeniem. Tabela składa się
z kodów \ac{URI} oraz kodów rozróżniających typ dokumentu.
Kody o typie \textit{RAPORT} albo \textit{RAPORT1} są kodami 
\ac{URI} do z adresami \ac{URL} do plików zawierających raporty 
o stanie techniki. Są to pliki w formacie \ac{PDF}.

Poniżej przedstawiono przykład tego jaką strukturę mogą tworzyć 
\cref{fig:raport-ex}. Są to raporty dla patentów $p_1, p_2, p_3$. 
Zawierają odniesienia do patentów, które istnieją w zbiorze danych, 
tj. $p_1, p_2, p_3$. Oprócz tego mają odniesienia do patentów 
spoza domeny $\hat p_4, \hat p_5$ oraz publikacji naukowych 
$\hat l_1, \hat l_2$, które nie są uwzględnione w poniższej analizie.

\begin{figure}[H]\centering
\begin{tikzpicture}
	\draw[draw=black, thin, solid] (-5.00,3.00) rectangle (-3.00,0.00);
	\draw[draw=black, thin, solid] (-1.50,3.50) rectangle (1.50,-0.50);
	\draw[draw=black, thin, solid] (2.50,3.50) rectangle (5.50,-0.50);
	\draw[draw=black, thin, solid] (-1.00,3.00) rectangle (1.00,0.00);
	\draw[draw=black, thin, solid] (3.00,3.00) rectangle (5.00,0.00);
	\draw[draw=black, thin, solid] (-5.50,3.50) rectangle (-2.50,-0.50);
	\node[black, anchor=south west] at (-5.56,3.75) {$p_1$};
	\node[black, anchor=south west] at (-1.56,3.75) {$p_2$};
	\node[black, anchor=south west] at (2.44,3.75) {$p_3$};
	\node[black, anchor=south west] at (-5.06,1.25) {$p_3$};
	\node[black, anchor=south west] at (-5.06,2.25) {$p_2$};
	\node[black, anchor=south west] at (-1.06,2.25) {$p_3$};
	\node[black, anchor=south west] at (-5.06,0.25) {$\hat l_1$};
	\node[black, anchor=south west] at (-1.06,0.25) {$\hat l_2$};
	\node[black, anchor=south west] at (2.94,1.25) {$\hat l_3$};
	\node[black, anchor=south west] at (-1.06,1.25) {$\hat p_4$};
	\node[black, anchor=south west] at (2.94,2.25) {$\hat p_5$};
\end{tikzpicture}
\caption{Przykład raportów o stanie techniki}
\label{fig:raport-ex}
\end{figure}

\subsubsection{Tworzenie grafu za pomocą danych z tabel
               raportów o stanie techniki}
\label{sec:graf-raporty}

Dane z raportów o stanie techniki tworzą graf skierowany
patentów $G$. Krawędź w takim grafie istnieje jeśli patent
$p_1$ zawiera w swoim raporcie wzmiankę o patencie~$p_2$.

Wracając do przykładu \cref{fig:raport-ex}, zastosowanie algorytmu 
tworzy graf o krawędziach $E = \{ (p_2, p_1), (p_3, p_1), (p_3, p_2) \}$ i
wierzchołkach $V = { p_1, p_2, p_3 }$.

\begin{figure}\centering
\begin{tikzpicture}
	\draw[draw=black, thin, solid] (-1.50,1.50) ellipse (0.50 and -0.50);
	\node[black, anchor=south west] at (-2.06,1.25) {$p_1$};
	\draw[draw=black, thin, solid] (1.50,2.50) ellipse (0.50 and -0.50);
	\draw[draw=black, thin, solid] (0.50,-0.50) ellipse (0.50 and -0.50);
	\node[black, anchor=south west] at (0.94,2.25) {$p_2$};
	\node[black, anchor=south west] at (-0.06,-0.75) {$p_3$};
	\draw[draw=black, -latex, thin, solid] (-0.14,0.04) -- (-0.92,0.96);
	\draw[draw=black, -latex, thin, solid] (0.78,0.32) -- (1.14,1.67);
	\draw[draw=black, -latex, thin, solid] (0.49,2.33) -- (-0.62,1.90);
\end{tikzpicture}
\caption{Graf dla przykładowego zestawu raportów \cref{fig:raport-ex}}
\label{fig:raport-ex-G}
\end{figure}

Zakładając, że wpisy ekspertów zawierają wyłącznie kody publikacji 
patentów, graf mógłby być stworzony przez bezpośrednie powiązanie
ich z danymi. Istotnym problemem w takiej sytuacji jest jakość \ac{OCR},
która jest dobra, ale nie pewna.\todonote{jakaś miara jakości}
Wpisy ekspertów nie są jednak jednorodne w taki sposób. Problem
rozpoznania znaków nie jest jedyny, bo pojawiają się kolejne:

\begin{itemize}
\item wątpliwa jakość \ac{OCR}
\item wpisy to nie tylko kody patentowe
\item wspominane kody patentowe nie zawsze są publikacjami,
      mogą to być np. kody złożenia aplikacji\todonote{przykład}
\end{itemize}

\begin{figure}[H]\centering
\includegraphics[width=0.8\textwidth]{ex-img/pat-ex-P.jpg}
\caption{Przykład odniesienia poprzez numer publikacji.}
\end{figure}

\begin{figure}[H]\centering
\includegraphics[width=0.8\textwidth]{ex-img/pat-ex-A.jpg}
\caption{Przykład odniesienia poprzez numer aplikacji.}
\end{figure}

\begin{figure}[H]\centering
\includegraphics[width=0.8\textwidth]{ex-img/pat-ex-A-P.jpg}
\caption{Przykład odniesienia poprzez numer aplikacji i publikacji
         jako jeden numer.}
\end{figure}

Sposobem na minimalizację zjawiska błędnych powiązań jest zastosowanie
algorytmu wyszukiwania (\cref{sec:wyszukiwanie}).

Ninejsze wyszukiwanie polega na wskazaniu sumy zbiorów
słów: numerów patentowych, słów języka naturalnego oraz dat. 
Wskazanie tej sumy zachodzi dla każdej pary wszystkich słów zapytań
ze wszystkimi słowami ze zbioru danych. Dodatkowo zachodzi łączenie
częściowe, które dopasowuje n-gramy poszczególnych słów. Od pewnego
minimalnego dopasowania zostają one uwzględnione. Całość wymaga
pewnych kroków optymalizacyjnych. Głównym jest zasada, że słowa
są dopasowywane tylko pod warunkiem, że zaszło dopasowanie numeru
patentu. \todonote{umieścić przykład}

W trakcie wyszukiwania tworzona jest jego punktacja, aby odróżnić
wartościowe wyniki. Oprócz punktacji jest też ustalanie poziomu
wyszukiwania na podstawie tego jakie rzeczywiste dane są łącznikami.
Wybierane są wyłącznie pojedyncze, najlepsze wyniki.\todonote{szczgł. opis}



\fig{../fig/F-results.rap.png}{Wyniki wyszukiwania cytowań z raportów o stanie techniki}

Raporty \ac{PDF} nie posiadają adnotacji tekstowych. 
Znaczy to tyle, że dane są zawarte w sposób czytelny
jedynie dla człowieka i nie są dostępne dla urządzeń w sposób
ustruktoryzowany inny niż ciąg binarny pikseli.
Rozwiązaniem jest proces \ac{OCR}, który obrazy zawierające tekst
przekształca na kod binarny, które można przetwarzać na komputerze
jako ciągi znaków odpowiadające prawdziwemu tekstowi. Pierwszą
czynnością w tym procesie jest zastosowanie pakietu \textit{paddle}.
Zastosowanie modułu pozwala na pozyskanie linijek tekstu z przypisaniem
do ich pozycji. Wynika z tego problem taki, że nie brak jest informacji
o tym gdzie zaczyna i kończy się tekst dotyczący wskazanej obserwacji.
W związku z tym nie sposób jest przypisać tekstu do odpowiednich
wpisów. Dodatkowo dochodzą problemy wynikające z błędów w procesie
skanowania samych dokumentów - zniekształcenia, zaciemnienia, czy
rotacje kartek sprawiają, że proces \ac{OCR} nie jest idealny.
Dodatkowo samo formatowanie nierzadko jest wadliwe co wynika
z wprowadzania danych jeszcze na etapie tworzenia dokumentów.



\subsubsection{Zastosowanie dużego modelu językowego}

Do skutecznego pozyskania danych z dokumentów kluczowe było zastosowanie
dużych modeli językowych z multimodalnymi wejściami. Stan tej technologii
na dzień procesu wyciągania danych był na tyle zaawansowany, że
aspekty techniczne ograniczają się do zastosowania zewnętrznego \ac{API}
dla modelu \textit{openai} \textit{GPT4o}. Model ten w wystarczający
sposób był w stanie przetworzyć obrazy zawierające tekst na ustruktoryzowany
zbiór wpisów tekstowych.

Mimo, że model \textit{paddle} nie dawał wyników pozwalających na
poprawną dalszą analizę to pozwolił na ograniczenie kosztów. Znalezienie
słów kluczowych nagłówków i stopek tabeli z informacjami było wystarczające
aby przyciąć zdjęcia do obszarów zainteresowania.


\bibliographystyle{plain}\bibliography{cit}

\listoffigures

\listoftables

\chapter*{Załączniki}
\begin{enumerate}
\item Płyta CD z niniejszą pracą w wersji elektronicznej.
\end{enumerate}

\chapter*{Streszczenie (Summary)}

\bigskip

\bigskip

\begin{center}
  \textbf{\tytul}
\end{center}

\bigskip

\begin{center}
  \textbf{\textit{\tytulangielski}}
\end{center}

\selecthyphenation{english}
{\it

}

\end{document}