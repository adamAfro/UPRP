    \newpage\section
  {Obszary peryferyjne}

  \newpage\charttripled
{../fig/endo/M-meandist.png}
{ Mapa i histogramy średniej odległości do innych osób pełniących role patentowe }
{../fig/endo/M-meandist100.png}
{ Mapa i histogramy średniej odległości do innych osób pełniących role patentowe do 100 km }
{../fig/endo/M-meandist50.png}
{ Mapa i histogramy średniej odległości do innych osób pełniących role patentowe do 50 km }




    \newpage\section
  {Połączenia opracowane npdst. raportów o stanie techniki}

\flow{graph}

  \chartside
{../fig/grph/F-nodes.png}
{ Statystyki wierzchołków grafu raportów stanu techniki }
{ Wykres \cref{../fig/grph/F-nodes.png} przedstawia statystyki
  dotyczące wierzchołków grafu opracowanego na podstawie raportów
  o stanie techniki. Należy zaznaczyć, że w histogramach zastosowano
  skalę logarytmiczną. W dwóch pierwszych wykresach wartości są skupione
  przy zerze. Obserwujemy więc, skośność w lewo, a fakt zastosowania
  skali logarytmicznej wskazuje, że jest ona bardzo duża.
  W przypadku rozkładu stopnii wierzchołków (wykres pierwszy od góry)
  można zauważyć, że dużo częściej wierzchołki mają niski stopień.
  Można to interpretować tak, że wynalazki osób zwiazanych z patentami w Polsce
  nie są wskazywane w raportach o stanie techniki często, a więc dyfuzja wiedzy
  jest ograniczona. }

Entropia klasyfikacji odnosi się do entropii sekcji klasyfikacji \ac{IPC} w jakich
publikowała dana osoba (pracowała przy patencie o danej klasyfikacji).
Podobnie jak w przypadku stopnia wierzchołka, wartości są skupione przy zerze.
W zwiazku z tym należy przypuszczać, że wynalazki tworzone przez osoby
pracujące przy patentach w Polsce nie są zróżnicowane pod względem klasyfikacji \ac{IPC}.

\TODO{interpretacja bliskości w grafie i popr. wykres}

\newpage

  \chartside
{../fig/grph/F-edges.png}
{ Statystyki krawędzi grafu raportów stanu techniki }
{ Wykres \cref{../fig/grph/F-edges.png} przedstawia statystyki krawędzi grafu.
  Można zaobserwować, że w przypadku prawie połowy krawędzi dystans jest zerowy.
  Oznacza to, że osoby pracujące przy patentach bardzo często inspirowały się
  pracami osób w bezpośrednim sąsiedztwie. Na historgamie widać również,
  duże skupienie wartości w okolicach 20 kilometrów. Kolejną modę można zauważyć
  przy wartościach około 270 kilometrów.
  Okres oczekiwania \TODO{krót. opis} ...}

Indeks Jaccarda
w dużej większości jest równy $1$, w mniejszej liczbie $0.5$.
Zgodnie z definicją \TODO{dodać def} interpretujemy to jako fakt, że
podobieństwo patentowe, identyfikowane w raportach o stanach techniki,
występuje zazwyczaj, gdy patenty są klasyfkowane identycznie, albo
różnica klasyfikacji jest niewielka.

Na następnej stronie znajdują się mapy z zaznaczonymi połączeniami.

\newpage

  \chart
{../fig/grph/M-rprt-dist.png}
{ Połączenia opracowane npdst. raportów o stanie techniki }\newpage

\newpage

  \chartside
{../fig/grph/F-rprt-comp.png}
{ Statystyki składowych wychodzących z wierzchołków patentowych
  grafu raportów stanu techniki }{
\Cref{../fig/grph/F-rprt-comp.png} dotyczy składowych grafu, wśród których
jest wierzchołek będący jednym z patentów, który dostał ochronę w badanym okresie.
Graf jest skierowany --- krawędzie wychodzą z patentów cytowanych.
Składowe, których raporty o stanie techniki są puste albo nie dotyczą 
polskich patentów tworzą składowe izolowane o 1 wierzchołku.
}



    \newpage\section
  {Globalna autokorelacja przestrzenna}

Statystyka Morana (\cref{Moran-I}) dla globalnej autokorelacji przestrzennej
jest obliczana zarówno dla ilości wszystkich patentów z danego powiatu, 
jak i poszczególnych klasyfikacji \ac{IPC}(\cref{IPC}).
P-wartości określamy permutacyjnie losując przyporządkowanie wartości
do obiektów przestrzennych\cite{pysal-07}.

\chartside{../fig/corr/T-Moran.png}
{ Statystyka Morana dla globalnej 
  autokorelacji przestrzennej }
{
  Zakładamy hipotezę zerową, że nie ma autokorelacji przestrzennej, oraz
alternatywnie, że taka autokorelacja istnieje:
\begin{itemize}
\item[$H_0$] Nie ma autokorelacji przestrzennej.
\item[$H_1$] Jest autokorelacja przestrzenna.
\end{itemize}}

Zgodnie z wynikami testu, nie ma podstaw do odrzucenia hipotezy zerowej
o losowości w położeniu punktów patentowych w Polsce dla ogólnej ilości patentów.
Podobnie należy stwierdzić w przypadku wyszczególnienia patentów z sekcji \ac{IPC},
z wyjątkiem \textbf{D} oraz \textbf{E}: z przyjętym poziomem istotności $\alpha=0.05$,
dla tych dwóch sekcji odrzucamy hipotezę zerową o losowości w położeniu punktów.

Należy więc stwierdzić, że nie ma podstaw do twierdzenia, że w Polsce istnieją
jakieś globalne klastry przestrzenne innowacyjności związanej z patentowaniem.
Mimo to, w przypadku włókiennictwa i papiernictwa (D) oraz 
budownictwa i górnictwa (E) można mówić o skupieniu patentów w przestrzeni
na poziomie krajowym.




    \newpage\section
  {Klastry przestrzenne z uwzględnieniem klasyfikacji \ac{IPC}}

  \figside
{../fig/clst/M-kmeans.png}
{Rozrzut przestrzenny puntków w klastrach}
{
Metodą $k$-średnich dla $k=$ wyznaczamy klastry przestrzenne. 
Cechami branymi pod uwagę są współrzędne przestrzenne punktów oraz
udział klasyfikacji \ac{IPC} w klastrach (\cref{udział-klasyfikacji}).
}

\tblside
{../fig/clst/T-kmeans-clsf.png}
{Udział klasyfikacji \ac{IPC} w klastrach}

\tblside
{../fig/clst/T-kmeans-meandist.png}
{Średnie odległości między punktami w klastrach}




    \newpage\section{Dynamika czasowa}

  \figside
{../fig/endo/F-pat-n-woj.png}
{ Liczba patentów, które otrzymały ochronę w poszczególnych województwach 
  w kolejnych latach }

  \figsides
{../fig/endo/M-woj-13.png}
{ Mapa rozrzutu geolokalizacji osób pełniących role patentowe 
  przy patentach, które otrzymały ochronę w 2013 roku}
{../fig/endo/M-woj-dt-14.png}
{ Mapa rozrzutu geolokalizacji osób pełniących role patentowe, 
  przy patentach, które otrzymały ochronę w 2014 roku}

  \newpage\figsides
{../fig/endo/M-woj-dt-15.png}{Mapa zmian w 2015}
{../fig/endo/M-woj-dt-16.png}{Mapa zmian w 2016}

  \figsidesTri
{../fig/endo/M-woj-dt-17.png}{2017}
{../fig/endo/M-woj-dt-18.png}{2018}
{../fig/endo/M-woj-dt-19.png}{2019}

  \figsidesTri
{../fig/endo/M-woj-dt-20.png}{2020}
{../fig/endo/M-woj-dt-21.png}{2021}
{../fig/endo/M-woj-dt-22.png}{2022}



  \newpage



  \newpage\subsection
{Analiza trendu}



  \fig
{../fig/endo/F-Q.png}{Ilość patentów w zależności od kwartału przyznania ochrony}

  \fig
{../fig/endo/F-mo.png}{Ilość patentów w zależności od miesiąca przyznania ochrony}





  \newpage\subsubsection
{Sezonowość}

  \figsides
{../fig/endo/F-pat-n-woj-Q.png}
{ Liczba patentów, które otrzymały ochronę w poszczególnych województwach 
  --- sumy kwartalne z lat 2013-2022 }
{../fig/endo/F-pat-n-woj-mo.png}
{ Liczba patentów, które otrzymały ochronę w poszczególnych województwach 
  --- sumy miesięczne z lat 2013-2022 }