\section{Klastry przestrzenne z uwzględnieniem klasyfikacji \ac{IPC}}

Metodą $k$-średnich dla $k=$ wyznaczamy klastry przestrzenne. 
Cechami branymi pod uwagę są współrzędne przestrzenne punktów oraz
udział klasyfikacji \ac{IPC} w klastrach. Udział klasyfikacji
odnosi się do tego jak wiele osób dostawało ochronę patentową
w danej sekcji ze wskazanego punktu, bądź współpracowało przy
takim dokumencie z innymi osobami.

\begin{multicols}{2}

\halffig{../fig/M-cluster.endo.png}
{Rozrzut przestrzenny puntków w klastrach}

\columnbreak

\halftable{../fig/T-cluster-clsf.endo.png}
{Udział klasyfikacji \ac{IPC} w klastrach}

\halftable{../fig/T-cluster-meandist.endo.png}
{Średnie odległości między punktami w klastrach}

\end{multicols}