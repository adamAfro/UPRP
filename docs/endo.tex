\section{Dynamika czasowa}

\begin{multicols}{2}

Zmienność danych geolokalizacyjnych jest niewidoczna w ciągu badanego okresu, co obrazuje
wykres obok. W latach 2013-2022 nie widać znaczących zmian w rozrzucie źródeł
aplikacji patentowych.

\columnbreak
\halffig{../fig/M-13-22.subj.png}{
  Mapa rozrzutu geolokalizacji osób
  pełniących role patentowe
  w różnych okresach czasu
}

\end{multicols}



\section{Klastry przestrzenne z uwzględnieniem klasyfikacji \ac{IPC}}

Metodą $k$-średnich dla $k=$ wyznaczamy klastry przestrzenne. 
Cechami branymi pod uwagę są współrzędne przestrzenne punktów oraz
udział klasyfikacji \ac{IPC} w klastrach. Udział klasyfikacji
odnosi się do tego jak wiele osób dostawało ochronę patentową
w danej sekcji ze wskazanego punktu, bądź współpracowało przy
takim dokumencie z innymi osobami.

\begin{multicols}{2}

\halffig{../fig/M-cluster.endo.png}
{Rozrzut przestrzenny puntków w klastrach}

\columnbreak

\halftable{../fig/T-cluster-clsf.endo.png}
{Udział klasyfikacji \ac{IPC} w klastrach}

\halftable{../fig/T-cluster-meandist.endo.png}
{Średnie odległości między punktami w klastrach}

\end{multicols}



\section{Obszary peryferyjne}

\begin{multicols}{2}

\halffig{../fig/map.subj.png}
{Mapa rozrzutu geolokalizacji osób pełniących role patentowe}

\halffig{../fig/M-meandist100.subj.png}
{Mapa średniej odległości do innych osób pełniących role patentowe do 100 km}

\columnbreak

\halffig{../fig/M-meandist.subj.png}
{Mapa średniej odległości do innych osób pełniących role patentowe}

\halffig{../fig/M-meandist50.subj.png}
{Mapa średniej odległości do innych osób pełniących role patentowe do 50 km}

\end{multicols}

\begin{multicols}{2}

\halftable{../fig/T-meandist.subj.png}{Statystyki dotyczące średnich odległości}

\begin{uwaga}
Średnia odległość po między geolokalizacjami patentowymi jest ważona ---
ilość patentów pochodzących z danej geolokalizacji jest wagą tego punktu.
\end{uwaga}

\end{multicols}

Obszary peryferyjne to miejsca odległe od centrów, w tym przypadku
głównych źródeł aplikacji patentowych. Wyznaczamy je poprzez obliczenie
średniego dystansu do innych osób pełniących role patentowe.
Oprócz dystansu do każdego innego punktu, warto jest ograniczyć
tę statystykę do pewnego obszaru. Jak wcześniej zaznaczono, Polska
nie jest jednorodna przestrzennie pod tym względem, stąd peryferyjność
nie może być rozumiana tylko na krajowym poziomie, ale również lokalnym.

Rysunek \ref{../fig/M-meandist.subj.png} przedstawia średni dystans
do innych osób pełniących role patentowe w Polsce. Widać, że Warszawa
jest najbardziej centralnym punktem, co jest zgodne z wcześniejszymi
obserwacjami. To samo dotyczy okręgu Katowic. Na podstawie tej statystyki 
należy przypuszczać, że są to regiony o względnej bliskości 
do innych źródeł patentowych na poziomie krajowym.

Na rysunkach \ref{../fig/M-meandist100.subj.png} oraz
\ref{../fig/M-meandist50.subj.png} przedstawiono ten sam
wskaźnik, ale ograniczony do obszaru o promieniu odpowiednio 100 i 50 kilometrów.
\todonote{potrz. wyjaśn.}