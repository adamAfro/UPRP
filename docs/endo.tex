\section{Dynamika czasowa}

  \figsides
{../fig/endo/M-13-22.png}
{ Mapa rozrzutu geolokalizacji osób 
  pełniących role patentowe w kolejnych latach }
{../fig/endo/F-pat-n-woj.png}
{ Ilość patentów z danego województwa w kolejnych latach }


\newpage
\section{Klastry przestrzenne z uwzględnieniem klasyfikacji \ac{IPC}}

\figside
  {../fig/endo/M-cluster.png}
    {Rozrzut przestrzenny puntków w klastrach}
{
  Metodą $k$-średnich dla $k=$ wyznaczamy klastry przestrzenne. 
  Cechami branymi pod uwagę są współrzędne przestrzenne punktów oraz
  udział klasyfikacji \ac{IPC} w klastrach. Udział klasyfikacji
  odnosi się do tego jak wiele osób dostawało ochronę patentową
  w danej sekcji ze wskazanego punktu, bądź współpracowało przy
  takim dokumencie z innymi osobami.
}

\tblside
  {../fig/endo/T-cluster-clsf.png}
    {Udział klasyfikacji \ac{IPC} w klastrach}

\tblside
  {../fig/endo/T-cluster-meandist.png}
    {Średnie odległości między punktami w klastrach}



\newpage
\section{Obszary peryferyjne}

Obszary peryferyjne to miejsca odległe od centrów, w tym przypadku
głównych źródeł aplikacji patentowych. Wyznaczamy je poprzez obliczenie
średniego dystansu do innych osób pełniących role patentowe.
Oprócz dystansu do każdego innego punktu, warto jest ograniczyć
tę statystykę do pewnego obszaru. Polska nie jest jednorodna przestrzennie 
pod tym względem, stąd peryferyjność nie może być rozumiana tylko 
na krajowym poziomie, ale również lokalnym.

\figsides
  {../fig/endo/M.png}
    {Mapa rozrzutu geolokalizacji osób pełniących role patentowe}
  {../fig/endo/M-meandist100.png}
    {Mapa średniej odległości do innych osób pełniących role patentowe do 100 km}

\figsides
  {../fig/endo/M-meandist.png}
    {Mapa średniej odległości do innych osób pełniących role patentowe}
  {../fig/endo/M-meandist50.png}
    {Mapa średniej odległości do innych osób pełniących role patentowe do 50 km}

\newpage
\tblside{../fig/endo/T-meandist.png}{Statystyki dotyczące średnich odległości}
{
  \begin{uwaga}
  Średnia odległość po między geolokalizacjami patentowymi jest ważona ---
  ilość patentów pochodzących z danej geolokalizacji jest wagą tego punktu.
  \end{uwaga}
}

\figsidesTri
  {../fig/endo/F-meandist.png}
    {Histogram średnich odległości do innych osób pełniących role patentowe}
  {../fig/endo/F-meandist50.png}
    {Histogram średnich odległości do innych osób pełniących role patentowe}
  {../fig/endo/F-meandist100.png}
    {Histogram średnich odległości do innych osób pełniących role patentowe}



\newpage

\section{Połączenia opracowane npdst. raportów o stanie techniki}

\figpage{0.8}
  {../fig/endo/M-rprtdist.png}
    {Połączenia opracowane npdst. raportów o stanie techniki}

