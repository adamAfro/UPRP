    \newpage\section
  {Obszary peryferyjne}

Obszary peryferyjne to miejsca odległe od centrów, w tym przypadku
głównych źródeł aplikacji patentowych. Wyznaczamy je poprzez obliczenie
średniego dystansu do innych osób pełniących role patentowe.
Oprócz dystansu do każdego innego punktu, warto jest ograniczyć
tę statystykę do pewnego obszaru. Polska nie jest jednorodna przestrzennie 
pod tym względem, stąd peryferyjność nie może być rozumiana tylko 
na krajowym poziomie, ale również lokalnym.

\begin{uwaga}
Średnia odległość po między geolokalizacjami patentowymi jest ważona ---
ilość patentów pochodzących z danej geolokalizacji jest wagą tego punktu.
\end{uwaga}

  \figsides
{../fig/endo/M.png}
{ Mapa rozrzutu geolokalizacji osób pełniących role patentowe }
{../fig/endo/M-meandist100.png}
{ Mapa średniej odległości do innych osób pełniących role patentowe do 100 km }

\figsides
{../fig/endo/M-meandist.png}
{ Mapa średniej odległości do innych osób pełniących role patentowe }
{../fig/endo/M-meandist50.png}
{ Mapa średniej odległości do innych osób pełniących role patentowe do 50 km }

\figsidesTri
{../fig/endo/F-meandist.png}
{ Histogram średnich odległości do innych osób pełniących role patentowe }
{../fig/endo/F-meandist50.png}
{ Histogram średnich odległości do innych osób pełniących role patentowe w promieniu 50 km }
{../fig/endo/F-meandist100.png}
{ Histogram średnich odległości do innych osób pełniących role patentowe w promieniu 100 km }




    \newpage\section
  {Połączenia opracowane npdst. raportów o stanie techniki}

\figpage{0.8}
{../fig/endo/M-rprtdist.png}
{ Połączenia opracowane npdst. raportów o stanie techniki }


  \newpage
  \figside
{../fig/endo/F-rprt-meandist.png}
{ Histogram pionowy odległości między punktami, które są 
  połączone opracowanymi na podstawie raportów o stanie techniki }




    \newpage\section
  {Klastry przestrzenne z uwzględnieniem klasyfikacji \ac{IPC}}

\figside
  {../fig/endo/M-cluster.png}
    {Rozrzut przestrzenny puntków w klastrach}
{
  Metodą $k$-średnich dla $k=$ wyznaczamy klastry przestrzenne. 
  Cechami branymi pod uwagę są współrzędne przestrzenne punktów oraz
  udział klasyfikacji \ac{IPC} w klastrach. Udział klasyfikacji
  odnosi się do tego jak wiele osób dostawało ochronę patentową
  w danej sekcji ze wskazanego punktu, bądź współpracowało przy
  takim dokumencie z innymi osobami.
}

\tblside
  {../fig/endo/T-cluster-clsf.png}
    {Udział klasyfikacji \ac{IPC} w klastrach}

\tblside
  {../fig/endo/T-cluster-meandist.png}
    {Średnie odległości między punktami w klastrach}




    \newpage\section{Dynamika czasowa}

  \figsides
{../fig/endo/M-woj-13.png}
{ Mapa rozrzutu geolokalizacji osób pełniących role patentowe 
  przy patentach, które otrzymały ochronę w 2013 roku}
{../fig/endo/M-woj-dt-14.png}
{ Mapa rozrzutu geolokalizacji osób pełniących role patentowe, 
  przy patentach, które otrzymały ochronę w 2014 roku}

  \figside
{../fig/endo/F-pat-n-woj.png}
{ Liczba patentów, które otrzymały ochronę w poszczególnych województwach 
  w kolejnych latach }

  \newpage\figsides
{../fig/endo/M-woj-dt-15.png}{Mapa zmian w 2015}
{../fig/endo/M-woj-dt-16.png}{Mapa zmian w 2016}

  \figsidesTri
{../fig/endo/M-woj-dt-17.png}{2017}
{../fig/endo/M-woj-dt-18.png}{2018}
{../fig/endo/M-woj-dt-19.png}{2019}

  \figsidesTri
{../fig/endo/M-woj-dt-20.png}{2020}
{../fig/endo/M-woj-dt-21.png}{2021}
{../fig/endo/M-woj-dt-22.png}{2022}



  \newpage\subsection
{Analiza trendu}



  \subsubsection
{Stacjonarność}

  \tblside
{../fig/endo/T-statio-woj.png}
{ Wyniki testów ADF oraz KPSS w Polsce i województwach}



  \fig
{../fig/endo/F-Q.png}{Ilość patentów w zależności od kwartału przyznania ochrony}

  \fig
{../fig/endo/F-mo.png}{Ilość patentów w zależności od miesiąca przyznania ochrony}



  \newpage\subsubsection
{Sezonowość}

  \figsides
{../fig/endo/F-pat-n-woj-Q.png}
{ Liczba patentów, które otrzymały ochronę w poszczególnych województwach 
  --- sumy kwartalne z lat 2013-2022 }
{../fig/endo/F-pat-n-woj-mo.png}
{ Liczba patentów, które otrzymały ochronę w poszczególnych województwach 
  --- sumy miesięczne z lat 2013-2022 }